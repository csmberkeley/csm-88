\begin{blocksection}
\question \textbf Write the \texttt{help} function in the Mentor class. When this function is called, the mentor should help students in the queue that need help (the queue is a linked list). If a student does not need help, the mentor should move on to the next student. See the sample use case at the bottom for an example of how the \texttt{help} method should work!

\begin{lstlisting}
class Mentee:
    def __init__(self, name, needs_help):
        self.name = name
        self.needs_help = needs_help

class Mentor:
    def __init__(self, name):
        self.name = name

    def help(self, students):
        ## TO DO

chi = Mentee("Chi", True)
mimi = Mentee("Mimi", False)
gilbert = Mentee("Gilbert", True)
ada = Mentee("Ada", True)

tony = Mentor("Tony")
students = Link(chi, Link(mimi, Link(gilbert, Link(ada))))

tony.help(students)
>>> Tony helped Chi!
>>> Tony helped Gilbert!
>>> Tony helped Ada!

tony.help(students)
## No one needs help anymore, so nothing should be printed!
\end{lstlisting}
\end{blocksection}
\begin{lstlisting}
def help(self, students):
    
\end{lstlisting}
\begin{blocksection}
\begin{solution}
\begin{lstlisting}
    if students is Link.empty:
        return
    else:
        curr = students.first
        if curr.needs_help:
            print(self.name + " helped " + curr.name + "!")
            curr.needs_help = False
        self.help(students.rest)
\end{lstlisting}
\end{solution}
\end{blocksection}