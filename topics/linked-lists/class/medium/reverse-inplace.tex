\begin{blocksection}
\question \textbf{(Optional)} Now write \texttt{reverse} by modifying the existing Links. Assume reverse returns the head of the new list (so the last \texttt{Link} object of the previous list).

First, draw out the box and pointer for the following:
\begin{lstlisting}
>>> a = Link(1, Link(2))
>>> a.rest.rest = a
>>> a.rest = Link.empty
\end{lstlisting}
Observe how the pointers change, as well as the order in which they are
modified.
\begin{solution}[0.6in]
\begin{lstlisting}
   +---+---+  +---+---+
+->| 1 | / |  | 2 | --|--+
|  +---+---+  +---+---+  |
|                        |
+------------------------+
\end{lstlisting}
\end{solution}
\end{blocksection}

\begin{blocksection}
Now, generalize this to reverse an entire linked list.
\begin{lstlisting}
def reverse(lst):
    """
    >>> a = Link(1, Link(2, Link(3)))
    >>> b = reverse(a)
    >>> b
    Link(3, Link(2, Link(1)))
    >>> a
    Link(3, Link(2, Link(1)))
    """
\end{lstlisting}
\begin{solution}[1.25in]
Here are three possible solutions.
\begin{lstlisting}
def reverse(lst):
    if lst == Link.empty or lst.rest == Link.empty:
        return lst
    else:
        new_start = reverse(lst.rest)
        lst.rest.rest = lst
        lst.rest = Link.empty
        return new_start
 
def reverse(lst):
    if lst is Link.empty or lst.rest is Link.empty:
        return lst
    secondElement = lst.rest
    lst.rest = Link.empty
    reversedRest = reverse(secondElement)
    secondElement.rest = lst
    return reversedRest
        
def reverse(lst):
    if lst.rest is not Link.empty:
        second, last = lst.rest, lst
        lst = reverse(second)
        second.rest, last.rest = last, Link.empty
    return lst
\end{lstlisting}
\end{solution}


\end{blocksection}