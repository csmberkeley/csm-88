\begin{blocksection}
\question What will Python output? Draw box-and-pointer diagrams to help determine this.

\begin{lstlisting}
>>> a = Link(1, Link(2, Link(3)))
\end{lstlisting}
\begin{solution}[0in]
\begin{lstlisting}
+---+---+  +---+---+  +---+---+
| 1 | --|->| 2 | --|->| 3 | / |
+---+---+  +---+---+  +---+---+
\end{lstlisting}
\end{solution}

\begin{lstlisting}
>>> a.first
\end{lstlisting}
\begin{solution}[.25in]
\begin{lstlisting}
1
\end{lstlisting}
\end{solution}

\begin{lstlisting}
>>> a.first = 5
\end{lstlisting}
\begin{solution}[0in]
\begin{lstlisting}
+---+---+  +---+---+  +---+---+
| 5 | --|->| 2 | --|->| 3 | / |
+---+---+  +---+---+  +---+---+
\end{lstlisting}
\end{solution}

\begin{lstlisting}
>>> a.first
\end{lstlisting}
\begin{solution}[.25in]
5
\end{solution}

\begin{lstlisting}
>>> a.rest.first
\end{lstlisting}
\begin{solution}[.25in]
2
\end{solution}

\begin{lstlisting}
>>> a.rest.rest.rest.rest.first
\end{lstlisting}
\begin{solution}[.25in]
Error: tuple object has no attribute rest (Link.empty has no rest)
\end{solution}
\end{blocksection}

\begin{blocksection}
\begin{lstlisting}
>>> a.rest.rest.rest = a
\end{lstlisting}
\begin{solution}[0in]
\begin{lstlisting}
   +---+---+  +---+---+  +---+---+
+->| 5 | --|->| 2 | --|->| 3 | --|--+
|  +---+---+  +---+---+  +---+---+  |
|                                   |
+-----------------------------------+
\end{lstlisting}
\end{solution}

\begin{lstlisting}
>>> a.rest.rest.rest.rest.first
\end{lstlisting}
\begin{solution}[.25in]
\begin{lstlisting}
2
\end{lstlisting}
\end{solution}

\end{blocksection}
