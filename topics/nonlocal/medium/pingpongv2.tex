\begin{blocksection}
\question \textbf{Pingpong again...}\\
\begin{nonsol}
The ping-pong sequence counts up starting from 1 and is
always either counting up or counting down.

At element k, the direction switches if k is a multiple of 7 or contains the
digit 7.

The first 30 elements of the ping-pong sequence are listed below, with direction
swaps marked using brackets at the 7th, 14th, 17th, 21st, 27th, and 28th
elements:
\begin{lstlisting}
1 2 3 4 5 6 [7] 6 5 4 3 2 1 [0] 1 2 [3] 2 1 0 [-1] 0 1 2 3 4
[5] [4] 5 6
\end{lstlisting}
\end{nonsol}

Implement a function \texttt{make\char`_pingpong\char`_tracker} that returns the
next value in the pingpong sequence each time it is called. In the body of
\texttt{make\char`_pingpong\char`_tracker}, you can use assignment statements.
\newline

\begin{lstlisting}
def has_seven(k): # Use this function for your answer below
    if k % 10 == 7:
        return True
    elif k < 10:
        return False
    else:
        return has_seven(k // 10)
\end{lstlisting}

\begin{nonsol}
\begin{lstlisting}
def make_pingpong_tracker():
    """ Returns a function that returns the next value in the
    pingpong sequence each time it is called.
    >>> output = []
    >>> x = make_pingpong_tracker()
    >>> for _ in range(9):
    ... output += [x()]
    >>> output
    [1, 2, 3, 4, 5, 6, 7, 6, 5]
    """
    index, current, add = 1, 0, True
    def pingpong_tracker():
        ___________________________
        if add:
            ________________________
        else:
            ________________________
        if _______________________:
            add = not add
        __________________________
        __________________________
    return pingpong_tracker

\end{lstlisting}
\end{nonsol}

\begin{solution}
\begin{lstlisting}
def make_pingpong_tracker():
    index, current, add = 1, 0, True
    def pingpong_tracker():
        nonlocal index, current, add
        if add:
            current = current + 1
        else:
            current = current - 1
        if has_seven(index) or index % 7 == 0:
            add = not add
        index += 1
        return current
    return pingpong_tracker
\end{lstlisting}
\end{solution}

\end{blocksection}

\begin{blocksection}
\question \textbf{(Optional)} Instead of using nonlocal for pingpong, let's use OOP!

\begin{nonsol}
\begin{lstlisting}
>>> tracker1 = PingPongTracker()
>>> tracker2 = PingPongTracker()
>>> tracker1.next()
1
>>> tracker1.next()
2
>>> tracker2.next()
1

class PingPongTracker:
    def __init__(self):
        self.current = 0
        self.index = 1
        self.add = True
    def next(self):
        """*** Enter solution below ***"""
\end{lstlisting}
\end{nonsol}

\begin{solution}[0.3in]
\begin{lstlisting}
class PingPongTracker:
    def __init__(self):
        self.current = 0
        self.index = 1
        self.add = True

    def next(self):
        if self.add:
            self.current += 1
        else:
            self.current -= 1
        if has_seven(self.index) or self.index % 7 == 0:
            self.add = not self.add
        self.index += 1
        return self.current

\end{lstlisting}
Notice how the OOP approach is insanely similar to the non local function.
Instead of using \texttt{nonlocal}, we use \texttt{self.varName} and the code
becomes exactly the same. We just store the data in a slightly different way.
This implies that OOP and functions are pretty similar, and it turns out you can
even write your own OOP framework using just functions and \texttt{nonlocal}!

In addition, there are a lot of python specific features that can be written
using functions or using classes. If you are interested, check out the powerful
python feature decorators, and note how we can write them both as functions and
as classes!
\end{solution}

\end{blocksection}
