\begin{blocksection}
\question The following is an \textbf{Abstract Data Type (ADT)} for colleges.
Each college has a name, and population of students. 

Our college ADT has one \textbf{constructor}: 

\lstinline{make_college(name, population)}: Creates a college object with given name and student population. 

Our college ADT has two \textbf{selectors}: 

\lstinline{get_name(col)}: Return the college's name. 

\lstinline{get_pop(col)}: Return the college's student population. 

Now implement the function \lstinline{student_diff} which computes the difference between the student population sizes of two colleges. Make sure the number you return is nonnegative!


\begin{lstlisting}
def student_diff(col1, col2):
    """
    >>> col1 = make_college('Hogwarts', 1500)
    >>> col2 = make_college('Brakebills', 950)
    >>> student_diff(col1, col2)
    550
    >>> student_diff(col2, col1)
    550
    """
\end{lstlisting}
\begin{solution}[1in]
\begin{lstlisting}
    return abs(get_pop(col1) - get_pop(col2))

\end{lstlisting}
\end{solution}

\end{blocksection}

%%% Question %%%

\begin{blocksection}
\question Next, implement \lstinline{similar_size}, a function that takes in a number of students and two colleges, and returns the name of the college that has the closest number of students. Hint: You can use the previos method, but that one takes in colleges, so you might need to make your own college.
 
\begin{lstlisting}
def similar_size(pop, col1, col2):
    """
    >>> berkeley = make_college('Berkeley', 40000)
    >>> stanford = make_college('Stanford', 20000)
    >>> similar_size(35000, berkeley, stanford)
    'Berkeley'
    """
\end{lstlisting}
\begin{solution}[1in]
\begin{lstlisting}
    temp = make_college('Temporary University', pop)
    if student_diff(temp, col1) < student_diff(temp, col2):
        return get_name(col1)
    else:
        return get_name(col2)
\end{lstlisting}
\end{solution}

\end{blocksection}

