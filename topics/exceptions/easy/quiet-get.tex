\begin{blocksection}
\question You have seen that indexing a list with an index that is not contained in the list generates and exception, as does looking up a key that does not exist in a dictionary. However, the \texttt{get} method of \texttt{dict} is more forgiving. If the key is not in the dictionary it returns a value that you provide, defaulting to None. Use exception handling in the function \texttt{quiet\_get} to obtain similar behavior for both lists and dictionaries.

\begin{lstlisting}
def quiet_get(data, selector, missing=None):
    """Return data[selector] if it exists, otherwise missing.
    >>> quiet_get([1,2,3], 1)
    2
    >>> quiet_get([1,2,3], 4)
    >>> quiet_get({'a':2, 'b':5}, 'a', -1)
    2
    >>> quiet_get({'a':2, 'b':5}, 'd', -1)
    -1
    """
\end{lstlisting}
\begin{solution}
\begin{lstlisting}
    try:
        return data[selector]
    except (KeyError, IndexError):
        return missing
\end{lstlisting}
\end{solution}
\end{blocksection}