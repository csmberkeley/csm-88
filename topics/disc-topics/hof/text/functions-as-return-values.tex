Often, we will need to write a function that returns another function. One way
to do this is to define a function \texttt{inner} inside of a function \texttt{outer}, and \texttt{outer} will return the function \texttt{inner}. 

Some cases where we might do this is:
\begin{itemize}
    \item need additional information (in the example below, we needed information of the \texttt{name} of whom to greet)
    \item might need to track other variables that aren't included
\end{itemize}
\vspace{2mm}

\begin{lstlisting}[language = Python]
def maker-greeter(greeting):
    def greet(name):
        print(greeting, name)
    return greet

>>> hello-greeter = make-greeter("Hello")
>>> greet("Alina")
Hello Alina
\end{lstlisting}


% Often, we will need to write a function that returns another function. One way
% to do this is to define a function inside of a function:



% The return value of \texttt{outer} is the function \texttt{inner}. This is a
% case of a function returning a function. In this example, \texttt{inner} is
% defined inside of \texttt{outer}. Although this is a common pattern, we can also
% define \texttt{inner} outside of \texttt{outer} and still use the same
% \texttt{return} statement. However, note that in this second example (unlike the
% first example), \texttt{inner} doesn't have access to variables defined within
% the \texttt{outer} function, like \texttt{x}.

