A lambda expression evaluates to a function, called a lambda function. For example, \texttt{lambda x, y: x + y} is a lambda expression, and can be read as “a function that takes in two parameters \texttt{x} and \texttt{y} returns \texttt{x + y}.” 

A lambda expression by itself evaluates to a function but does not bind it to a name. Also note that the return expression of this function is not evaluated until the lambda is called. This is similar to how defining a new function using a def statement does not execute the function’s body until it is later called. 

\begin{lstlisting}[linewidth=\textwidth]
>>> what = lambda x : x + 5
>>> what
<function <lambda> at 0xf3f490>
\end{lstlisting}


Unlike \texttt{def} statements, lambda expressions can be used as an operator or an operand to a call expression. This is because they are simply one-line expressions that evaluate to functions.

\begin{lstlisting}[linewidth=\textwidth]
>>> (lambda y: y + 5)(4) 
9
>>> (lambda f, x: f(x))(lambda y: y + 1, 10)
11
\end{lstlisting}