\question
Write a function that takes in two single-argument functions, \textbf{f} and \textbf{g}, and returns another \textbf{function} that has a single paramater \textbf{x}. The returned function should return \textbf{True} if \textbf{f(g(x))} is equal to \textbf{g(f(x))}. You can assume the output of \textbf{g(x)} is a valid input for \textbf{f} and vice versa. You may use the \textbf{compose1} function defined below.
\begin{lstlisting}[language=Python]
def compose1(f, g):
    """Return the composition function which given x, computes f(g(x)).

    >>> add_one = lambda x: x + 1        # adds one to x
    >>> square = lambda x: x**2
    >>> a1 = compose1(square, add_one)   # (x + 1)^2
    >>> a1(4)
    25
    >>> mul_three = lambda x: x * 3      # multiplies 3 to x
    >>> a2 = compose1(mul_three, a1)    # ((x + 1)^2) * 3
    >>> a2(4)
    75
    >>> a2(5)
    108
    """
    return lambda x: f(g(x))

def composite_identity(f, g):
    """
    Return a function with one parameter x that returns True if f(g(x)) is
    equal to g(f(x)). You can assume the result of g(x) is a valid input for f
    and vice versa.

    >>> add_one = lambda x: x + 1        # adds one to x
    >>> square = lambda x: x**2
    >>> b1 = composite_identity(square, add_one)
    >>> b1(0)                            # (0 + 1)^2 == 0^2 + 1
    True
    >>> b1(4)                            # (4 + 1)^2 != 4^2 + 1
    False
    """
\end{lstlisting}
\begin{solution}[1in]
\begin{lstlisting}
    def identity(x):
        return compose1(f, g)(x) == compose1(g, f)(x)
    return identity

    # Alternative solution
    return lambda x: f(g(x)) == g(f(x))

    # Video Walkthrough: https://youtu.be/jVhZJ5TL4zc
\end{lstlisting}
\end{solution}
