\question
Determine the number of calls to \textbf{scheme\_eval} and the number of calls to \textbf{scheme\_apply} for the following expressions. Use the visualizer at \url{code.cs61a.org} if you're not sure how an expression is evaluated.

\begin{lstlisting}[language=Scheme]
> (+ 1 2)
3
\end{lstlisting}
\begin{solution}[1in]
4 calls to eval: \\
1. Evaluate entire expression \\
2. Evaluate + \\
3. Evaluate 1 \\
4. Evaluate 2

1 call to apply: \\
1. Apply + to 1 and 2 \\
\end{solution}

\begin{lstlisting}[language=Scheme]
> (if 1 (+ 2 3) (/ 1 0))
5
\end{lstlisting}
\begin{solution}[1in]
6 calls to eval: \\
1. Evaluate entire expression \\
2. Evaluate predicate 1 \\
3. Since 1 is true, evaluate the entire sub-expression ( + 2 3) \\
4. Evaluate + \\
5. Evaluate 2 \\
6. Evaluate 3 \\

1 call to apply: \\
1. Apply + to 2 and 3 \\
\end{solution}

\begin{lstlisting}[language=Scheme]
> (or #f (and (+ 1 2) 'apple) (- 5 2))
apple
\end{lstlisting}
\begin{solution}[1in]
8 calls to eval: \\
1. Evaluate entire expression \\
2. Evaluate false \\
3. Evaluate entire sub-expression (and  (+ 1 2) 'apple) \\
4. Evaluate entire sub-expression (+ 1 2) \\
5. Evaluate + \\
6. Evaluate 1 \\
7. Evaluate 2 \\
8. Evaluate 'apple \\
Since the and expression evaluates to true, we short circuit here. \\
1 call to apply: \\
1. Apply + to 2 and 3 \\
\end{solution}

\begin{lstlisting}[language=Scheme]
> (define (add x y) (+ x y))
add
> (add (- 5 3) (or 0 2))
2
\end{lstlisting}

\begin{solution}[1in]
13 calls to eval: \\
1. Evaluate entire define expression \\
2. Evaluate entire (add ...) expression \\
3. Evaluate add operator \\
4. Evaluate entire (- 5 3) sub-expression \\
5. Evaluate - \\
6. Evaluate 5 \\
7. Evaluate 3 \\
8. Evaluate entire (or 0 2) sub-expression \\
9. Evaluate 0 (and short circuit, since 0 is truthy in Scheme\\
10. Evaluate (+ x y) (after applying add and entering the body of the add function) \\
11. Evaluate + \\
12. Evaluate x \\
13. Evaluate y \\

2 call to apply: \\
1. Apply - to 5 and 3 \\
2. Apply + to -2 and 0 \\
\end{solution}


