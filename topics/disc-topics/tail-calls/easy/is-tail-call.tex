\question
For each of the following functions, identify whether it contains a recursive
call in a tail context. Also indicate if it uses a constant number of frames.

% \begin{minipage}{\linewidth}
% \begin{lstlisting}[language=Scheme]
% (define (question-a x)
%   (if (= x 0) 0
%       (+ x (question-a (- x 1)))))
% \end{lstlisting}
% \end{minipage}
% \begin{solution}[0in]
% In the recursive case, the last expression that is evaluated is a call to
% \texttt{+}. Therefore, the recursive call is not in tail context, and each of
% the frames remain active. This function uses $\Theta(n)$ frames.
% \end{solution}

\begin{minipage}{\linewidth}
\begin{lstlisting}[language=Scheme]
(define (question-b x y)
  (if (= x 0) y
      (question-b (- x 1) (+ y x))))
\end{lstlisting}
\end{minipage}
\begin{solution}[0in]
The recursive call is the third operand in the if expression, so it is in tail
context. This means that the last expression that will be evaluated in the body
of this function is the recursive function call, so this function uses
$\Theta(1)$ frames.
\end{solution}

\begin{minipage}{\linewidth}
\begin{lstlisting}[language=Scheme]
(define (question-c x y)
  (if (> x y)
      (question-c (- y 1) x)
      (question-c (+ x 10) y)))
\end{lstlisting}
\end{minipage}
\begin{solution}[0in]
The recursive calls are the second and third operands of the \texttt{if}
expression. Only one of these calls is actually evaluated, and whichever one it
is will be the last expression evaluated in the body of the function.  This
function therefore uses $\Theta(1)$ frames.

Note that if you actually try and evaluate this function, it will never
terminate. But at least it won't crash from hitting max recursion depth!
\end{solution}

\begin{minipage}{\linewidth}
\begin{lstlisting}[language=Scheme]
(define (question-d n)
  (if (question-d n)
      (question-d (- n 1))
      (question-d (+ n 10))))
\end{lstlisting}
\end{minipage}
\begin{solution}[0in]
The second and third recursive calls are in tail context, but the first is not.
Since not all the recursive calls are tail calls, this function does not use a
constant number of frames.
\href{https://youtu.be/QbtKMreFRl8?t=9m}{Video walkthrough}
\end{solution}

\begin{minipage}{\linewidth}
\begin{lstlisting}[language=Scheme]
(define (question-e n)
  (cond ((= n 0) 1)
        ((question-e (- n 1)) (question-e (- n 2)))
        (else (begin (print 2) (question-e (- n 3))))))
\end{lstlisting}
\end{minipage}
\begin{solution}[0in]
The second and third recursive calls are the second expressions in a clause, so
they are in tail context. However, the first recursive call is not in tail
context. Since not all recursive calls are tail calls, this function is not
tail recursive and does not use a constant number of frames.
\end{solution}
