Scheme implements tail-call optimization, which allows programmers to write
recursive functions that use a constant amount of space. A \define{tail call}
occurs when a function calls another function as its \textbf{last action of the
current frame}. In this case, the frame is no longer needed, and we
can remove it from memory. In other words, if this is the last thing you are
going to do in a function call, we can reuse the current frame instead
of making a new frame.

Consider this implementation of \texttt{factorial}.

\begin{lstlisting}[language=Scheme]
(define (fact n)
  (if (= n 0)
      1
      (* n (fact (- n 1)))))
\end{lstlisting}

The recursive call occurs in the last line, but it is not the last expression
evaluated. After calling \texttt{(fact (- n 1))}, the function still needs to
multiply that result with \texttt{n}. The final expression that is evaluated is
a call to the multiplication function, not \texttt{fact} itself.  Therefore,
the recursive call is \textbf{not} a tail call.

We can rewrite this function using a helper function that remembers the
temporary product that we have calculated so far in each recursive step.

\begin{lstlisting}[language=Scheme]
(define (fact n)
  (define (fact-tail n result)
    (if (= n 0)
        result
        (fact-tail (- n 1) (* n result))))
  (fact-tail n 1))
\end{lstlisting}

\texttt{fact-tail} makes a single recursive call to \texttt{fact-tail}, and
that recursive call is the last expression to be evaluated, so it is a tail
call. Therefore, \texttt{fact-tail} is a tail recursive process. We say that a
recursive function is \define{tail recursive} if all of its recursive calls are
tail calls.

\subsection*{Using a constant number of frames}

Tail recursive processes can use a constant amount of memory because each
recursive call frame does not need to be saved.

Our original implementation of \texttt{fact} required the program to keep each
frame open because the last expression multiplies the recursive result with
\texttt{n}.  Therefore, at each frame, we need to remember the current value of
\texttt{n}.

In contrast, the tail recursive \texttt{fact-tail} does not require the
interpreter to remember the values for \texttt{n} or \texttt{result} in each
frame. Instead, we can just \emph{update} the value of \texttt{n} and
\texttt{result} of the current frame! Therefore, we can keep reusing a single
frame to complete this calculation.

\begin{blocksection}
\subsection*{Tail context}
When trying to identify whether a given function call within the body of a
function is a tail call, we look for whether the call expression is in
\textbf{tail context}.

Given that each of the following expressions is the last expression in 
the body of the function, the following expressions are tail contexts:
\begin{itemize}
\item the second or third operand in an \texttt{if} expression
\item any of the non-predicate sub-expressions in a \texttt{cond} expression
  (i.e. the second expression of each clause)
\item the last operand in an \texttt{and} or an \texttt{or} expression
\item the last operand in a \texttt{begin} expression's body
\item the last operand in a \texttt{let} expression's body
\end{itemize}

For example, in the expression \texttt{(begin (+ 2 3) (- 2 3) (* 2 3))},
\texttt{(* 2 3)} is a tail call because it is the last operand expression to be
evaluated.

\end{blocksection}
\begin{solution}[0.1in]
\href{https://youtu.be/QbtKMreFRl8?t=51s}{Video walkthrough}
\end{solution}
