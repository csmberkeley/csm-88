\newcommand{\growth}{\Theta}

When we talk about the efficiency of a function, we are often interested in the
following: as the size of the input grows, how does the runtime of the function
change? And what do we mean by ``runtime''?

\begin{itemize}
\item  \lstinline$square(1)$ requires one primitive operation: \lstinline$*$
    (multiplication). \lstinline$square(100)$ also requires one. No matter
    what input \lstinline$n$ we pass into \lstinline$square$, it always takes one
    operation.
    \begin{center}
    \begin{tabular}{ |c|c|c|c| }
        \hline
        input & function call & return value & number of operations\\
        \hline
        $1$   & square(1)     & $1\cdot 1$          & $1$          \\
        $2$   & square(2)     & $2\cdot 2$          & $1$          \\
        \vdots& \vdots        & \vdots              & \vdots       \\
        $100$ & square(100)   & $100\cdot 100$      & $1$          \\
        \vdots& \vdots        & \vdots              & \vdots       \\
        $n$   & square($n$)   & $n\cdot n$          & $1$          \\
        \hline
    \end{tabular}
    \end{center}
\item
    \lstinline$factorial(1)$ requires one multiplication, but
    \lstinline$factorial(100)$ requires 100 multiplications. As we increase the
    input size of n, the runtime (number of operations) increases linearly
    proportional to the input.
    \begin{center}
    \begin{tabular}{ |c|c|c|c| }
      \hline
      input & function call & return value & number of operations\\
      \hline
      $1$   & factorial(1)    & $1\cdot 1$                    & $1$   \\
      $2$   & factorial(2)    & $2\cdot 1\cdot 1$             & $2$   \\
      \vdots& \vdots          & \vdots                        & \vdots\\
      $100$ & factorial(100)  & $100\cdot 99\cdots 1\cdot 1$  & $100$ \\
      \vdots& \vdots          & \vdots                        & \vdots\\
      $n$   & factorial($n$)  & $n\cdot (n-1)\cdots 1\cdot 1$ & $n$   \\
      \hline
    \end{tabular}
    \end{center}
\end{itemize}
