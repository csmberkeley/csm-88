More generally, we can say that {\tt foo} is in some $\Theta(f(n))$ if there exist
some constants $k_1$ and $k_2$ such that
\begin{equation*}
 k_1 \cdot f(n) \leq R(n) \leq k_2 \cdot f(n)
\end{equation*}
for $n > N$, where $N$ is sufficiently large.

This is a mathematical definition of big Theta notation. To prove that {\tt foo}
is in $\Theta(f(n))$, we only need to find constants $k_1$ and $k_2$ where the
above holds.

Fortunately, in CS 61A, we're not that concerned with rigorous mathematical
proofs. (You'll get more details in CS 61B!) What we want you to develop in CS
61A is the intuition to reason out the orders of growth for certain functions.
