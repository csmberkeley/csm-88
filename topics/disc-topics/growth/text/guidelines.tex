Here are some general guidelines for finding the order of growth for the
runtime of a function:

\begin{itemize}
\item If the function is recursive or iterative, you can subdivide the problem
    as seen above:
    \begin{itemize}
        \item Count the number of recursive calls/iterations that will be made
            in terms of input size $n$.
        \item Find how much work is done per recursive call or iteration in
            terms of input size $n$.
    \end{itemize}
    The answer is usually the product of the above two, but be sure to pay
    attention to control flow!
\item If the function calls helper functions that are not constant-time, you
    need to take the runtime of the helper functions into consideration.
\item We can ignore constant factors. For example,  $\Theta(1000000n) =
    \Theta(n)$.
\item We can also ignore lower-order terms. For example, $\Theta(n^3 + n^2 + 4n
    + 399) = \Theta(n^3)$. This is because the $n^{3}$ term dominates as $n$
    gets larger.
\end{itemize}
