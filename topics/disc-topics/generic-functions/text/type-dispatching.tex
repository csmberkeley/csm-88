Revisiting the complex number example, we have:

\begin{lstlisting}
tags = {ComplexRI: 'com', ComplexMA: 'com', Rational: 'rat'}

def type_tag(x):
    return tags[type(x)]
\end{lstlisting}


\begin{lstlisting}
tags = {AlbertRecord: 'AW', MarkRecord: 'MM'}

def type_tag(x):
    return tags[type(x)]
\end{lstlisting}

Now {\tt type\_tag.tags} is a dictionary that associates data types
(specifically, a class name) with a key word that we can use to look up
the type tag.

Next, we can implement a generic add function:

\begin{lstlisting}
def add(z1, z2):
    types = (type_tag(z1), type_tag(z2))
    return add.implementations[types](z1, z2)

add.implementations = {}
add.implementations[('com', 'com')] = add_complex
add.implementations[('com', 'rat')] = add_complex_and_rational
add.implementations[('rat', 'com')] = lambda x, y:
     add_complex_and_rational(y, x)
add.implementations[('rat', 'rat')] = add_rational
\end{lstlisting}

So what happens when we call {\tt add(ComplexRI(2, 3), ComplexRI(4,
5))}?  Let's refer to the two complex numbers as {\tt z1} and {\tt z2}.
{\tt type\_tag} looks up the tag for each them and returns 'com' and
'com'. We then look up {\tt ('com', 'com')} in our table of supported
implementations of add and see that we should use {\tt add\_complex}.
We then invoke {\tt add\_complex(z1, z2)} which works without a hitch
because all the data types match up.
