\question
What is the definition of an iterable? What is the definition of an iterator? What is the definition of a generator? What built-in functions or keywords are associated with each. Give an example of each.

\begin{solution}[2in]
An iterable is any object that can be passed to the built-in iter function. In other words, an iterable is any object that can produce iterators. \\
An iterator is an object that provides sequential access to values one by one. Its contents can be accessed through the built-in next function, and it will signal there are no more values available with a StopIteration exception when next is called. \\
A generator object is an iterator, but it is created in a special way -- generator functions are defined as a function that yields its values instead of returning them. When generator functions are called, they return a generator object, which can then be used as an iterator.

\end{solution}

\question
Evaluate if each line is valid? If not, state the error and how you would fix it.

\begin{lstlisting}
>>> new_list = [2, 3, 6, 8, 8, 3]
>>> next(new_list)


>>> iter(new_list)[1]


>>> [x for x in iter(new_list)]


>>> for i in range(len(iter(new_list))):
... 		new_list.append(2)


\end{lstlisting}

\begin{solution}[1in]
\begin{verbatim}

A)
Error: new_list is an Iterable not Iterator
Fix:
>>> next(iter(new_list))
Output:
2

B)
Error, can't use indexing on an iterator
>>> new_list[1]
3

C)
[2, 3, 6, 8, 8, 3]

D)
Error, can’t call len on iterator object
>>> for i in range(len(new_list)):
	new_list.append(2)

\end{verbatim}
\end{solution}
\newpage

\question
What is the difference between these two statements?

\begin{lstlisting}
a.    def infinity1(start):
		while True:
			start = start + 1
			return start

b.    def infinity2(start):
		while True:
			start = start + 1
			yield start			
|\begin{solution}
(a)is a function since it uses a return statement. Even tho while True is always true, it will stop after the first iteration when it returns start.
On the other hand, (b) is a generator since it uses a yield statement. Since while True is always true, calling next will iterate once and yield start
\end{solution}|

What would python display?

>>> infinity1
|\begin{solution}
<Function>
\end{solution}|
>>> infinity2
|\begin{solution}
<Function>
\end{solution}|
>>> infinity1(2)
|\begin{solution}
3
\end{solution}|
>>> infinity2(2)
|\begin{solution}
<Generator Instance>
\end{solution}|
>>> x = infinity1(2)
|\begin{solution}
Nothing
\end{solution}|
>>> next(x)
|\begin{solution}
Error, cant call next on integer
\end{solution}|
>>> y = infinity2(2)
|\begin{solution}
Nothing
\end{solution}|
>>> next(y)
|\begin{solution}
3
\end{solution}|
>>> next(y)
|\begin{solution}
4
\end{solution}|
>>> next(infinity2(2))
|\begin{solution}
3	
\end{solution}|
\end{lstlisting}
\newpage



\question
They can't stop all of us!!! Write a function \textbf{generate\_constant} which, a generator function that repeatedly yields the same value forever.

\begin{lstlisting}
def generate_constant(x):
	"""A generator function that repeats the same value x forever.
	>>> area = generate_constant(51)
	>>> next(area)
	51
	>>> next(area)
	51
	>>> sum([next(area) for _ in range(100)])
	5100
	"""
	
	
	
	
	
	
\end{lstlisting}

\begin{solution}
\begin{verbatim}
while True: 
    yield x
 \end{verbatim}
\end{solution}




\question
4.2 Now implement  \textbf{black\_hole} , a generator that yields items in seq until one of them matches trap, in which case that value should be repeated yielded forever. You may assume that  \textbf{generate\_constant}  works. You may not index into or slice seq.

\begin{lstlisting}
def black_hole(seq, trap):
	"""A generator that yields items in SEQ until one of them matches TRAP, in which case that value 	should be repeatedly yielded forever.
	>>> trapped = black_hole([1, 2, 3], 2)
	>>> [next(trapped) for _ in range(6)]
	[1, 2, 2, 2, 2, 2]
	>>> list(black_hole(range(5), 7))
	[0, 1, 2, 3, 4]
	"""
\end{lstlisting}

\begin{solution} [.5in]
\begin{lstlisting}
for item in seq:
	if item == trap:
		yield from generate_constant(trap)
	else:
		yield item

 \end{lstlisting}
\end{solution}
\newpage




\question
What Would Python Display?

\begin{lstlisting}

>>> def weird_gen(x):
... 	if x % 2 == 0:
... 	  yield x * 2
>>> wg = weird_gen(2)
>>> next(wg)
>>> next(weird_gen(2))
|\begin{solution}
4
4
\end{solution}|

>>> next(wg)
|\begin{solution}
StopIteration
\end{solution}|

>>> def greeter(x):
...   while x % 2 != 0:
...     print('hi')
...     yield x
...     print('bye')
>>> greeter(5)
|\begin{solution}
<Generator Object>
\end{solution}|

>>> gen = greeter(5)
>>> g = next(gen)
|\begin{solution}
hi
\end{solution}|
>>> g = (g, next(gen))
>>> g


|\begin{solution}
bye
hi
(5, 5)
\end{solution}|
>>> next(gen)


|\begin{solution}
bye
hi
5
\end{solution}|
>>> next(g)


|\begin{solution}
Error, tuple is not iterator
\end{solution}|
An iterator ______________________ a generator
A generator is a(n)  ______________________ iterator
|\begin{solution}
An iterator is not always represented by  a generator
A generator is a(n)  a special type of/user defined iterator

\end{solution}|
\end{lstlisting}
\newpage


\question
Write a generator function \textbf{gen\_inf} that returns a generator which yields all the numbers in the provided list one by one in an infinite loop. 

\begin{lstlisting}
>>> t = gen_inf([3, 4, 5])
>>> next(t)
3
>>> next(t)
4
>>> next(t)
5
>>> next(t)
3
>>> next(t)
4
def gen_inf(lst):
\end{lstlisting}

\begin{solution}[1in]
\begin{verbatim}
#solution 1
def gen_inf(lst):
    while True:
        for elem in lst:
            yield elem
#solution 2
def gen_inf(lst):
    while True:
        yield from iter(lst)
\end{verbatim}
\end{solution}
\newpage


\question
Implement a generator function called \texttt{filter(iterable, fn)} that only yields
elements of iterable for which fn returns True.

\begin{lstlisting}[language=Python]

def naturals(): 
	i = 1 
	while True: 
		yield i 
		i += 1


def filter(iterable, fn):
    """
    >>> is_even = lambda x: x % 2 == 0
    >>> list(filter(range(5), is_even))
    [0 , 2 , 4]
    >>> all_odd = (2*y-1 for y in range (5))
    >>> list(filter(all_odd, is_even))
    []
    >>> s = filter(naturals(), is_even)
    >>> next(s)
    2
    >>> next(s)
    4
    """
    
    
    
    
    

\end{lstlisting}
\begin{solution}[.75in]
\begin{lstlisting}[language=Python]
    for elem in iterable:
        if fn(elem):
            yield elem
\end{lstlisting}
\end{solution}

\question What could you use a generator for that you could not use a standard iterator paired with a function for? 
\begin{solution}[.75in]
Call next on an infinite iterator
\end{solution}
\newpage


\question
Define \textbf{tree\_sequence}, a generator that iterates through a tree by first yielding the root value and then yielding the values from each branch.


\begin{lstlisting}
def tree_sequence(t):
	"""
	>>> t = tree(1, [tree(2, [tree(5)]), tree(3, [tree(4)])])
	>>> print(list(tree_sequence(t)))
	[1, 2, 5, 3, 4]
	"""

\end{lstlisting}
\begin{solution}
\begin{verbatim}
yield label(t)
  for branch in branches(t):
     for value in tree_sequence(branch):
        yield value

Alternate solution:
yield label(t)
   for branch in branches(t):
      yield from tree_sequence(branch)

\end{verbatim}
\end{solution}


\newpage

\question
Write a function \textbf{make\_digit\_getter} that, given a positive integer n, returns a new function that returns the digits in the integer one by one, starting from the rightmost digit. 

Once all digits have been removed, subsequent calls to the function should return the sum of all the digits in the original integer.

\begin{lstlisting}
def make_digit_getter(n):
	""" Returns a function that returns the next digit in n
	each time it is called, and the total value of all the integers
	once all the digits have been returned.
	>>> year = 8102
	>>> get_year_digit = make_digit_getter(year)
	>>> for _ in range(4):
	... print(get_year_digit())
	2
	0
	1
	8
	>>> get_year_digit()
	11
	"""

\end{lstlisting}
\begin{solution}
\begin{lstlisting}
def make_digit_getter(n):
	total = 0
	def get_next():
		nonlocal n, total
		if n == 0:
			return total
		val = n % 10
		n = n // 10
		total += val
		return val
	return get_next
\end{lstlisting}
\end{solution}


\newpage

\question
Sorry another environment diagram, but it's the last one I promise.

\begin{lstlisting}
def iter(iterable):
    def iterator(msg):
        nonlocal iterable
        if msg == 'next':
            next = iterable[0]
            iterable = iterable[1:]
            return next
        elif msg == 'stop':
            raise StopIteration
    return iterator 
def next(iterator):
    return iterator('next')
def stop(iterator):
    iterator('stop')
    
lst = [1, 2, 3]
iterator = iter(lst)
elem = next(iterator)

\end{lstlisting}
\begin{solution}
\begin{lstlisting}
https://tinyurl.com/y3xxycgp
\end{lstlisting}
\end{solution}
