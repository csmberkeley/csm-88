\define{Generators} can be used to create iterators as well. Generators use a
\texttt{yield} statement instead of \texttt{return}. When a generator function
is called, the body of the function is not evaluated. Instead, an iterator is
created and is the return value of the function call. The elements of this
iterator are the yielded values of the function. For extra fun, \texttt{yield from}
lets generators yield multiple values at once.

\vspace{1em}
\begin{lstlisting}[language=Python]
>>> square = lambda x: x*x
>>> def many_squares(s):
...     for x in s:
...         yield square(x)
...     yield from [square(x) for x in s]
...     yield from map(square, s)
...
>>> list(many_squares([1, 2, 3]))
[1, 4, 9, 1, 4, 9, 1, 4, 9]
\end{lstlisting}
