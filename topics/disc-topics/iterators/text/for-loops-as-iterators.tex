One important application of iterables 
and iterators is the \lstinline$for$ loop.
We've seen how we can use \lstinline$for$ loops to iterate over iterables like lists and 
dictionaries.

This only works because the \lstinline$for$ loop implicitly creates an iterator
using the built-in \lstinline$iter$ function.
Python then calls \lstinline$next$ repeatedly on the iterator, until it raises
\lstinline$StopIteration$.

The code to the right shows how we can mimic the behavior of \lstinline$for$ loops
using \lstinline$while$ loops.

Note that most iterators are also iterables - that is, calling \lstinline$iter$ on 
them will return an iterator. This means that we can use them inside \lstinline$for$ loops. 
However, calling \lstinline$iter$ on most iterators will not create a new iterator -
instead, it will simply return the same iterator.

We can also iterate over iterables in a list comprehension or pass in an iterable to
the built-in function \lstinline$list$ in order to put the items of an iterable into
a list.

