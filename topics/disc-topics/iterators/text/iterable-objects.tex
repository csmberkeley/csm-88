An \define{iterable} object is any container that can be processed sequentially.
Examples of iterables are lists, tuples, strings, and dictionaries. To process
the elements sequentially, call \texttt{iter} on the iterable to retrieve an
iterator.

An \define{iterator} is an object that tracks the position in a sequence of
values in order to provide sequential access. It returns elements one at a time
and is only good for one pass through the sequence. To access the next element
of an iterator, call \texttt{next} on the object. Each time \texttt{next} is
called, the iterator advances.

We can create as many iterators as we would like from a single iterable. However,
iterators will go through the elements of the sequence they represent only once.
To go through an iterable twice, create two iterators!

\vspace{1em}
\begin{lstlisting}[language=Python]
>>> iterable = [4, 8, 15, 16, 23, 42]
>>> iterator1 = iter(iterable)
>>> next(iterator1)
4
>>> next(iterator1)
8
>>> next(iterator1)
15
>>> iterator2 = iter(iterable)
>>> next(iterator2)
4
\end{lstlisting}
