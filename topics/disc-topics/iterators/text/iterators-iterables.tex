An \define{iterable} is a data type which contains
a collection of values which can be processed one by one 
sequentially. Some examples of iterables we've seen include
lists, tuples, strings, and dictionaries. In general, any
object that can be iterated over in a \lstinline$for$ loop
can be considered an iterable.


While an iterable contains values that can be iterated over,
we need another type of object called an \define{iterator}
to actually retrieve values contained in an iterable. Calling
the \lstinline$iter$ function on an iterable will create an iterator
over that iterable. Each iterator keeps track of its position within
the iterable. Calling the \lstinline$next$ function on an iterator will 
give the current value in the iterable and move the iterator's position 
to the next value. 


In this way, the relationship between an
iterable and an iterator is analogous to the relationship between
a book and a bookmark - an iterable contains the data that is
being iterated over, and an iterator keeps track of your position 
within that data.


Once an iterator has returned all the values in an iterable, subsequent
calls to \lstinline$next$ on that iterable will result in a 
\lstinline$StopIteration$ exception. In order to be able to access the values
in the iterable a second time, you would have to create a second iterator.

