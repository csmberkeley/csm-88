Iterables can be used in for loops and as arguments to functions that require a
sequence (e.g.\ \lstinline$map$ and \lstinline$zip$). For example:
\begin{lstlisting}
>>> for n in Range(2):
...     print(n)
...
0
1
\end{lstlisting}

This works because the for loop implicitly creates an iterator using the
\lstinline$__iter__$ method. Python then repeatedly calls \lstinline$next$
repeatedly on the iterator, until it raises \lstinline$StopIteration$. In other
words, the loop above is (basically) equivalent to:

\begin{blocksection}
\begin{lstlisting}
range_iterator = iter(Range(2))
is_done = False
while not is_done:
    try:
        val = next(range_iterator)
        print(val)
    except StopIteration:
        is_done = True
\end{lstlisting}
\end{blocksection}
