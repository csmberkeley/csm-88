An \define{iterable} is any container that can be processed sequentially.
Examples include lists, tuples, strings, and dictionaries.
Often we want to access the elements of an iterable, one at a time.
We find ourselves writing \lstinline$lst[0]$, \lstinline$lst[1]$,
\lstinline$lst[2]$, and so on.
It would be more convenient if there was an object that could do this for us,
so that we don't have to keep track of the indices.

This is where \define{iterators} come in. Given an iterable, we can call the
\lstinline$iter$ function on that iterable to return a new iterator object.
Each time we call \lstinline$next$ on the iterator object, it gives us one
element at a time, just like we wanted.
When it runs out of elements to give, calling \lstinline$next$ on the iterator
object will raise a \lstinline$StopIteration$ exception.

We can create as many iterators as we would like from a single iterable.
But, each iterator goes through the elements of the iterable only once.
If you want to go through an iterable twice, create two iterators!
