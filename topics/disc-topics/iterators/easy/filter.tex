\question
Implement an iterator class called Filter. The \texttt{\_\_init\_\_} method for
Filter takes an iterable and a one-argument function that either returns True
or False. The Filter iterator represents a sequence that only contains elements
of the iterable for which the predicate function returns True. Do not use a
generator in your solution.

\begin{lstlisting}[language=Python]
class Filter :
    """
    >>> is_even = lambda x: x % 2 == 0
    >>> for elem in Filter(range(5) , is_even):
    ...     print(elem)
    0
    2
    4
    >>> all_odd = (2*y-1 for y in range (5))
    >>> for elem in Filter(all_odd, is_even):
    ...     print(elem) # No elements are even !
    >>> s = Filter(naturals(), is_even)
    >>> next(s)
    2
    >>> next(s)
    4
    """
    def __init__(self, iterable, fn):
\end{lstlisting}
\begin{solution}[1.25in]
\begin{lstlisting}[language=Python]
        self.iterator = iter(iterable)
        self.fn = fn
\end{lstlisting}
\end{solution}
\begin{lstlisting}
    def __iter__( self ):
\end{lstlisting}
\begin{solution}[.75in]
\begin{lstlisting}[language=Python]
    return self
\end{lstlisting}
\end{solution}
\begin{lstlisting}
    def __next__( self ):
\end{lstlisting}
\begin{solution}[.75in]
\begin{lstlisting}[language=Python]
    candidate = next(self.iterator)
    while not self.fn(candidate):
        candidate = next(self.iterator)
    return candidate
\end{lstlisting}
\end{solution}
