\question Define an iterator whose $i$th element is the result of combining the
$i$th elements of two input iterators using some binary operator, also given as
input. The resulting iterator should have a size equal to the size of the
shorter of its two input iterators.

\begin{minipage}{\textwidth}
\begin{lstlisting}
>>> from operator import add
>>> evens = IteratorCombiner(Naturals(), Naturals(), add)
>>> next(evens)
0
>>> next(evens)
2
>>> next(evens)
4
\end{lstlisting}
\end{minipage}
\begin{lstlisting}[language=Python]
class IteratorCombiner(object):
    def __init__(self, iterator1, iterator2, combiner):
\end{lstlisting}

\begin{solution}[1in]
\begin{lstlisting}[language=Python]
        self.iterator1 = iterator1
        self.iterator2 = iterator2
        self.combiner = combiner
\end{lstlisting}
\end{solution}
\begin{lstlisting}[language=Python]
    def __next__(self):
\end{lstlisting}
\begin{solution}[.75in]
\begin{lstlisting}[language=Python]
        return self.combiner(next(self.iterator1), next(self.iterator2))
\end{lstlisting}
\end{solution}
\begin{lstlisting}[language=Python]
    def __iter__(self):
\end{lstlisting}
\begin{solution}[.5in]
\begin{lstlisting}[language=Python]
        return self
\end{lstlisting}
\end{solution}

\question What is the result of executing this sequence of commands?
\begin{lstlisting}[language=Python]
>>> nats = Naturals()
>>> doubled_nats = IteratorCombiner(nats, nats, add)
>>> next(doubled_nats)
\end{lstlisting}

\begin{solution}[.25in]
1
\end{solution}
\begin{lstlisting}[language=Python]
>>> next(doubled_nats)
\end{lstlisting}
\begin{solution}[.25in]
5
\end{solution}
