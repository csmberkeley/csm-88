\begin{blocksection}

\question Write facts for {\tt prefix}, a relation between two lists that is
satisfied if and only if elements of the first list are the first elements of
the second list, in order.

\lstset{language={}}
\begin{lstlisting}
> (query (prefix (being for the) (being for the
                  benefit of mister kite)))
Success!
> (query (prefix (for no one) (for no one)))
Success!
> (query (prefix () (got to get you into my life)))
Success!
> (query (prefix (want i to) (i want to hold your hand)))
Failed.
\end{lstlisting}

\begin{solution}[1.7in]
\begin{lstlisting}
(fact (prefix () ?any-list))
(fact (prefix (?first . ?small) (?first . ?big))
      (prefix ?small ?big))
\end{lstlisting}

Note: You should try out the query 
\texttt{(query (prefix ?sub (1 2 3 4 5)))}.
\end{solution}

\end{blocksection}

\begin{blocksection}

\question Write facts for {\tt sublist}, a relation between two lists that is
satisfied if and only if the first is a consecutive sublist of the second. For
example:

\begin{lstlisting}
> (query (sublist (give) (never gonna give you up)))
Success!
> (query (sublist (you up) (never gonna give you up)))
Success!
> (query (sublist () (never gonna give you up)))
Success!
> (query (sublist (never give up) (never gonna give you up)))
Failed.
> (query (sublist (let you down) (never gonna give you up)))
Failed.
\end{lstlisting}

Hint: You will want to use the prefix fact that you previously defined.

\begin{solution}[2in]
\begin{lstlisting}
(fact (sublist ?a ?b) (prefix ?a ?b))
(fact (sublist ?sub (?first . ?rest)) (sublist ?sub ?rest))
\end{lstlisting}
\end{solution}

\end{blocksection}
