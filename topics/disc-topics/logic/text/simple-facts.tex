In Logic, you can define \textit{facts}. Facts are simply Scheme lists of
relations and relations are simply Scheme lists of symbols. Remember, relations
are {\bf not} call expressions; instead, relations are used to express
relationships between symbols.

Here's an example of a fact:

\begin{lstlisting}
> (fact (sells supermarket groceries))
\end{lstlisting}

This line of code says: ``This is a fact: supermarkets sell groceries''. When we
declare something as a fact, we are simply saying that it is a \textbf{true
statement}.  You can think of a fact as an axiom, i.e., something that is
fundamentally true.

``sells'' is a quality that relates two things, ``supermarket'' and
``groceries.''  What are the values of ``supermarket'' and ``groceries''? They
have no values! They are \textit{symbols} -- symbols are Logic's primitives.

Having defined some facts, we can make \textit{queries} -- in other words, we
can ask Logic for information based on the facts that we've defined:

\begin{lstlisting}
> (query (sells supermarket groceries))
Success!
> (query (sells supermarket books))
Failed.
> (query (sells supermarket ?stuff))
Success!
stuff: groceries
\end{lstlisting}

The first query asks, ``Is it a fact that supermarkets sell groceries?'' and the
second query asks, ``Is it a fact that supermarkets sell books?''.  The third
query above is equivalent to asking ``What do supermarkets sell?'' Logic replies
that supermarkets sell groceries, \textit{based on the previously defined fact}.

Note that {\tt ?stuff} is a variable in Logic, whereas {\tt supermarket} is a
symbol.  {\tt supermarket} is always going to be {\tt supermarket}, but {\tt
?stuff} is unknown -- it is only \textit{after} the query that we know what the
value of {\tt ?stuff} is.

% A similar query is

% \begin{lstlisting}
% > (query (sells ?place groceries))
% Success!
% place: supermarket
% \end{lstlisting}

% This query is equivalent to asking ``Which places sell groceries?'' Once again,
% Logic replies based on the previously defined fact.

We can also query both multiple elements of a relation:

\begin{lstlisting}
> (query (sells ?place ?stuff))
Success!
place: supermarket stuff: groceries
\end{lstlisting}

This is equivalent to asking ``What are places that sell stuff, and what stuff
do they sell?''  Logic will tell you what each variable should be based on
previously defined facts.

% In Logic, we can also model hierarchical data by nesting relations inside of
% other relations.  For example:

% \begin{lstlisting}
% (fact (person (name bob) (age 49)))
% (fact (person (name alice) (age 20)))
% \end{lstlisting}

% declares two facts. The first fact asserts that there exists a person whose name
% is Bob and whose age is 49. The second fact asserts that there exists a person
% whose name is Alice and whose age is 20.

% Moreover, we can query the facts that we previously defined:

% \begin{lstlisting}
% > (query (person (name ?first-name) (age 49)))
% Success!
% first-name: bob
% > (query (person (name bob) ?age))
% Success!
% age: (age 49)
% \end{lstlisting}

% The first query asks, "What is the name of a person whose age is 49?" and the
% second query asks, "What is the age of a person named Bob?".
