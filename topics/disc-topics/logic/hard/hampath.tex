\tikzset{main node/.style={circle,fill=blue!20,draw,minimum size=1cm,inner sep=0pt},}

Similarly to the map coloring example from lecture, we can construct the graph
shown on the right in Logic using some simple facts:\\

\begin{lstlisting}
> (fact (vertices A B C D))
> (fact (edge B A))
> (fact (edge B C))
> (fact (edge C A))
> (fact (edge D B))
\end{lstlisting}

\vspace{-3cm}
\hspace{9cm}
\begin{tikzpicture}[>=stealth,auto,semithick]

\node[main node] (a) {$A$};
\node[main node] (b) [right=1.7cm of a] {$B$};
\node[main node] (c) [below=1.7cm of a] {$C$};
\node[main node] (d) [right=1.7cm of c] {$D$};

\path[->] (b) edge node {} (a);
\path[->] (b) edge node {} (c);
\path[->] (c) edge node {} (a);
\path[->] (d) edge node {} (b);

\end{tikzpicture}

We can implement some pretty interesting relations for this graph. In these
questions we will implement the Hamiltonian path relation, which is true for a
vertex if there a path starting from that vertex that visits each node once. For
example, in the given graph, {\tt (D B C A)} is a Hamiltonian path, and this is
actually the only one. Let's start with a few helper relations.

\vspace{1cm}

\begin{questions}
\question
Implement facts for the {\tt remove} relation, which relates {\tt elem} and two
lists if, and only if, the second list has all and only the elements of the
first list but with one occurrence of {\tt elem} removed. {\tt elem} must
appear at least once in the first list.

\begin{lstlisting}
> (query (remove a (b a n a n a s) ?lst))
Success!
lst: (b n a n a s)
lst: (b a n n a s)
lst: (b a n a n s)
> (query (remove no (not in this list) ?lst))
Failed.
\end{lstlisting}

\begin{solution}[1.7in]
\begin{lstlisting}
(fact (remove ?elem (?elem . ?rest) ?rest))
(fact (remove ?elem (?first . ?rest1) (?first . ?rest2))
      (remove ?elem ?rest1 ?rest2))
\end{lstlisting}
\end{solution}

\newpage
\question
Use {\tt remove} to implement fact for the {\tt contain-same} relation, which
relates two lists that contain exactly the same elements, though not necessarily
in the same order.

\begin{lstlisting}
> (query (contain-same (A B C D) (D C A B)))
Success!
> (query (contain-same (A B C) (B B C A)))
Failed.
\end{lstlisting}

\begin{solution}[1.5in]
\begin{lstlisting}
(fact (contain-same () ()))
(fact (contain-same (?first . ?rest) ?other)
      (remove ?first ?other ?other-rest)
      (contain-same ?rest ?other-rest))
\end{lstlisting}
\end{solution}

\question
Our last helper relation is aptly named {\tt hamil-helper}, a relation between a
vertex {\tt v} and a list of vertices {\tt so-far} that is satisfied if the rest
of a Hamiltonian path can be constructed from {\tt v} given that we have already
visited the vertices in {\tt so-far}. {\tt contain-same} may be useful here.

\begin{lstlisting}
> (query (hamil-helper A (B C D)))
Success!
> (query (hamil-helper A (B C)))
Failed.
\end{lstlisting}

\begin{solution}[2in]
\begin{lstlisting}
(fact (hamil-helper ?v ?so-far)
      (vertices . ?vertices)
      (contain-same (?v . ?so-far) ?vertices))
(fact (hamil-helper ?v ?so-far)
      (edge ?v ?v2)
      (hamil-helper ?v2 (?v . ?so-far)))
\end{lstlisting}
\end{solution}

\question
Finally, write {\tt hamiltonian}, a relation that holds for a vertex {\tt v} if
there is a Hamiltonian path starting from {\tt v}. This should be very easy now!

\begin{lstlisting}
> (query (hamiltonian D))
Success!
> (query (hamiltonian B))
Failed.
\end{lstlisting}

\begin{solution}[0.5in]
\begin{lstlisting}
(fact (hamiltonian ?v)
      (hamil-helper ?v ()))
\end{lstlisting}
\end{solution}

\end{questions}
