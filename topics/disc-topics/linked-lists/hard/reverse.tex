\question
Define \texttt{reverse}, which takes in a linked list and reverses the order of the links. The function may \emph{not} return a new list; it must mutate the original list. Return a pointer to the head of the reversed list.
\begin{lstlisting}
def reverse(lnk):
    """
    >>> a = Link(1, Link(2, Link(3)))
    >>> r = reverse(a)
    >>> r.first
    3
    >>> r.rest.first
    2
    """
\end{lstlisting}

\begin{solution}[1.2in]
{\Large Recursive solution:}
\begin{lstlisting}
    if lnk is Link.empty or lnk.rest is Link.empty:
        return lnk
    rest_rev = reverse(lnk.rest)
    lnk.rest.rest = lnk
    lnk.rest = Link.empty
    return rest_rev
\end{lstlisting}

For the base case, a linked list with no items or a single item is trivial to
reverse.

Let's formally name our variables to make the explanation of the following
process a bit easier. The original list is \texttt{lnk}, we reverse
\texttt{lnk.rest} recursively and get back a pointer to the head of the reversed
version of \texttt{lnk.rest}, which is \texttt{rest\char`_rev}.

Notice that \texttt{lnk.rest} is the last item of the list referred to by
\texttt{rest\char`_rev}. All we have to do is attach the first item of
\texttt{lnk} to the end of the reversed rest, and then make sure that
\texttt{lnk.rest} is the empty list as it is now the last item in the reversed
list.\\
\href{https://www.youtube.com/watch?v=eEIMcIN9Teg&index=3&list=PLx38hZJ5RLZdJgRCgpaTbmRXKAHOUmomO}{Video walkthrough}

{\Large Iterative solution (1):}
\begin{lstlisting}
    if lnk is Link.empty:
        return lnk
    cur = lnk
    nxt = lnk.rest
    cur.rest = Link.empty

    while nxt is not Link.empty:
        after = nxt.rest
        nxt.rest = cur
        cur = nxt
        nxt = after
    return cur
\end{lstlisting}

The iterative solution is quite different from the recursive solution. We go
through every item in our linked list, and reattach them in reverse order. That
is, we attach the second item back to the first item, the third item back to the
second item, and so on.

The tricky part is figuring what information we need to keep track of in order
to do this. We use a two pointer method that tracks a current and a following
position in a linked list. The logic is not too complicated, but the best way to
understand it is to work through an example with a box and pointer diagram.

{\Large Iterative solution (2):}
\begin{lstlisting}
    new_lnk = Link.empty
    while lnk is not Link.empty:
        new_lnk, lnk.rest, lnk = lnk, new_lnk, lnk.rest
    return new_lnk
\end{lstlisting}

Here's yet another iterative approach, different from both the previous
iterative and recursive approaches.

We begin by asking how we'd gather the values in a linked list into a list (not
a linked list, but a regular Python list) in reverse order.

\begin{lstlisting}
>>> xs = Link(1, Link(2, Link(3)))
>>> xs
Link(1, Link(2, Link(3)))
>>> reverse(xs)
[3, 2, 1]
\end{lstlisting}

We could do this by iterating through the linked list and inserting the values
into the front of the list:

\begin{lstlisting}
def reverse(lnk):
    new_list = []
    while lnk is not Link.empty:
        new_list.insert(0, lnk.first)  # Insert the link value before all existing elements of new_list
        lnk = lnk.rest
    return new_list
\end{lstlisting}

This works because if value A comes before value B in the original list, then B
will be inserted before value A in our new list.

We could keep this same approach, but have \texttt{new\char`_lnk} be a linked
list instead of an ordinary list.

\begin{lstlisting}
def reverse(lnk):
    new_lnk = Link.empty
    while lnk is not Link.empty:
        new_lnk = Link(lnk.first, new_lnk)  # Create a new link and insert it before all existing links in new_lnk
        lnk = lnk.rest
    return new_lnk
\end{lstlisting}

This works, but we are not allowed to create new \texttt{Link} instances.

So, instead of copying \texttt{lnk} by constructing a new \texttt{Link} with the
same first attribute as \texttt{lnk}, we should just make \texttt{lnk} the new
head of \texttt{new\char`_lnk}. But that means we simultaneously need to make
three updates:
\begin{itemize}
    \item \texttt{new\char`_lnk} should point at what \texttt{lnk} was pointing
        at (because \texttt{lnk} is the new head of \texttt{new\char`_lnk}).
    \item \texttt{lnk.rest} should point at what \texttt{new\char`_lnk} was
        pointing at (this makes \texttt{lnk} the new head of
        \texttt{new\char`_lnk}).
    \item \texttt{lnk} should point at what \texttt{lnk.rest} was pointing at
        (because we still need to iterate through the original list).
\end{itemize}

This leads us the final solution presented earlier.
\end{solution}
