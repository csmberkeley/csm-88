\question
Write a function that takes a sorted linked list of integers and mutates
it so that all duplicates are removed.

\begin{lstlisting}
def remove_duplicates(lnk):
    """
    >>> lnk = Link(1, Link(1, Link(1, Link(1, Link(5)))))
    >>> remove_duplicates(lnk)
    >>> lnk
    Link(1, Link(5))
    """
\end{lstlisting}
\begin{solution}[1.75in]
Recursive solution:
\begin{lstlisting}
    if lnk is Link.empty or lnk.rest is Link.empty:
        return
    if lnk.first == lnk.rest.first:
        lnk.rest = lnk.rest.rest
        remove_duplicates(lnk)
    else:
        remove_duplicates(lnk.rest)
\end{lstlisting}
For a list of one or no items, there are no duplicates to remove.

Now consider two possible cases:
\begin{itemize}
    \item If there is a duplicate of the first item, we will find that the first
        and second items in the list will have the same values (that is,
        \texttt{lnk.first == lnk.rest.first}). We can confidently state this
        because we were told that the input linked list is in sorted order, so
        duplicates are adjacent to each other. We'll remove the second item from
        the list.

        Finally, it's tempting to recurse on the remainder of the list
        (\texttt{lnk.rest}), but remember that there could still be more
        duplicates of the first item in the rest of the list! So we have to
        recurse on \texttt{lnk} instead. Remember that we have removed an item
        from the list, so the list is one element smaller than before. Normally,
        recursing on the same list wouldn't be a valid subproblem.
    \item Otherwise, there is no duplicate of the first item. We can safely
        recurse on the remainder of the list.
\end{itemize}

Iterative solution:
\begin{lstlisting}
    while lnk is not Link.empty and lnk.rest is not Link.empty:
        if lnk.first == lnk.rest.first:
            lnk.rest = lnk.rest.rest
        else:
            lnk = lnk.rest
\end{lstlisting}

The loop condition guarantees that we have at least one item left to consider
with \texttt{lnk}.

For each item in the linked list, we pause and remove all adjacent items that
have the same value. Once we see that \texttt{lnk.first != lnk.rest.first}, we
can safely advance to the next item. Once again, this takes advantage of the
property that our input linked list is sorted.

\end{solution}
