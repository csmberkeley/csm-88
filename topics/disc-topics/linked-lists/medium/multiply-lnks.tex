\question
Write a function that takes in a Python list of linked lists and
multiplies them element-wise. It should return a new linked list.

If not all of the \texttt{Link} objects are of equal length, return a
linked list whose length is that of the shortest linked list given. You
may assume the \texttt{Link} objects are shallow linked lists, and that
\texttt{lst\_of\_lnks} contains at least one linked list.

\begin{lstlisting}
def multiply_lnks(lst_of_lnks):
    """
    >>> a = Link(2, Link(3, Link(5)))
    >>> b = Link(6, Link(4, Link(2)))
    >>> c = Link(4, Link(1, Link(0, Link(2))))
    >>> p = multiply_lnks([a, b, c])
    >>> p.first
    48
    >>> p.rest.first
    12
    >>> p.rest.rest.rest is Link.empty
    True
    """
\end{lstlisting}
\begin{solution}[1.0in]
Recursive solution:
\begin{lstlisting}
    product = 1
    for lnk in lst_of_lnks:
        if lnk is Link.empty:
            return Link.empty
        product *= lnk.first
    lst_of_lnks_rests = [lnk.rest for lnk in lst_of_lnks]
    return Link(product, multiply_lnks(lst_of_lnks_rests))
\end{lstlisting}
For our base case, if we detect that any of the lists in the list of
\texttt{Link}s is empty, we can return the empty linked list as we're not going
to multiply anything.

Otherwise, we compute the product of all the firsts in our list of
\texttt{Link}s. Then, the subproblem we use here is the rest of all the linked
lists in our list of Links. Remember that the result of calling
\texttt{multiply\char`_lnks} will be a linked list! We'll use the product we've
built so far as the first item in the returned \texttt{Link}, and then the
result of the recursive call as the rest of that \texttt{Link}.

Iterative solution:
\begin{lstlisting}
    import operator
    from functools import reduce
    def prod(factors):
         return reduce(operator.mul, factors, 1)

    head = Link.empty
    tail = head
    while Link.empty not in lst_of_lnks:
        all_prod = prod([l.first for l in lst_of_lnks])
        if head is Link.empty:
            head = Link(all_prod)
            tail = head
        else:
            tail.rest = Link(all_prod)
            tail = tail.rest
        lst_of_lnks = [l.rest for l in lst_of_lnks]
    return head
\end{lstlisting}

The iterative solution is a bit more involved than the recursive solution.
Instead of building the list ``backwards'' as in the recursive solution (because
of the order that the recursive calls result in, the last item in our list will
be finished first), we'll build the resulting linked list as we go along.

We use \texttt{head} and \texttt{tail} to track the front and end of the new
linked list we're creating. Our stopping condition for the loop is if any of the
\texttt{Link}s in our list of \texttt{Link}s runs out of items.

Finally, there's some special handling for the first item. We need to update
both head and tail in that case. Otherwise, we just append to the end of our
list using tail, and update tail.
\end{solution}
