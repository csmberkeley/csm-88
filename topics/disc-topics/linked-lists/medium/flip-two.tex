\question
Write a recursive function \texttt{flip\_two} that takes as input a
linked list \texttt{lnk} and mutates \texttt{lnk} so that every pair
is flipped.

\begin{lstlisting}
def flip_two(lnk):
    """
    >>> one_lnk = Link(1)
    >>> flip_two(one_lnk)
    >>> one_lnk
    Link(1)
    >>> lnk = Link(1, Link(2, Link(3, Link(4, Link(5)))))
    >>> flip_two(lnk)
    >>> lnk
    Link(2, Link(1, Link(4, Link(3, Link(5)))))
    """
\end{lstlisting}
\begin{solution}[1.25in]
Recursive solution:
\begin{lstlisting}
    if lnk is Link.empty or lnk.rest is Link.empty:
        return
    lnk.first, lnk.rest.first = lnk.rest.first, lnk.first
    flip_two(lnk.rest.rest)
\end{lstlisting}
If there's only a single item (or no item) to flip, then we're done.

Otherwise, we swap the contents of the first and second items in the list. Since
we've handled the first two items, we then need to recurse on

Although the question explicitly asks for a recursive solution, there is also a
fairly similar iterative solution:
\begin{lstlisting}
    while lnk is not Link.empty and lnk.rest is not Link.empty:
        lnk.first, lnk.rest.first = lnk.rest.first, lnk.first
        lnk = lnk.rest.rest
\end{lstlisting}
We will advance \texttt{lnk} until we see there are no more items or there is
only one more Link object to process. Processing each \texttt{Link} involves
swapping the contents of the first and second items in the list (same as the
recursive solution).

Notice that the code is remarkably similar to the recursive implementation of
\texttt{flip\char`_two}.\\
\href{https://www.youtube.com/watch?v=wPF9gOx15V0&index=2&list=PLx38hZJ5RLZdJgRCgpaTbmRXKAHOUmomO&t=9m54s}{Video walkthrough}
\end{solution}
