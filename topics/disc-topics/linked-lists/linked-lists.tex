\question
What is a linked list? Why do we consider it a naturally recursive structure?

\begin{solution}[0.25in]
A linked list is a data structure with a first and a rest, where the first is some arbitrary element and the rest MUST be another linked list (or Link.empty)
\end{solution}

\question
Draw a box and pointer diagram for the following:
\begin{lstlisting}[language=Python]
Link('c', Link(Link(6, Link(1, Link('a'))), Link('s')))
\end{lstlisting}
\begin{solution}[0.25in]
\end{solution}

\question
The Link class can represent lists with cycles. That is, a list may contain itself as a sublist. Implement \textbf{has\_cycle} that returns whether its argument, a Link instance, contains a cycle. There are two ways to do this: iteratively with two pointers, or keeping track of Link objects we've seen already. Try to come up with both!

\begin{lstlisting}[language=Python]

def has_cycle(link):
    """
    >>> s = Link(1, Link(2, Link(3)))
    >>> s.rest.rest.rest = s
    >>> has_cycle(s)
    True
    """
    |\begin{solution}[1in]
    \begin{verbatim}
    
    # solution 1
    tortoise = link
    hare = link.rest
    while tortoise.rest and hare.rest and hare.rest.rest:
        if tortoise is hare:
            return True
        tortoise = tortoise.rest
        hare = hare.rest.rest
    return False
    
    # solution 2
    seen = []
    while link.rest:
        if link in seen:
            return True
        seen.append(link)
        link = link.rest
    return False
    \end{verbatim}
    \end{solution}|
    
\end{lstlisting}

\question
Fill in the following function, which checks to see if \textbf{sub\_link}, a particular sequence of items in one linked list,  can be found in another linked list (the items have to be in order, but not necessarily consecutive).

\begin{lstlisting}
def seq_in_link(link, sub_link):
    """
    >>> lnk1 = Link(1, Link(2, Link(3, Link(4))))
    >>> lnk2 = Link(1, Link(3))
    >>> lnk3 = Link(4, Link(3, Link(2, Link(1))))
    >>> seq_in_link(lnk1, lnk2)
    True
    >>> seq_in_link(lnk1, lnk3)
    False
    """
    |\begin{solution}
    \begin{verbatim}
    if sub_link is Link.empty:
        return True
    if link is Link.empty:
        return False
    if link.first == sub_link.first:
        return seq_in_link(link.rest, sub_link.rest)
    else:
        return seq_in_link(link.rest, sub_link)
    \end{verbatim}
    \end{solution}|
\end{lstlisting}