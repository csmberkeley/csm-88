\question
What do lambda expressions do? Can we write all functions as lambda expressions? In what cases are lambda expressions useful? 
\begin{solution}[0.75in]
Lambda expressions create functions. When a lambda expression is evaluated, it produces a function. We often use lambdas to create short anonymous functions that we won?t need for too long. \newline We can?t write all functions as lambda expressions because lambda functions all have to have ?return? statements and they can?t contain very complex multi-line expressions.
\end{solution}

\question
Determine if each of the following will error: \newline
\begin{lstlisting}
>>> 1/0
|\begin{solution}
Error
\end{solution}|
>>> boom = lambda: 1/0
|\begin{solution}
No error, since we don't evaluate the body of the lambda when we define it.
\end{solution}|
>>> boom()
|\begin{solution}
Error
\end{solution}|
\end{lstlisting}

\question
Express the following lambda expression using a \textbf{def} statement, and the \textbf{def} statement using a lambda expression.
\begin{lstlisting}
pow = lambda x, y: x**y
|\begin{solution}[0.25in]
\begin{verbatim}
def pow(x, y):
    return x**y
\end{verbatim}
\end{solution}|
\end{lstlisting}

\begin{lstlisting}
def foo(x):
    def f(y):
        def g(z):
            return x + y * z
        return g
    return f
\end{lstlisting}
\begin{solution}[0.25in]
foo = lambda x: lambda y: lambda z: x + y * z
\end{solution}
\newpage
\question 
Draw Environment Diagrams for the following lines of code
\begin{lstlisting}
square = lambda x: x * x
higher = lambda f: lambda y: f(f(y))
higher(square)(5)
\end{lstlisting}
\begin{solution}
Solution: https://goo.gl/LATqV9
\end{solution}

\begin{lstlisting}
a = (lambda f, a: f(a))(lambda b: b * b, 2)
\end{lstlisting}
\begin{solution}
Solution: https://goo.gl/TyriuP
\end{solution}
\newpage
\question
Write \textbf{make\_skipper}, which takes in a number n and outputs a function. When this function takes in a number x, it prints out all the numbers between 0 and x, skipping every nth number (meaning skip any value that is a multiple of n).
\begin{lstlisting}[language=Python]
def make_skipper(n):
    """
    >>> a = make_skipper(2)
    >>> a(5)
    1
    3
    5
    """
|\begin{solution}[1.5in] \begin{verbatim}
    def skipper(x):
        for i in range(x + 1):
            if i % n != 0:
                print(i)
    return skipper
    \end{verbatim}
\end{solution}|
\end{lstlisting}