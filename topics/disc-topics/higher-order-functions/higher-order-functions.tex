\question
What do lambda expressions do? Can we write all functions as lambda expressions? (Hint: think about the limitations of lambdas) In what cases are lambda expressions useful? 
\begin{solution}[0.75in]
Lambda expressions create functions. When a lambda expression is evaluated, it produces a function. We often use lambdas to create short anonymous functions that we won't need for too long. \newline We can't write all functions as lambda expressions because lambda functions all have to have "return" statements. In addition, they can't contain very complex multi-line expressions.  \newline Lambda expressions are very useful. In HOF, we want to pass functions into another function or have a function return a function. Lambdas expressions serve as a short and convenient way to define functions in HOFs.
\end{solution}

\question
Determine if each of the following will error: \newline
\begin{lstlisting}
>>> 1/0
\end{lstlisting}
\begin{solution}[0.25in]
Error
\end{solution}
\begin{lstlisting}
>>> boom = lambda: 1/0
\end{lstlisting}
\begin{solution}[0.25in]
No error, since we don't evaluate the body of the lambda when we define it.
\end{solution}
\begin{lstlisting}
>>> boom()
\end{lstlisting}
\begin{solution}[0.25in]
Error
\end{solution}


\question
Express the following lambda expression using a \textbf{def} statement, and the \textbf{def} statement using a lambda expression.
\begin{lstlisting}
pow = lambda x, y: x**y
\end{lstlisting}
\begin{solution}[0.5in]
\begin{verbatim}
def pow(x, y):
    return x**y
\end{verbatim}
\end{solution}

\begin{lstlisting}
def foo(x):
    def f(y):
        def g(z):
            return x + y * z
        return g
    return f
\end{lstlisting}
\begin{solution}[0.25in]
foo = lambda x: lambda y: lambda z: x + y * z
\end{solution}
\newpage

\question 
For each of the following lines of code, determine what would be printed as the output. \newline
\begin{lstlisting}
>>> plus_one = lambda i: print(i+1)
>>> plus_one
\end{lstlisting}
\begin{solution}[0.25in]
function $<$lambda$>$ at 0x61CSTUFF
\end{solution}
\begin{lstlisting}
>>> plus_one(6)
\end{lstlisting}
\begin{solution}[0.25in]
7
\end{solution}
\begin{lstlisting}
>>> multiply = lambda x, y: x*y
>>> harder_lambda = lambda func: print(func(4, 5))
>>> harder_lambda(multiply)
\end{lstlisting}
\begin{solution}[0.25in]
20
\end{solution}


\question 
\textbf{Challenge Problem}: 
Draw Environment Diagrams for the following lines of code. 
\newline
Note: When working with lambdas in environment diagram problems, it is really helpful to write down  which line the lambda was define on.
\begin{lstlisting}
square = lambda x: x * x
higher = lambda f: lambda y: f(f(y))
b = higher(square)(5)
a = (lambda f, a: f(a))(lambda b: b * b, 2)
\end{lstlisting}
\begin{solution}
Solution: https://tinyurl.com/y69u6hu3
\end{solution}
