\question
What Would Scheme Display?

\begin{lstlisting}[language=Scheme]
> (define a (cons-stream 4 (cons-stream 6 (cons-stream 8 a))))
> (car a)
|\begin{solution}
4
\end{solution}|
> (cdr a)
|\begin{solution}
[promise (not forced)]
\end{solution}|
> (cdr-stream a)
|\begin{solution}
(6 . [promise (not forced)])
\end{solution}|
> (define b (cons-stream 10 a))
> (cdr b)
|\begin{solution}
[promise (not forced)]
\end{solution}|
> (cdr-stream b)
|\begin{solution}
(4 . [promise (forced)])
\end{solution}|
> (define c (cons-stream 3 (cons-stream 6)))
> (cdr-stream c)
|\begin{solution}
Error: too few operands in form
\end{solution}|
\end{lstlisting}

\question
Write a function \textbf{merge} that takes in two sorted infinite streams and returns a new infinite stream containing all the elements from both streams, in sorted order. 

\begin{lstlisting}[language=Scheme]
(define (merge s1 s2)







)
\end{lstlisting}

\begin{solution}
\begin{verbatim}
(define (merge s1 s2)
	(if (null? s1)
		s2
		(if (<= (car s1) (car s2))
			(cons-stream (car s1) (merge (cdr-stream s1) s2))
			(cons-stream (car s2) (merge s1 (cdr-stream s2)))
		)
	)
)
	
; Alternate solution
(define (merge s1 s2)
	(if (null? s1)
		s2
		(if (<= (car s1) (car s2))
			(cons-stream (car s1) (merge (cdr-stream s1) s2))
			(merge s2 s1)
		)
	)
)
\end{verbatim}
\end{solution}
