We can also create mutable functions in Scheme. In Python, when we want to
modify a binding in a parent frame, we declare it to be {\tt nonlocal} at the
start of the function and then assign to the name as normal.

In Scheme, we don't need to declare which bindings from a parent frame we wish
to modify. Instead, we use a new special form call {\tt set!} when we want to
modify an existing binding (regardless of whether that binding exists in the
current frame or a parent frame).

Just like the {\tt define} special form, {\tt set!} takes in two operands: the
symbol to be re-assigned and an expression that should be evaluated and assigned
to that symbol.

\centerline{\lstinline{(set! <symbol> <expression>)}}

Here's the same {\tt stepper} function from earlier, now written in Scheme.

\begin{lstlisting}[language=Scheme]
  (define (stepper num)
    (define (step)
      (set! num (+ num 1))
      num)
    step)
\end{lstlisting}

{\tt set!} will always modify the most local binding for that symbol that
exists. In other words, if the symbol is bound in the current frame, {\tt set!}
works the same as {\tt define}. Otherwise, it proceeds through parent frames
until it finds the symbol, and then re-binds it in that frame. If the binding
does not exist anywhere within the current environment, {\tt set!} will error.

Unlike {\tt nonlocal}, {\tt set!} can even modify bindings in the global frame.
For example:

\begin{lstlisting}[language=Scheme]
  scm> (define count 0)
  count
  scm> (define (increment) (set! count (+ count 1)))
  increment
  scm> (increment)
  scm> (increment)
  scm> (increment)
  scm> count
  3
\end{lstlisting}
