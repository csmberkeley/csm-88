\question
What will Scheme output? Draw the box and pointer whenever the expression evaluates to some pair or list.

\begin{lstlisting}
> (or 'false (/ 1 0) 'true)
|\begin{solution}
Error
\end{solution}|
> '(1 2 3)
|\begin{solution}
(1 2 3)
\end{solution}|
> (cons 2 '())
|\begin{solution}
(2)
\end{solution}|
> (cons 1 (cons 2 '()))
|\begin{solution}
(1 2)
\end{solution}|
> (cadar '((1 2) 3 (4 5)))
|\begin{solution}
2
\end{solution}|
> (caddr '((1 2) 3 (4 5)))
|\begin{solution}
(4 5)
\end{solution}|
> (cddar '((1 2) 3 (4 5)))
|\begin{solution}
()
\end{solution}|
> (cddr '((1 2) 3 (4 5)))
|\begin{solution}
((4 5))
\end{solution}|
\end{lstlisting}


\question
Define \textbf{append}, which takes in two lists and concatenates them together.

\begin{lstlisting}
> (append '(1 2 3) '(4 5 6))
(1 2 3 4 5 6)
\end{lstlisting}

\begin{solution}
\begin{verbatim}
(define (append lst1 lst2)
  (cond ((null? lst1) lst2)
(else (cons (car lst1) (append (cdr lst1) lst2)))))
\end{verbatim}
\end{solution}
\newpage

\question
Define \textbf{reverse}. You may use \textbf{append} in your definition.

\begin{lstlisting}
> (reverse '(1 2 3))
(3 2 1)
\end{lstlisting}

\begin{solution}[0.5in]
\begin{verbatim}
(define (reverse lst)
    (if (null? lst)
        lst
    (append (reverse (cdr lst)) (list (car lst)))))
\end{verbatim}
\end{solution}

\question
Define \textbf{reverse} without using \textbf{append}. (Hint: use a helper function and \textbf{cons})

\begin{solution}[0.5in]
\begin{verbatim}
(define (reverse lst)
    (define (helper lst reversed)
    (if (null? lst)
      reversed
      (helper (cdr lst) (cons (car lst) reversed ))))
    (helper lst '()))
\end{verbatim}
\end{solution}

\question
Define \textbf{add-to-all}, which takes in an item and a list of lists, and adds that item to the front of each nested list.

\begin{lstlisting}
> (add-to-all 'foo '((1 2) (3 4) (5 6)))
((foo 1 2) (foo 3 4) (foo 5 6))
\end{lstlisting}

\begin{solution}[0.5in]
\begin{verbatim}
(define (add-to-all item lst)
  (if (null? lst)
    lst
    (cons (cons item (car lst))
          (add-to-all item (cdr lst)))))
\end{verbatim}
\end{solution}

\question
Define \textbf{map}, which takes in a function and a list, and applies that function to each item in the list.

\begin{lstlisting}
> (map (lambda (x) (+ x 1)) '(1 2 3))
(2 3 4)
\end{lstlisting}

\begin{solution}[0.5in]
\begin{verbatim}
(define (map f lst)
  (if (null? lst)
      lst
      (cons (f (car lst)) (map f (cdr lst)))))
\end{verbatim}
\end{solution}

\question
Define \textbf{add-to-all} using one call to \textbf{map}. (Hint: consider using a lambda expression!)

\begin{solution}[0.25in]
\begin{verbatim}
(define (add-to-all item lst)
    (map (lambda (inner-lst) (cons item inner-lst)) lst))
\end{verbatim}
\end{solution}

\question
Define \textbf{sublists}. (Hint: use \textbf{add-to-all})

\begin{lstlisting}
> (sublists '(1 2 3))
(() (3) (2) (2 3) (1) (1 3) (1 2) (1 2 3))
\end{lstlisting}

\begin{solution}[0.5in]
\begin{verbatim}
Solution 1:
(define (sublists lst)
  (if (null? lst)
    '(())
    (append (sublists (cdr lst))
            (add-to-all (car lst) (sublists (cdr lst))))))

Solution 2: (using let to avoid calling sublists twice)
(define (sublists lst)
  (if (null? lst) '(())
    (let ((recur (sublists (cdr lst))))
      (append recur
              (add-to-all (car lst) recur)))))
\end{verbatim}
\end{solution}
