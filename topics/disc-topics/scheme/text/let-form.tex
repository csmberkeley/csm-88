\texttt{let} is another special form based around \texttt{lambda}. The structure
of \texttt{let} is as follows:

\begin{lstlisting}[language=Scheme]
(let ( (<SYMBOL1> <EXPR1>)
       ...
       (<SYMBOLN> <EXPRN>) )
       <BODY> )
\end{lstlisting}

This binds the results of evaluating expressions 1 through n to their associated
symbols, creating temporary variables. Finally, the body of the \texttt{let} is
evaluated.

This special form is really just equivalent to:

\begin{lstlisting}[language=Scheme]
( (lambda (<SYMBOL1> ... <SYMBOLN>) <BODY>) <EXPR1> ... <EXPRN>)
\end{lstlisting}

Think of the temporary variables as being the parameters of a lambda function.
Then, the arguments are the values of the expressions, which we bind to the
temporary variables by calling the lambda.

Consider the following example:
\begin{lstlisting}[language=Scheme]
(let ((x 1)
      (y 2))
  (+ x y))
\end{lstlisting}
This is equivalent to:
\begin{lstlisting}[language=Scheme]
((lambda (x y) (+ x y)) 1 2)
\end{lstlisting}
