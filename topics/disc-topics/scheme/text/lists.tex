All lists in Scheme are linked lists. Scheme lists are composed of two element pairs.
We define a list as being either

\begin{itemize}
\item the empty list, \texttt{nil}
\item a pair whose second element is a list
\end{itemize}

As in Python, linked lists are recursive data structures. The base case is the empty list.

We use the following procedures to construct and select from lists:
\begin{itemize}
\item {\tt (cons first rest)} constructs a list with the given first element
and rest of the list. For now, if \texttt{rest} is not a pair or \texttt{nil} it will error.
\item {\tt (car lst)} gets the first item of the list
\item {\tt (cdr lst)} gets the rest of the list
\end{itemize}

\begin{lstlisting}[language=Scheme]
scm> nil
()
scm> (define lst (cons 1 (cons 2 (cons 3 nil))))
lst
scm> lst
(1 2 3)
scm> (car lst)
1
scm> (cdr lst)
(2 3)
scm> (car (cdr lst))
2
scm> (cdr (cdr (cdr lst)))
()
\end{lstlisting}

The rule for displaying lists is very similar to that for the Link class from earlier in the class's \texttt{\_\_str\_\_} method. It prints out the elements in the linked list as if the list has no nested structure.

\begin{lstlisting}[language=Scheme]
scm> (cons 1 (cons 2 (cons 3 nil)))
(1 2 3)
scm> (cons 1 (cons (cons 2 (cons 3 nil)) nil))
(1 (2 3))
\end{lstlisting}

Two other common ways of creating lists is using the built-in \texttt{list}
procedure or the \texttt{quote} special form:

\begin{itemize}
\item The \texttt{list} procedure takes in an arbitrary amount of arguments.
Because it is a procedure, all operands are evaluated when \texttt{list} is
called. A list is constructed with the values of these operands and is
returned.
\item The \texttt{quote} special form takes in a single operand. It returns
this operand exactly as is, without evaluating it. Note that this special form
can be used to return any value, not just a list.
\end{itemize}

\begin{lstlisting}[language=Scheme]
scm> (define x 2)
scm> (list 1 x 3)
(1 2 3)
scm> (quote (1 x 3))
(1 x 3)
scm> '(1 x 3)   ; Equivalent to the previous quote expression
(1 x 3)
\end{lstlisting}
