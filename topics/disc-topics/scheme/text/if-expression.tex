An
\lstinline$if$ expression looks like this:

\centerline{\lstinline{(if <predicate> <if-true> [if-false])}}

{\tt <predicate>} and {\tt <if-true>} are required expressions and {\tt
[if-false]} is optional.

The rules for evaluation are as follows:
\begin{enumerate}
\item Evaluate {\tt <predicate>}.
\item If {\tt <predicate>} evaluates to a truth-y value, evaluate {\tt
<if-true>} and return its value. Otherwise, evaluate {\tt [if-false]} if
provided and return its value.
\end{enumerate}

This is a special form because not all operands will be evaluated! Only one of
the second and third operands is evaluated, depending on the value of the first
operand.

Remember that only \lstinline{#f} is a false-y value in Scheme; everything else
is truth-y.

\begin{lstlisting}[language=Scheme]
scm> (if (< 4 5) 1 2)
1
scm> (if #f (/ 1 0) 42)
42
\end{lstlisting}
