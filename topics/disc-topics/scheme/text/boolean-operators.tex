Like Python, Scheme also has the boolean operators \lstinline$and$,
\lstinline$or$, and \lstinline$not$.  \lstinline$and$ and \lstinline$or$ are
special forms because they are short-circuiting operators.

\begin{itemize}
\item \texttt{and} takes in any amount of operands and evaluates these operands
from left to right until one evaluates to a false-y value. It returns
that first false-y value. If there are no false-y values, it returns the value
of the last expression (or \texttt{\#t} if there are no operands)
\item \texttt{or} also evaluates any number of operands from left to right
until one evaluates to a truth-y value. It returns that first truth-y 
value. If there are no truth-y values, it returns the value of the last expression
(or \texttt{\#f} if there are no operands)
\item \texttt{not} takes in a single operand, evaluates it, and returns its
opposite truthiness value. Note that \texttt{not} is a regular procedure
and not a special form.
\end{itemize}

\begin{lstlisting}[language=Scheme]
scm> (and 25 32)
32
scm> (or 1 (/ 1 0))    ; Short-circuits
1
scm> (not (odd? 10)) 
#t
\end{lstlisting}

