Call expressions apply a procedure to some arguments.

\centerline{\lstinline{(<operator> <operand1> <operand2> ...)}}

Call expressions in Scheme work exactly like they do in Python. To evaluate
them:
\begin{enumerate}
\item Evaluate the operator to get a procedure.
\item Evaluate each of the operands from left to right.
\item Apply the value of the operator to the evaluated operands.
\end{enumerate}

For example, consider the call expression \texttt{(+ 1 2)}. First, we evaluate
the symbol \texttt{+} to get the built-in addition procedure. Then we evaluate
the two operands \texttt{1} and \texttt{2} to get their corresponding atomic
values. Finally, we apply the addition procedure to the values \texttt{1} and
\texttt{2} to get the return value \texttt{3}.

Operators may be symbols, such as \lstinline$+$ and \lstinline$*$, or more
complex expressions, as long as they evaluate to procedure values.

\begin{lstlisting}[language=Scheme]
scm> (- 1 1)                 ; 1 - 1
0
scm> (/ 8 4 2)               ; 8 / 4 / 2
1
scm> (* (+ 1 2) (+ 1 2))     ; (1 + 2) * (1 + 2)
9
\end{lstlisting}

Some important built-in functions you'll want to know are:
\begin{itemize}
\item {\tt +}, {\tt -}, {\tt *}, {\tt /}
\item {\tt =}, {\tt >}, {\tt >=}, {\tt <}, {\tt <=}
\item {\tt quotient}, {\tt modulo}, {\tt even?}, {\tt odd?}
\end{itemize}
