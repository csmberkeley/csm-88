Until now, you've been able to access names in parent frames, but you have not
been able to modify them. The {\tt nonlocal} keyword can be used to modify a
binding in a parent frame. For example, consider {\tt stepper}, which uses {\tt
nonlocal} to modify {\tt num}:

\begin{lstlisting}
def stepper(num):
    def step():
        nonlocal num  # declares num as a nonlocal name
        num = num + 1 # modifies num in the stepper frame
        return num
    return step

>>> step1 = stepper(10)
>>> step1()                # Modifies and returns num
11
>>> step1()                # num is maintained across separate calls to step
12
>>> step2 = stepper(10)    # Each returned step function keeps its own state
>>> step2()
11
\end{lstlisting}

As illustrated in this example, \texttt{nonlocal} is useful for maintaining
state across different calls to the same function.

However, there are two important caveats with {\tt nonlocal} names:
\begin{description}
  \item[$\bullet$ Global names] cannot be modified using the {\tt nonlocal}
  keyword.
  \item[$\bullet$ Names in the current frame] cannot be overridden
  using the {\tt nonlocal} keyword. This means we cannot have both a local and
  nonlocal binding with the same name in a single frame.
\end{description}

Because {\tt nonlocal} lets you modify bindings in parent frames, we call
functions that use it \textbf{mutable functions}.
