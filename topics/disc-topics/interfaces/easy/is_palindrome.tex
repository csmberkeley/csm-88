\question Write the function {\tt is\_palindrome} such that it works for any data type that 
implements the sequence interface. 

Assume that the Link class has implemented the \lstinline{\_\_len\_\_} method and a \lstinline{\_\_getitem\_\_} 
method which takes in integers.

\begin{lstlisting}
def is_palindrome(seq):
    """ Returns True if the sequence is a palindrome. A palindrome is a sequence 
    that reads the same forwards as backwards
    >>> is_palindrome(Link("l", Link("i", Link("n", Link("k")))))
    False
    >>> is_palindrome(["l", "i", "n", "k"])
    False
    >>> is_palindrome("link")
    False
    >>> is_palindrome(Link.empty)
    True
    >>> is_palindrome([])
    True
    >>> is_palindrome("")
    True
    >>> is_palindrome(Link("a", Link("v", Link("a"))))
    True
    >>> is_palindrome(["a", "v", "a"])
    True
    >>> is_palindrome("ava")
    True
    """
\end{lstlisting}
\begin{solution}[1in]
\begin{lstlisting}
    for i in range(len(seq)//2):
        if seq[i] != seq[len(seq) - 1 - i]:
          return False
    return True
\end{lstlisting}
\end{solution}
