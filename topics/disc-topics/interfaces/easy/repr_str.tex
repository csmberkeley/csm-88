\question What would Python display?

\begin{lstlisting}
class A():
    def __init__(self, x):
        self.x = x
    def __repr__(self):
         return self.x
    def __str__(self):
         return self.x * 2

class B():
    def __init__(self):
         print("boo!")
         self.a = []
    def add_a(self, a):
         self.a.append(a)
    def __repr__(self):
         print(len(self.a))
         ret = ""
         for a in self.a:
             ret += str(a)
         return ret

>>> A("one")
\end{lstlisting}
\begin{solution}[0.25in]
\lstinline{one}
\end{solution}

\begin{lstlisting}
>>> print(A("one"))
\end{lstlisting}
\begin{solution}[0.25in]
\lstinline{oneone}
\end{solution}

\begin{lstlisting}
>>> repr(A("two"))
\end{lstlisting}
\begin{solution}[0.25in]
\lstinline{'two'}
\end{solution}

\begin{lstlisting}
>>> b = B()
\end{lstlisting}
\begin{solution}[0.25in]
\lstinline{boo!}
\end{solution}

\begin{lstlisting}
>>> b.add_a(A("a"))
>>> b.add_a(A("b"))
>>> b
\end{lstlisting}
\begin{solution}[0.25in]
\lstinline{2}
\end{solution}
\begin{solution}[0.25in]
\lstinline{aabb}
\end{solution}
\begin{lstlisting}
>>> c = A("c")
>>> b.add_a(c)
>>> str(b)
\end{lstlisting}
\begin{solution}[0.25in]
\lstinline{3}
\end{solution}
\begin{solution}[0.25in]
\lstinline{'aabbcc'}
\end{solution}
