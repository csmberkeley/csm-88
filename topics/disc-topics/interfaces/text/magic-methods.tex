Python defines many interfaces that can be implemented by user-defined classes.
For example, the interface for arithmetic consists of the following methods:

\begin{itemize}
\item
\texttt{def \_\_add\_\_(self, other):} Allows objects to do \texttt{self +
other}.
\item
\texttt{def \_\_sub\_\_(self, other):} Allows objects to do \texttt{self -
other}.
\item
\texttt{def \_\_mul\_\_(self, other):} Allows objects to do \texttt{self *
other}.
\end{itemize}

In addition, there is also an interface for sequences:

\begin{itemize}
\item
\texttt{def \_\_len\_\_(self):} Allows objects to do \texttt{len(self)}.
\item
\texttt{def \_\_getitem\_\_(self, index):} Allows objects to do
\texttt{self[i]}.
\end{itemize}
