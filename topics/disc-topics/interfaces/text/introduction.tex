In computer science, an \define{interface} is a shared set of attributes, along
with a specification of the attributes' behavior.  For example, an interface for
vehicles might consist of the following methods:
\begin{itemize}
\item \texttt{def drive(self):} Drives the vehicle if it has stopped.
\item \texttt{def stop(self):} Stops the vehicle if it is driving.
\end{itemize}
Data types can implement the same interface in different ways. For example, a
\texttt{Car} class and a \texttt{Train} can both implement the interface
described above, but the \texttt{Car} probably has a different mechanism for
\texttt{drive} than the \texttt{Train}.

The power of interfaces is that other programs don't have to know \emph{how}
each data type implements the interface -- only that they \emph{have}
implemented the interface. The following \texttt{travel} function can work with
both \texttt{Car}s and \texttt{Train}s:

\begin{lstlisting}
def travel(vehicle):
    while not at_destination():
        vehicle.drive()
    vehicle.stop()
\end{lstlisting}
