\section*{Secrets to Success in CS 61A}
CS 61A is definitely a challenge, but we all want you to learn and
succeed, so here are a few tips that might help:

\begin{itemize}
\item Ask questions. When you encounter something you don't know, {\it
ask}. That is what we are here for. This is not to say you should raise your
hand impulsively, but you are going to see a lot of challenging stuff in this
class, and you can always come to us for help.
\item Study in groups. Again, this class is not trivial; you might feel
overwhelmed going at it alone. Work with someone, either on homework, on lab, or
for midterms, as long as you don't violate the cheating policy!
\item Go to office hours. Office hours give you time with the instructor or TAs
by themselves, and you will be able to get some (nearly) one-on-one instruction
to clear up confusion. You are {\it not} intruding; the instructors and TAs {\it
like} to teach!  Remember, if you cannot make office hours, you can always
make separate appointments with us!
\item Do (or at least attempt seriously) all the homework.  We do not give many
homework problems, but those we do give are challenging, time-consuming, and
rewarding. The fact that homework is graded on effort does not imply that you
should ignore it: it will be one of your primary sources of preparation and
understanding.
\item Do all the lab exercises. Most of them are simple and take no more than an
hour or two. This is a great time to get acquainted with new material. If you do
not finish, work on it at home, and come to office hours if you need more
guidance!
\item Optional lab questions are ``optional'' in the sense that they are extra
practice, not that they are material that's out of scope. Make sure you do them
if you have time!
\item Do the readings before lecture! There is a reason why they are
assigned. And it is not because we are evil; that is only partially true.
\item When preparing for the midterms and final, do past exam questions! Lecture,
lab, and discussion provide a great introduction to the material, but the only way
to learn how to solve exam-level problems is to do exam-level problems.
\item Most importantly, {\it have fun!}
\end{itemize}
