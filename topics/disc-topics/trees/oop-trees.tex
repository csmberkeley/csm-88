\question
Define \textbf{filter\_tree}, which takes in a tree \textbf{t} and one argument predicate function \textbf{fn}. It should mutate the tree by removing all branches of any node where calling \textbf{fn} on its label returns \textbf{False}. In addition, if this node is not the root of the tree, it should remove that node from the tree as well.

\begin{lstlisting}[language=Python]
def filter_tree(t, fn):
    """
    >>> t = Tree(1, [Tree(2), Tree(3, [Tree(4)]), Tree(6, [Tree(7)])])
    >>> filter_tree(t, lambda x: x % 2 != 0)
    >>> t
    tree(1, [Tree(3)])
    >>> t2 = Tree(2, [Tree(3), Tree(4), Tree(5)])
    >>> filter_tree(t2, lambda x: x != 2)
    >>> t2
    Tree(2)
    """
    |\begin{solution}[1in]
    \begin{verbatim}
    if not fn(t.label):
        t.branches = []
    else:
        for b in t.branches[:]:
            if not fn(b.label): 
                t.branches.remove(b)
            else: 
                filter_tree(b, fn)
    \end{verbatim}
    \end{solution}|
\end{lstlisting}

\question
Fill in the definition for \textbf{nth\_level\_tree\_map}, which also takes in a function and a tree, but mutates the tree by applying the function to every nth level in the tree, where the root is the 0th level.

\begin{lstlisting}[language=Python]
def nth_level_tree_map(fn, tree, n):
   """Mutates a tree by mapping a function all the elements of a tree.
   >>> tree = Tree(1, [Tree(7, [Tree(3), Tree(4), Tree(5)]),
                Tree(2, [Tree(6), Tree(4)])])
   >>> nth_level_tree_map(lambda x: x + 1, tree, 2)
   >>> tree
   Tree(2, [Tree(7, [Tree(4), Tree(5), Tree(6)]), 
            Tree(2, [Tree(7), Tree(5)])])
   """
    |\begin{solution}[1in]
    \begin{verbatim}
    def helper(tree, level):
          if level % n == 0:
               tree.label = fn(tree.label)
          for b in tree.branches:
               helper(b, level + 1)
  	helper(tree, 0)
    \end{verbatim}
    \end{solution}|
\end{lstlisting}
