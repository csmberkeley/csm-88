\question We can represent the hailstone sequence as a tree in the figure below,
showing the route different numbers take to reach 1. Remember that a hailstone
sequence starts with a number $n$, continuing to $n/2$ if $n$ is even or $3n+1$
if $n$ is odd, ending with 1. Write a function \texttt{hailstone\_tree(n, h)}
which generates a tree of height \texttt{h}, containing hailstone numbers that
will reach \texttt{n}.

\bigskip

\textit{Hint:} A node of a hailstone tree will always have at least one, and at most two branches (which are also hailstone trees). Under what conditions do you add the second branch?

\begin{center}
  \begin{tikzpicture}[grow'=right,level 1/.style={level distance=1.5cm}]
    \Tree [.1 [.2 [.4 [.8 [.16 [.32 [.64 128 21 ] ] [.5 [.10 20 3 ] ] ] ] ] ] ]
  \end{tikzpicture}

  \texttt{hailstone\_tree(1, 7)}

  \begin{tikzpicture}[grow'=right,level 1/.style={level distance=1.5cm}]
    \Tree [.16 [.32 [.64 128 21 ] ] [.5 [.10 20 3 ] ] ]
  \end{tikzpicture}

  \texttt{hailstone\_tree(16, 3)}

\end{center}

\begin{lstlisting}
def hailstone_tree(n, h):
    """Generates a tree of hailstone numbers that will
       reach N, with height H.
    >>> hailstone_tree(1, 0)
    [1]
    >>> hailstone_tree(1, 4)
    [1, [2, [4, [8, [16]]]]]
    >>> hailstone_tree(8, 3)
    [8, [16, [32, [64]], [5, [10]]]]
    """
\end{lstlisting}
\begin{solution}
\begin{lstlisting}
    if h == 0:
        return tree(n)
    branches = [hailstone_tree(n * 2, h - 1)]
    if (n - 1) % 3 == 0 and ((n - 1) // 3) % 2 == 1 and (n - 1) // 3 > 1:
        branches += [hailstone_tree((n - 1) // 3, h - 1)]
    return tree(n, branches)
\end{lstlisting}
\end{solution}
