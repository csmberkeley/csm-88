\question An \define{expression tree} is a tree that contains a function for
each non-leaf node, which can be either \texttt{'+'} or \texttt{'*'}. All leaves
are numbers. Implement \texttt{eval\char`_tree}, which evaluates an expression tree
to its value. You may want to use the functions \texttt{sum} and
\texttt{prod}, which take a list of numbers and compute the sum and product
respectively.

\begin{lstlisting}
def eval_tree(tree):
    """Evaluates an expression tree with functions the root.
    >>> eval_tree(tree(1))
    1
    >>> expr = tree('*', [tree(2), tree(3)])
    >>> eval_tree(expr)
    6
    >>> eval_tree(tree('+', [expr, tree(4), tree(5)]))
    15
    """
\end{lstlisting}
\begin{solution}[1.2in]
\begin{lstlisting}
    if is_leaf(tree):
        return label(tree)
    args = [eval_tree(branch) for branch in branches(tree)]
    if label(tree) == '+':
        return sum(args)
    else: # label(tree) == '*'
        return prod(args)
\end{lstlisting}
Leaf values are guaranteed to be a number, so we can just return their label.

Otherwise, we have to evaulate each of the branches and then combine their
result using whichever operater we have in our current root.

If you want to try this out yourself, note that \texttt{prod} isn't actually a
built-in operator in Python. You can write it yourself using something like the
following:
\begin{lstlisting}
def prod(iterable):
    from functools import reduce
    from operator import mul
    return reduce(mul, iterable, 1)
\end{lstlisting}
\href{https://www.youtube.com/watch?v=Am6m8YgAnYY&list=PLx38hZJ5RLZdJgRCgpaTbmRXKAHOUmomO&index=4&t=12m40s}{Video walkthrough}
\end{solution}
