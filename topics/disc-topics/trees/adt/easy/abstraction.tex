\question
Consider a tree ADT \texttt{t} constructed by calling \texttt{tree(1, [tree(2),
tree(4)])}.  For each of the following expressions, answer these two questions:

\begin{itemize}
\item What does the expression evaluate to?
\item Does the expression violate any abstraction barriers? If so, write an
equivalent expression that does not violate abstraction barriers.
\end{itemize}

\begin{enumerate}
\item \texttt{label(t)}
\begin{solution}[0.20in]
Evaluates to 1, the label of the entire tree. This is simply using a selector
to get the label, which is not violating any abstraction barriers.
\end{solution}
\item \texttt{t[0]}
\begin{solution}[0.20in]
This expression evaluates to 1, the label of the entire tree. However, it makes
use of the fact that trees are implemented using lists, and violates the
abstraction barrier. An equivalent expression is \texttt{label(t)}.
\end{solution}
\item \texttt{label(branches(t)[0])}
\begin{solution}[0.20in]
This expression evaluates to the label of the first branch of \texttt{t}.  It
is not a violation to index into \texttt{branches(t)} because it is given in
the description of the ADT that \texttt{branches(t)} returns a list of
branches.
\end{solution}
\item \texttt{is\_leaf(t[1:][1])}
\begin{solution}[0.20in]
This expression accesses the branches of \texttt{t} by slicing \texttt{t}.
Although this works because this is technically what \texttt{branches(t)}
returns, this is an abstraction violation because we cannot assume the
implementation of \texttt{branches(t)}.

It then accesses the second branch by indexing into the list of branches, which
is \emph{not} an abstraction violation because we are allowed to assume that
branches is a list.  This expression evaluates to \texttt{True} because the
second branch of \texttt{t} is a leaf. An equivalent expression is
\texttt{is\_leaf(branches(t)[1])}.
\end{solution}
\item \texttt{[label(b) for b in branches(t)]}
\begin{solution}[0.20in]
This expression uses the \texttt{branches} selector to access the branches of
\texttt{t} and then iterates through it to construct a new list containing the
labels of the branches. The result list is \texttt{[2, 4]}. It does not violate
any abstraction barriers.
\end{solution}
\item \textbf{Challenge:} \texttt{branches(tree(5, [t, tree(3)]))[0][0]}
\begin{solution}[0.20in]
This expression evaluates to the label of the tree \texttt{t}, which is 1. This
is because the expression \texttt{tree(5, [t, tree(3)])} evaluates to a tree
whose first branch is the tree \texttt{t} that we constructed above! Howvever,
this expression violates the abstraction barrier by indexing into \texttt{t} to
get its label. An equivalent expression would be \texttt{label(branches(tree(5,
[t, tree(3)]))[0])}.
\end{solution}
\end{enumerate}
