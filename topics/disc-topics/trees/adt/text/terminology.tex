Some terminology regarding trees:
\begin{itemize}
\item \define{Parent node}: A node that has branches. Parent nodes can have
  multiple branches.
\item \define{Child node}: A node that has a parent. A child node can only
  belong to one parent.
\item \define{Root}: The top node of the tree. In our example, the node that
  contains $7$ is the root.
\item \define{Label}: The value at a node. In our example, all of the integers
  are values.
\item \define{Leaf}: A node that has no branches. In our example, the nodes that
  contain $-4$, $0$, $6$, $17$, and $20$ are leaves.
\item \define{Branch}: A subtree of the root. Note that trees have branches, which are trees themselves: this is why trees are \emph{recursive} data structures.
\item \define{Depth}: How far away a node is from the root. In other
  words, the number of edges between the root of the tree to the
  node. In the diagram, the node containing $19$ has depth 1; the node
  containing $3$ has depth 2. Since there are no edges between the
  root of the tree and itself, the depth of the root is 0.
\item \define{Height}: The depth of the lowest leaf. In the diagram, the nodes
  containing $-4$, $0$, $6$, and $17$ are all the ``lowest leaves,'' and they
  have depth 4. Thus, the entire tree has height 4.
\end{itemize}

In computer science, there are many different types of trees. Some vary in
the number of branches each node has; others vary in the structure of the
tree.
