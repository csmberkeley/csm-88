A tree has both a value for the root node and a sequence of branches,
which are also trees. In our implementation, we represent the branches as
a list of trees. Since a tree is an abstract data type, our choice to use
lists is just an implementation detail.

\begin{itemize}
\item The arguments to the constructor \texttt{tree} are the value for
    the root node and a list of branches.
\item The selectors for these are \texttt{label} and \texttt{branches}.
\end{itemize}

\medskip

Note that \texttt{branches} returns a list of trees and not a tree directly.
It's important to distinguish between working with a tree and working with a
\textbf{list of} trees.

We have also provided a convenience function, \texttt{is\char`_leaf}.

\medskip
Let's try to create the tree below.

\begin{center}
\begin{tikzpicture}[very thick,level/.style={sibling distance=30mm/#1},
                    level distance=24pt]
\node [vertex] (r){$1$}
  child {
    node [vertex] (a) {$3$}
    child {
      node [vertex] {$4$}
    }
    child {
      node [vertex] {$5$}
    }
    child {
      node [vertex] {$6$}
    }
  }
  child {
    node [vertex] {$2$}
  };
\end{tikzpicture}
\end{center}

\begin{lstlisting}
# Example tree construction
t = tree(1,
      [tree(3,
          [tree(4),
           tree(5),
           tree(6)]),
      tree(2)])
\end{lstlisting}
