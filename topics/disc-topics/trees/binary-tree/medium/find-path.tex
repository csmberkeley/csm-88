\question Define the procedure {\tt find\_path} that, given a BinTree {\tt t} and
a value {\tt root}, returns a list containing the nodes along the path
required to get from the root of {\tt t} to {\tt root}. If {\tt root} is not
present in {\tt t}, return {\tt False}. Assume that the elements in {\tt t} are
unique.

For instance, for the following tree, {\tt find\_path} should return:
\begin{center}
\begin{tikzpicture}[very thick,level/.style={sibling distance=70mm/#1},
                    level distance=24pt]
\node [vertex] (r){$2$}
  child {
    node [vertex] (a) {$7$}
    child {
      node [vertex] {$2$}
    }
    child {
      node [vertex] {$6$}
      child {
        node [vertex] {$5$}
      }
      child {
        node [vertex] {$11$}
      }
    }
  }
  child {
    node [vertex] {$15$}
  };
\end{tikzpicture}
\end{center}

\begin{lstlisting}
>>> find_path(tree_ex, 5)
[2, 7, 6, 5]
\end{lstlisting}

\begin{lstlisting}
def find_path(t, root):
\end{lstlisting}
\begin{solution}[1in]
\begin{lstlisting}
    if not t:
        return False
    if t.root == root:
        return [root]
    for child in (t.left, t.right):
        path = find_path(child, root)
        if path:
            return [t.root] + path
    return False
\end{lstlisting}
\end{solution}
