\question Language can be modeled as a binary tree, where each node is a part of speech. Using such a syntactic tree, we can model question formation by moving the question
word in the tree.
\begin{center}
\begin{tikzpicture}[sibling distance=5mm,level distance=7mm]
  \Tree [.S [.NP \`{A}(T\'{u}) ]
            [.VP  [.V escribiste ]
                      [.PP [.P a ] [.NP qui\'{e}n? ] ] ] ]
\end{tikzpicture}
$\Longrightarrow$
\begin{tikzpicture}[sibling distance=5mm,level distance=7mm]
  \Tree [.S [.PP \`{A}A qui\'{e}n ]
            [.VP  [.V escribiste ]
                      [.NP (t\'{u})? ] ] ]
\end{tikzpicture}
\\
Who did you write to?
\end{center}

\begin{enumerate}[a.]
    \item How would we modify the Tree class so that each node remembers its
        parent?
        \begin{solution}[1in]
            One possible solution is:
\begin{lstlisting}
class Tree:
    def __init__(self, entry, left=None, right=None):
        self.entry = entry
        self.left, self.right = left, right
        if self.left is not None:
            self.left.parent = self
        if self.right is not None:
            self.right.parent = self
\end{lstlisting}
            \end{solution}

    \item Now write a method that swaps a tree's own left child with the right
        child of {\tt other}. Assume that this function is part of the \texttt{Tree} class.
\begin{lstlisting}
def left_swap(self, other):
\end{lstlisting}
\begin{solution}[1in]
\begin{lstlisting}
    assert self.left is not None and other.right is not None, "Must have children to swap."
    self.left, other.right = other.right, self.left
    self.left.parent, other.right.parent = self other
\end{lstlisting}

The important part here is that the parent pointers must be updated as well.
\end{solution}

\end{enumerate}
