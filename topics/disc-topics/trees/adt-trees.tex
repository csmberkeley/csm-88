\question 
Fill in this implementation of the Tree ADT.

\begin{lstlisting}[language=Python]
def tree(label, branches = []):
    for b in branches:
        assert is_tree(b), 'branches must be trees'
    return [label] + list(branches)

def is_tree(tree):
    if type(tree) != list or len(tree) < 1:
        return False
    for b in branches(tree):
        if not is_tree(b):
            return False
    return True

def label(tree):
    |\begin{solution}
    \begin{verbatim}
    return tree[0] 
\end{verbatim}
\end{solution}|

def branches(tree):
    |\begin{solution}
    \begin{verbatim}
    return tree[1:] 
\end{verbatim}
\end{solution}|

def is_leaf(tree):
    |\begin{solution}
    \begin{verbatim}
    return not branches(tree) 
\end{verbatim}
\end{solution}|
\end{lstlisting}

\question
A min-heap is a tree with the special property that every node’s value is less than or equal to the values of all of its children. For example, the following tree is a min-heap:
\begin{verbatim}
                               1
                            /  |  \
                          5    3   6
                          |   / \  
                          7  9   4
\end{verbatim}
However, the following tree is not a min-heap because the node with value 3 has a value greater than one of its children:
\begin{verbatim}
                               1
                            /  |  \
                          5    3   6
                          |   / \  
                          7  9   2
\end{verbatim}
\newpage
Write a function \textbf{is\_min\_heap} that takes a tree and returns True if the tree is a min-heap and False otherwise.
\begin{lstlisting}
def is_min_heap(t):
    |\begin{solution}[0.75in]
    \begin{verbatim}
    for b in branches(t):
        if label(t) > label(b) or not is_min_heap(b):
             return False
    return True    
    \end{verbatim}
    \end{solution}|
\end{lstlisting}

\question
Write a function \textbf{largest\_product\_path} that finds the largest product path possible. A product path is defined as the product of all nodes between the root and a leaf. The function takes a tree as its parameter. Assume all nodes have a non-negative value.
\begin{verbatim}
                               3
                            /  |  \
                          7    8   4
                          |    |  
                          2    1    
\end{verbatim}

For example, calling \textbf{largest\_product\_path} on the above tree would return 42, since 3 * 7 * 2 is the largest product path.
\begin{lstlisting}
def largest_product_path(tree):
    """
    >>> largest_product_path(None)
    0
    >>> largest_product_path(tree(3))
    3
    >>> t = tree(3, [tree(7, [tree(2)]), tree(8, [tree(1)]), tree(4)])
    >>> largest_product_path(t)
    42
    """
    |\begin{solution}[1in]
    \begin{verbatim}
    if not tree:
        return 0
    elif is_leaf(tree):
        return label(tree)
    else:
        paths = [largest_product_path(t) for t in branches(tree)]
        return label(tree) * max(paths)
    \end{verbatim}
    \end{solution}|

\end{lstlisting}
\question 
Check your understanding:
\begin{paragraph}
1  Given the first tree in 4.2, write the corresponding python call to create the tree 
\end{paragraph}
\begin{solution}
tree(1, [tree(5, [tree(7)]),tree(3,[tree(9),tree(4)]),tree(6)])
\end{solution}
\begin{paragraph}
2 What is the benefit of using a tree as a data structure, rather than a list or linked list? 
\end{paragraph}
\begin{solution}
Trees can be organized more effectively (in the case of a binary search tree), and in the general case you can tell whether or not an element is in that tree faster than if it were in a linked list or a list. You can also search faster in a BST. Trees can also be used to express hierarchy. 
\end{solution}
\begin{paragraph}
3 Below is the function contains, which takes in an input of a tree, t and a value, e. The function returns true if e exists as a label inside t. However, the function does not work properly, debug this code and find the error(s). 
\end{paragraph}
\begin{lstlisting}
def contains(t, e):
     if is_leaf(t):
        return False
     elif e == label(t):
        return True
     else:
        for b in branches(t):
            return contains(b, e)
        return True
\end{lstlisting}
\begin{solution}
Errors: \\
-One should check the label before checking if the tree is a leaf, in case the leaf contains the value of the tree \\
-In the for loop it should not return contains(b,e), but rather return True, if contains(b,e) evaluates to true. \\ 
-At the end, if we have searched through all the trees, we should return False if the value, e is not found.  

\end{solution}
\begin{paragraph}
4 Implement a function max\_tree, which takes a tree t. It returns a new tree with the exact same structure as t; at each node in the new tree, the entry is the largest number that is contained in that node's subtrees or the corresponding node in t. 
\end{paragraph}
\begin{lstlisting}
def max_tree(t):
>>> max_tree(tree(1, [tree(5, [tree(7)]),tree(3,[tree(9),tree(4)]),tree(6)]))
    tree(9, [tree(7, [tree(7)]),tree(9,[tree(9),tree(4)]),tree(6)])
    if ___________:
        return _________________
    else:
    	new_branches= ______________________________
    	new_label = __________________________________
    	return ______________________
\end{lstlisting}
\begin{solution}
\begin{lstlisting}
def max_tree(t):
     if is_leaf(t):
    	return tree(root(t))
     else:
    	new_branches = [max_tree(b) for b in branches(t)]
    	new_label = max([root(t)] + [root(s) for s in new_branches])
    	return tree(new_label, new_branches)
\end{lstlisting}
\end{solution}
\newpage
\question
Challenge Question: The level-order traversal of a tree is defined as visiting the nodes in each level of a tree before moving onto the nodes in the next level. For example, the level order of the following tree is: 3 7 8 4 
\begin{verbatim}
                               3
                            /  |  \
                          7    8   4 
\end{verbatim}
 Write a function \textbf{level\_order} that takes in a tree as the parameter and returns a list of the values of the nodes in level order. 
 
 \begin{lstlisting}
 def level_order(tree):
 |\begin{solution}
 \begin{verbatim}
 #iterative solution
def level_order(tree)
    if not tree:
        return []
    current_level, next_level = [label(tree)], [tree]
    while next_level:
        find_next= []
        for b in next_level:
            find_next.extend(branches(b))
        next_level = find_next
        current_level.extend([label(t) for t in next_level])
    return current_level
    
 \end{verbatim}
 \end{solution}|
 \end{lstlisting}
 \hfill\break
\hfill\break
\hfill\break
 \hfill\break
\hfill\break
\hfill\break

\question 
Challenge Question: Write a function \textbf{all\_paths} which will return a list of lists of all the possible paths of an input tree, t. When the function is called on the same tree as the problem above, the function would return: $[[3,7],[3,8],[3,4]]$
\begin{lstlisting}
def all_paths(t):
    if _______________:
        ____________________
    else:
        ___________________
        ____________________
            ________________________
                ______________________
        __________________
\end{lstlisting}
\begin{solution}
\begin{lstlisting}
def all_paths(t):
    if is_leaf(t):
        return [[label(t)]]
    else:
        total = []
        for b in branches(t):
            for path in all_paths(b):
                total.append([label(t)] + path)
        return total
\end{lstlisting}
\end{solution}

