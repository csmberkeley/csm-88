In a binary tree, each tree node has at most two subtrees: \texttt{left} and 
\texttt{right}. We can use a binary tree to create a 
\textit{binary search tree}. A binary search tree is a binary tree with
the following additional constraints: 
\begin{itemize}
\item all entries in the left side of the tree are smaller than the root entry
\item all entries in the right side of the tree are bigger than the root entry
\end{itemize}
If we organize data in a binary search tree, we can find a specific element very quickly. How? Compare the element we want with the root entry, and if the element is smaller, search the left side; if bigger, search the right side; recurse!
