\question Define the procedure \texttt{find\_path} that, given a Tree \texttt{t} and
an \texttt{entry}, returns a list containing the nodes along the path
required to get from the root of \texttt{t} to \texttt{entry}. If \texttt{entry} is not
present in \texttt{t}, return \texttt{False}.\\
Assume that the elements in \texttt{t} are unique. Find the path to an
        element.

        For instance, for the following tree, \texttt{find\_path} should return:
        \begin{center}
        \begin{tikzpicture}[very thick,level/.style={sibling distance=70mm/#1},
                            level distance=24pt]
        \node [vertex] (r){$2$}
          child {
            node [vertex] (a) {$7$}
            child {
              node [vertex] {$3$}
            }
            child {
              node [vertex] {$6$}
              child {
                node [vertex] {$5$}
              }
              child {
                node [vertex] {$11$}
              }
            }
          }
          child {
            node [vertex] {$1$}
          };
        \end{tikzpicture}
        \end{center}

        \begin{lstlisting}
        >>> find_path(tree_ex, 5)
        [2, 7, 6, 5]
        \end{lstlisting}

        \begin{lstlisting}
        def find_path(t, entry):
        \end{lstlisting}
        \begin{solution}[2.5in]
        \begin{lstlisting}
            if t.entry == entry:
                return [entry]
            for subtree in t.subtrees:
                path = find_path(subtree, entry)
                if path:
                    return [t.entry] + path
            return False
        \end{lstlisting}
        \end{solution}

    \item Now assume that the elements of the tree might not be unique. How
        would you change your answer from part a to find the shortest path?  Try
        to implement the function \texttt{find\_shortest}, which has the same
        parameters as \texttt{find\_path}.

        \begin{solution}[4.5in]
        We want to check every child's entry before moving on to that child's
        children.  It's easier to do this using an iterative implementation
        rather than a recursive implementation. In these two functions, we are
        actually running depth-first search and breadth-first search
        respectively, both of which are well known graph-traversal algorithms.
        For CS61A, you just have to recognize that a function call does not
        return until it is fully evaluated, so the \texttt{
        find\_path} solution might not give the shortest path.

        \begin{lstlisting}
        def find_shortest(t, entry):
            tree_queue = [(t, [])]
            while len(tree_queue) > 0:
                curr_tree, path = tree_queue.pop(0)
                if curr_tree.entry == entry:
                    return path + [entry]
                for branch in curr_tree.branches:
                    tree_queue.append((branch, path + [curr_tree.entry]))
            return False
        \end{lstlisting}
        \end{solution}
