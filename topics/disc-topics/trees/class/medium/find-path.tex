\question Define the procedure \texttt{find\_path} that, given a Tree \texttt{t} and
an \texttt{entry}, returns a list containing the nodes along the path
required to get from the root of \texttt{t} to \texttt{entry}. If \texttt{entry} is not
present in \texttt{t}, return \texttt{False}.\\
Assume that the elements in \texttt{t} are unique. Find the path to an
element.

For instance, for the following tree \texttt{tree\_ex}, \texttt{find\_path} should return:
\begin{center}
\begin{tikzpicture}[very thick,level/.style={sibling distance=70mm/#1},
                    level distance=24pt]
\node [vertex] (r){$2$}
  child {
    node [vertex] (a) {$7$}
    child {
      node [vertex] {$3$}
    }
    child {
      node [vertex] {$6$}
      child {
        node [vertex] {$5$}
      }
      child {
        node [vertex] {$11$}
      }
    }
  }
  child {
    node [vertex] {$1$}
  };
\end{tikzpicture}
\end{center}

\begin{lstlisting}
>>> find_path(tree_ex, 5)
[2, 7, 6, 5]
\end{lstlisting}

\begin{lstlisting}
def find_path(t, entry):
\end{lstlisting}
\begin{solution}[2.5in]
\begin{lstlisting}
    if t.entry == entry:
        return [entry]
    for subtree in t.subtrees:
        path = find_path(subtree, entry)
        if path:
            return [t.entry] + path
    return False
\end{lstlisting}
\end{solution}
