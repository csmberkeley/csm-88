\question Write a generator function \texttt{combiner} that combines two input
iterators using a given combiner function. The resulting iterator should have a
size equal to the size of the shorter of its two input iterators.

\begin{minipage}{\textwidth}
\begin{lstlisting}[language=Python]
>>> from operator import add
>>> evens = combiner(gen_naturals(), gen_naturals(), add)
>>> next(evens)
0
>>> next(evens)
2
>>> next(evens)
4
\end{lstlisting}
\begin{lstlisting}[language=Python]
def combiner(iterator1, iterator2, combiner):
\end{lstlisting}

\begin{solution}[0.75in]
\begin{lstlisting}[language=Python]
    while True:
        yield combiner(next(iterator1), next(iterator2))
\end{lstlisting}
While this is the most compact solution, it may not be immediately obvious that
we would arrive at this. It's acceptable to start with the ``basic skeleton'' of
all generators:

\begin{lstlisting}
    while True:
        <do some work here>
        yield <something>
        <do some other work here>
\end{lstlisting}

From this, we put in some basic steps:
\begin{itemize}
    \item We want to fetch the next item from both our iterators.
    \item Then, we would want to combine them using our combiner function.
    \item Finally, we want to yield the result (be very careful not to return!).
\end{itemize}

\begin{lstlisting}
    while True:
        n1 = next(iterator1)
        n2 = next(iterator2)
        result = combiner(n1, n2)
\end{lstlisting}

\end{solution}
\end{minipage}

\question What is the result of executing this sequence of commands?
\begin{lstlisting}[language=Python]
>>> nats = gen_naturals()
>>> doubled_nats = combiner(nats, nats, add)
>>> next(doubled_nats)
\end{lstlisting}

\begin{solution}[0.25in]
1
\end{solution}
\begin{lstlisting}[language=Python]
>>> next(doubled_nats)
\end{lstlisting}
\begin{solution}[0.25in]
5
\end{solution}
\begin{solution}
The naturals iterator has been fed into \texttt{combiner} twice. So the
first yield will get the first two numbers out of naturals, the second yield
will the the third and fourth numbers, and so on.
\begin{itemize}
    \item 0 + 1 = 1
    \item 2 + 3 = 5
\end{itemize}
If you expected this to return 0 then 2, think about what would need to
be changed in how we use \texttt{combiner}. Also, let's assume we don't want to
change the behaviour of the \texttt{combiner} function.
\end{solution}



\walkthrough{https://www.youtube.com/watch?v=2WRsGzuR7tA&list=PLx38hZJ5RLZccXWDLoOVdTsx8VT60WYgn&index=5}
