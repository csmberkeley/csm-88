The \texttt{yield from} statement is similar to a \texttt{yield} statement.
\texttt{yield from} takes in an iterator and yields each of the values from
that iterator. It can be used in conjunction with other \texttt{yield}s and
\texttt{yield from}s.

\vspace{1em}
\begin{lstlisting}[language=Python]
>>> square = lambda x: x*x
>>> def many_squares(s):
...     for x in s:
...         yield square(x)
...     yield from [square(x) for x in s]
...     yield from map(square, s)
...
>>> list(many_squares([1, 2, 3]))
[1, 4, 9, 1, 4, 9, 1, 4, 9]
\end{lstlisting}

When the \texttt{list} function in Python receives an iterator, it calls the
\texttt{next} function on the input until it raises a \texttt{StopIteration}. It
puts each of the elements from the calls to \texttt{next} into a new list and
returns it.
