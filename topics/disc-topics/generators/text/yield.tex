The \lstinline$yield$ statement is similar to a \lstinline$return$ statement.
However, while a \lstinline$return$ statement closes the current frame after the
function exits, a \lstinline$yield$ statement causes the frame to be saved until
the next time \lstinline$next$ is called, which allows the generator to
automatically keep track of the iteration state.

Once \lstinline$next$ is called again, execution resumes where it last
stopped and continues until the next \lstinline$yield$ statement or the end of
the function. A generator function can have multiple \lstinline$yield$
statements.

Including a \lstinline$yield$ statement in a function automatically tells Python
that this function will create a generator. When we call the function, it
returns a generator object instead of executing the body.  When the
generator's \lstinline$next$ method is called, the body is executed until
the next \lstinline$yield$ statement is executed.

