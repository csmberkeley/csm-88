We can make our own classes iterable using the \lstinline$__iter__$ method,
which returns an iterator object. Because generators are technically iterators,
you can implement \lstinline$__iter__$ methods using them. For example:

\begin{lstlisting}
class Naturals():
    def __iter__(self):
        current = 0
        while True:
            yield current
            current += 1
\end{lstlisting}

\lstinline$Naturals$'s \lstinline$__iter__$ method now returns a generator
object. The behavior of \lstinline$Naturals$ is almost the same as before:

\begin{lstlisting}
>>> nats = Naturals()
>>> nats_iterator1 = iter(nats)
>>> next(nats_iterator1)
0
>>> next(nats_iterator1)
1
>>> nats_iterator2 = iter(nats)
>>> next(nats_iterator2)
0
\end{lstlisting}

In this example, we can iterate over the same object more than once by calling
\textbf{iter} multiple times. Note that \lstinline$nats$ is an iterable
object and the \lstinline$nats_iterator$'s are generators.
