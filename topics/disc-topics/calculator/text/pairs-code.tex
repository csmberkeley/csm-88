Here's an implementation of what we described:

\begin{blocksection}
\begin{lstlisting}
class Pair:
    """Represents the built-in pair data structure in Scheme."""
    def __init__(self, first, second):
        self.first = first
        if not scheme_valid_cdrp(second):
            raise SchemeError("cdr can only be a pair, nil, or a promise but was {}".format(second))
        self.second = second

    def map(self, fn):
        """Maps fn to every element in a list, returning a new
        Pair.

        >>> Pair(1, Pair(2, Pair(3, nil))).map(lambda x: x * x)
        Pair(1, Pair(4, Pair(9, nil)))
        """
        assert isinstance(self.second, Pair) or self.second is nil, \
            "Second element in pair must be another pair or nil"
        return Pair(fn(self.first), self.second.map(fn))

    def __repr__(self):
        return 'Pair({}, {})'.format(self.first, self.second)
\end{lstlisting}

%\begin{lstlisting}
%    def __getitem__(self, i):
%       """Allows us to index into lists and treat them like Python iterables.

%       >>> p = Pair(1, Pair(2, Pair(3, nil)))
%       >>> p[1]
%       2
%       >>> list(p)
%       [1, 2, 3]
%       """
%       assert isinstance(self.second, Pair) or self.second is nil, \
%         "Second element in pair must be another pair or nil"
%       if i == 0:
%            return self.first
%       return self.second[i - 1]

%    def __len__(self):
%        return 1 + len(self.second)
%\end{lstlisting}

\begin{lstlisting}
class nil:
    """Represents the special empty pair nil in Scheme."""
    def map(self, fn):
        return nil
    def __getitem__(self, i):
        raise IndexError('Index out of range')
    def __repr__(self):
        return 'nil'

nil = nil() # this hides the nil class *forever*
\end{lstlisting}
\end{blocksection}
