The evaluation component of an interpreter determines the type of an expression
and executes corresponding evaluation rules.

Here are the evaluation rules for the three types of Calculator expressions:

\begin{enumerate}[1.]
    \item \textbf{Numbers} are self-evaluating. For example, the numbers 3.14
    and 165 just evaluate to themselves.

    \item \textbf{Names} are looked up in the \texttt{OPERATORS} dictionary.
    Each name (e.g. \texttt{'+'}) is bound to a corresponding function in Python
    that does the appropriate operation on a list of numbers (e.g. \texttt{sum}).

    \item \textbf{Call expressions} are evaluated the same way you've been
    doing them all semester:
        \begin{enumerate}[(1)]
            \item \textbf{Evaluate} the operator, which evaluates to a function.
            \item \textbf{Evaluate} the operands from left to right.
            \item \textbf{Apply} the function to the value of the operands.
        \end{enumerate}
\end{enumerate}

\begin{minipage}{\textwidth}
The function {\tt calc\char`_eval} takes in a Calculator expression represented
in Python and implements each of these rules:
\begin{lstlisting}
def calc_eval(exp):
    """Evaluates a Calculator expression represented as a Pair."""
    if isinstance(exp, Pair): # Call expressions
        fn = calc_eval(exp.first)
        args = list(exp.second.map(calc_eval))
        return calc_apply(fn, args)
    elif exp in OPERATORS:    # Names
        return OPERATORS[exp]
    else:                     # Numbers
        return exp
\end{lstlisting}
\end{minipage}

Note that \texttt{calc\_eval} is recursive! In order to evaluate call
expressions, we must call \texttt{calc\_eval} on the operator and each of the
operands.

The \textit{apply} step in the Calculator language is straight-forward, since we only
have primitive procedures. This step is more complex when it comes to
applying Scheme procedures, which may include user-defined procedures.

Given the Python function that implements the appropriate Calculator operation
and a Python list of numbers, the {\tt calc\char`_apply} function simply calls
the function on the arguments, and regular Python evalutation rules take place.

\begin{lstlisting}
def calc_apply(fn, args):
    """Applies a Calculator operation to a list of numbers."""
    return fn(args)
\end{lstlisting}

%The \texttt{*args} syntax expands a list of arguments. For example:
%\begin{lstlisting}
%>>> calc_apply(print, [1, 2, 3]) # Becomes print(1, 2, 3), not print([1, 2, 3])
%1 2 3
%\end{lstlisting}
%\begin{solution}[0.1in]
%\href{https://youtu.be/pRo4QX4c5lk}{Video walkthrough}
%\end{solution}
