Recall that the reader component of an interpreter parses input strings and
represents them as data structures in the implementing language. In this case,
we need to represent Calculator expressions as Python objects. To represent
numbers, we can just use Python numbers. To represent the names of the
arithmetic procedures, we can use Python strings (e.g. \texttt{'+'}).

Call expressions are a bit more complicated. First, note that like Scheme call
expressions, call expressions in Calculator look just like Scheme lists.  For
example, to construct the expression {\tt (+ 2 3)} in Scheme, we would do the
following:

\begin{lstlisting}
scm> (cons '+ (cons 2 (cons 3 nil)))
(+ 2 3)
\end{lstlisting}

To represent Scheme lists in Python, we will use the {\tt Pair} class.  A {\tt
Pair} instance holds exactly two elements. Accordingly, the {\tt Pair}
constructor takes in two arguments, and to make a list we must nest calls to
the constructor and pass in {\tt nil} as the second element of the last pair.
Note that in our implementation, {\tt nil} is bound to a special user-defined
object that represents an empty list, whereas {\tt nil} in Scheme is actually
an empty list.

\begin{lstlisting}
>>> Pair('+', Pair(2, Pair(3, nil)))
Pair('+', Pair(2, Pair(3, nil)))
\end{lstlisting}

Each {\tt Pair} instance has two instance attributes: {\tt first} and {\tt
second}, which are bound to the first and second elements of the pair
respectively.

\begin{lstlisting}
>>> p = Pair('+', Pair(2, Pair(3, nil)))
>>> p.first
'+'
>>> p.second
Pair(2, Pair(3, nil))
>>> p.second.first
2
\end{lstlisting}

%{\tt Pair} is very similar to {\tt Link}, the class we developed for
%representing linked lists, except that the second attribute doesn't have to be
%a linked list.
%In addition to {\tt Pair} objects, we include a {\tt nil} object
%to represent the empty list. {\tt Pair} instances have methods:
%\begin{enumerate}
%\itemsep-0.5em
%\item[1.] {\tt \char`_\char`_len\char`_\char`_}, which returns the length of the
%    list.
%\item[2.] {\tt \char`_\char`_getitem\char`_\char`_}, which allows indexing into
%    the pair.
%\item[3.] {\tt map}, which applies a function, {\tt fn}, to all of the elements
%    in the list.
%\end{enumerate}
%
%{\tt nil} has the methods {\tt \char`_\char`_len\char`_\char`_}, {\tt
%\char`_\char`_getitem\char`_\char`_}, and {\tt map}.
