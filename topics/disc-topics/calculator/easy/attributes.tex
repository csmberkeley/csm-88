\question Answer the following questions about a {\tt Pair} instance
representing the Calculator expression \texttt{(+ (- 2 4) 6 8)}.

\begin{blocksection}
\begin{subparts}
\subpart Write out the Python expression that returns a {\tt Pair}
representing the given expression, and draw a box and pointer diagram
corresponding to it.

\begin{solution}[0.4in]
\begin{lstlisting}
>>> Pair('+', Pair(Pair('-', Pair(2, Pair(4, nil))), Pair(6, Pair(8, nil))))
\end{lstlisting}
\href{https://goo.gl/2pxLFY}{Box and pointer diagram}
\end{solution}

\subpart What is the operator of the call expression? If the {\tt Pair}
you constructed in the previous part was bound to the name {\tt p}, how would
you retrieve the operator?

\begin{solution}[0.4in]
\texttt{p.first}
\end{solution}

\subpart What are the operands of the call expression? Given that the {\tt Pair}
you constructed in Part (i) was bound to the name {\tt p}, how would
you retrieve a list containing all of the operands? How would you retrieve
only the first operand?

\begin{solution}[0.4in]
\texttt{p.second} to get a list containing all the operands.
\texttt{p.second.first} to get the first operand by itself.
\end{solution}
\end{subparts}
\end{blocksection}
