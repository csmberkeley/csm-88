\question
Before we can create functions and bind symbols to values, we need a way to keep
track of different frames and environments. Fill in the {\tt define} and {\tt
lookup} methods in the {\tt Frame} class. The {\tt define} method should assign
the key {\tt name} to the value {\tt value} in the bindings of the current
frame. The {\tt lookup} method should return the value bound to {\tt name} in
the current frame, or lookup in the parent if there is one. Otherwise, raise a
{\tt NameError}.\\

\begin{lstlisting}
class Frame:
    def __init__(self, parent=None):
        self.bindings = {}
        self.parent = parent

    def define(self, name, value):
\end{lstlisting}
\begin{solution}[0.5in]
\begin{lstlisting}
        self.bindings[name] = value
\end{lstlisting}
\end{solution}
\begin{lstlisting}
    def lookup(self, name): 
\end{lstlisting}
\begin{solution}[1.5in]
\begin{lstlisting}
        if name in self.bindings:
            return self.bindings[name]
        elif self.parent is None:
            raise NameError('name {} is not defined'.format(name))
        else:
            return self.parent.lookup(name)
\end{lstlisting}
\end{solution}
\begin{lstlisting}
global_frame = Frame()
\end{lstlisting}

\clearpage

Note that, to handle environments and the {\tt define} and {\tt lambda} special
forms, we have to modify {\tt calc\_eval} as follows:\\

\begin{lstlisting}
def calc_eval(exp, env):
    """Evaluates a Calculator expression."""
    if isinstance(exp, Pair):
        first, second = exp.first, exp.second
        if first in SPECIAL_FORMS:
            return SPECIAL_FORMS[first](second, env)
        op = calc_eval(first, env)
        args = second.map(lambda exp: calc_eval(exp, env))
        return calc_apply(op, list(args))
    elif exp in OPERATORS:
        return OPERATORS[exp]
    elif isinstance(exp, str):
        return env.lookup(exp)
    else:
        return exp
\end{lstlisting}

{\tt calc\_eval} has to take in an additional parameter {\tt env}, which is the
current frame that {\tt exp} is being evaluated in. If {\tt exp} is a string,
meaning it is a symbol, we simply look it up in {\tt env}.

To handle special forms, we create a dictionary {\tt SPECIAL\_FORMS} that maps
the strings {\tt "define"} and {\tt "lambda"} to the functions {\tt
do\_define\_form} and {\tt do\_lambda\_form}, respectively. These functions take
in the rest of {\tt exp} and perform the order of evaluation specific to their
special form.\\

\begin{lstlisting}
SPECIAL_FORMS = {'define': do_define_form,
                 'lambda': do_lambda_form}
\end{lstlisting}

\question
Let's now implement {\tt do\_define\_form}. This function takes in {\tt exp},
which is the rest of the expression after the {\tt define}), and a frame {\tt
env} and binds the name given by {\tt exp.first} to the value that {\tt
exp.second.first} evaluates to in {\tt env}.\\

\begin{lstlisting}
def do_define_form(exp, env):
    target = exp.first
\end{lstlisting}
\begin{solution}[1in]
\begin{lstlisting}
    value = calc_eval(exp.second.first, env)
    env.define(target, value)
\end{lstlisting}
\end{solution}
\begin{lstlisting}
    return target
\end{lstlisting}

\question
We can now bind symbols to values! But there's more work to be done to allow for
user-defined functions.

We will first implement a class that represents procedures. Instances of the
{\tt LambdaProcedure} class are created with {\tt formals}, a {\tt Pair}
containing the names of the parameters, {\tt body}, a {\tt Pair} representing
the expression that is the body of the procedure, and {\tt env}, the frame where
the procedure was created.

We will restrict ourselves to procedures with only one expression in the body,
similar to how lambda functions are restricted in Python. In the project, you
will have to handle arbitrary procedures.

Implement the {\tt make\_call\_frame} method, which takes in a {\tt Pair} of
arguments {\tt args} and creates and returns a new frame where the formal
parameters of the procedure are bound to the elements of {\tt args}. Make sure
the frame that is created has the correct parent.\\

\begin{lstlisting}
class LambdaProcedure:
    """A procedure defined by a lambda expression."""

    def __init__(self, formals, body, env):
        self.formals = formals
        self.body = body
        self.env = env

    def make_call_frame(self, args):
\end{lstlisting}
\begin{solution}[1.5in]
\begin{lstlisting}
        frame = Frame(self.env)
        for i in range(len(self.formals)):
            frame.define(self.formals[i], args[i])
        return frame
\end{lstlisting}
\end{solution}

\clearpage

Now, it is easy to implement {\tt do\_lambda\_form}:\\

\begin{lstlisting}
def do_lambda_form(exp, env):
    return LambdaProcedure(exp.first, exp.second.first, env)
\end{lstlisting}

Evaluating a {\tt lambda} special form simply creates and evaluates to the
procedure itself. However, there is one more step: currently, our {\tt
calc\_apply} does not know how to apply user-defined functions. Modify it below
so that it can handle instances of the {\tt LambdaProcedure} class:\\

\begin{lstlisting}
def calc_apply(op, args):
    """Applies an operator to a Pair of arguments."""
    if isinstance(op, LambdaProcedure):
\end{lstlisting}
\begin{solution}[1in]
\begin{lstlisting}
        new_env = op.make_call_frame(args)
        return calc_eval(op.body, new_env)
\end{lstlisting}
\end{solution}
\begin{lstlisting}
    else:
        return op(*args)
\end{lstlisting}

We have now added variables and procedures to our Calculator language! With a
few small changes to the parser (to allow for symbols) and the REPL (to handle
environments), we are able to use our interpreter like this:\\

\begin{lstlisting}
calc> (+ 4 5)
9
calc> (define x 4)
x
calc> (+ x 6)
10
calc> ((lambda (x) (* x x)) 7)
49
calc> (define f (lambda (x y) (* (+ x y) (- x y))))
f
calc> (f 5 4)
9
\end{lstlisting}

Download the Calculator interpreter from Lecture 21 and see if you can figure
out the necessary changes to get this to work!
