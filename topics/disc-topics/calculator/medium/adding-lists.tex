\question Now that we can create Scheme-style lists in calculator, let's modify the + operator so
that it can add lists together elementwise. You can assume that the lists are the same
length and contain only numbers.
\begin{lstlisting}
> (+ (quote (7 4 3 9)) (quote 6 2 6 2))
(13 6 9 2)
> (+ (quote (1 2 3 4)) (list (+ 2 2) 3 (/ 4 2) 1))
(5 5 5.0 5)
\end{lstlisting}
\begin{solution}[1.5in]
\begin{lstlisting}
def calc_apply(operator, args):
    ...
    elif operator == '+':
        if type(args[0]) == Pair:
            return reduce(add_pairs, \
                          args[1:], args[0])
        return sum(args)
    ...

def add_pairs(x, y):
    if (x is nil) or (y is nil):
        return nil
    return Pair(x.first + y.first, \
                add_pairs(x.second, y.second))
\end{lstlisting}
\end{solution}
