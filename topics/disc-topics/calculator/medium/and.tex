\question Suppose we want to add handling for comparison operators \texttt{>},
\texttt{<}, and \texttt{=} as well as \texttt{and} expressions to our
Calculator interpreter.  These should work the same way they do in Scheme.

\begin{lstlisting}
calc> (and (= 1 1) 3)
3
calc> (and (+ 1 0) (< 1 0) (/ 1 0))
#f
\end{lstlisting}

\begin{subparts}
\subpart Are we able to handle expressions containing the comparison operators
(such as \texttt{<}, \texttt{>}, or \texttt{=}) with the existing implementation
of \texttt{calc\_eval}? Why or why not?

\begin{solution}[0.3in]
Comparison expressions are regular call expressions, so we need to evaluate the
operator and operands and then apply a function to the arguments.  Therefore,
we do not need to change \texttt{calc\_eval}. We simply need to add new entries
to the \texttt{OPERATORS} dictionary that map \texttt{'<'}, \texttt{'>'}, and
\texttt{'='} to functions that perform the appropriate comparison operation.
\end{solution}

\subpart Are we able to handle \texttt{and} expressions with the existing
implementation of \texttt{calc\_eval}? Why or why not?

\begin{solution}[0.3in]
Since \texttt{and} is a special form that short circuits on the first false-y
operand, we cannot handle these expressions the same way we handle call
expressions.  We need to add special handling for combinations that don't
evaluate all the operands.
\end{solution}

\subpart Now, complete the implementation below to handle {\tt and}
expressions. You may assume the conditional operators (e.g. {\tt <}, {\tt >},
{\tt =}, etc) have already been implemented for you.
\begin{lstlisting}
def calc_eval(exp):
    if isinstance(exp, Pair):
        if _______________________: # and expressions
            return eval_and(exp.second)
        else:                       # Call expressions
            return calc_apply(calc_eval(exp.first), list(exp.second.map(calc_eval)))
    elif exp in OPERATORS:          # Names
        return OPERATORS[exp]
    else:                           # Numbers
        return exp
\end{lstlisting}
\begin{lstlisting}
def eval_and(operands):
\end{lstlisting}
\begin{solution}[0in]
\begin{lstlisting}
def calc_eval(exp):
    if isinstance(exp, Pair):
        if exp.first == 'and':  # and expressions
            return eval_and(exp.second)
        else:                   # Call expressions
            return calc_apply(calc_eval(exp.first), list(exp.second.map(calc_eval)))
    elif exp in OPERATORS:      # Names
        return OPERATORS[exp]
    else:                       # Numbers
        return exp

def eval_and(operands):
    curr, val = operands, True
    while curr is not nil:
        val = calc_eval(curr.first)
        if val is False:
            return False
        curr = curr.second
    return val
\end{lstlisting}
\end{solution}
\end{subparts}
