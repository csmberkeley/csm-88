\question What would Python print?

\begin{lstlisting}
>>> a = lambda: 5
>>> a()
\end{lstlisting}
\begin{solution}[0.2in]
\begin{lstlisting}
5
\end{lstlisting}
\end{solution}

\begin{lstlisting}
>>> a(5)
\end{lstlisting}
\begin{solution}[0.2in]
\begin{lstlisting}
TypeError: <lambda>() takes 0 positional arguments but 1 was given
\end{lstlisting}
\end{solution}

\begin{lstlisting}
>>> b = lambda: lambda x: 3
>>> b()(15)
\end{lstlisting}
\begin{solution}[0.2in]
\begin{lstlisting}
3
\end{lstlisting}
\end{solution}

\begin{lstlisting}
>>> c = lambda x, y: x + y
>>> c(4, 5)
\end{lstlisting}
\begin{solution}[0.2in]
\begin{lstlisting}
9
\end{lstlisting}
\end{solution}

\begin{lstlisting}
>>> d = lambda x: lambda y: x * y
>>> d(3)
\end{lstlisting}
\begin{solution}[0.2in]
\begin{lstlisting}
<function ...>
\end{lstlisting}
\end{solution}

\begin{lstlisting}
>>> d(3)(3)
\end{lstlisting}
\begin{solution}[0.2in]
\begin{lstlisting}
9
\end{lstlisting}
\end{solution}

\begin{lstlisting}
>>> e = d(2)
>>> e(5)
\end{lstlisting}
\begin{solution}[0.2in]
\begin{lstlisting}
10
\end{lstlisting}
\end{solution}

\begin{lstlisting}
>>> f = lambda: print(1)
\end{lstlisting}
\begin{solution}[0.2in]
\begin{lstlisting}
# No output
\end{lstlisting}
\end{solution}

\begin{lstlisting}
>>> g = f()
\end{lstlisting}
\begin{solution}[0.2in]
\begin{lstlisting}
1
\end{lstlisting}
\end{solution}
