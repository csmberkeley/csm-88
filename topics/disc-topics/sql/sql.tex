\question
We will be working with a Facebook-like website called Fakebook. There are four tables in our Fakebook data, described below:\\

\textbf{people(name, age, state, hobby)}: a person on Fakebook \\

\textbf{posts(post\_id, poster, text, time)}: a post with its creator and creation time (in minutes, starting at 0)\\

\textbf{likes(post\_id, name, time)}: a like -- post\_id of the post that was liked, name of person who liked the post, and time (in minutes) of like\\

\textbf{requests(friend1, friend2)}: a friend request from friend1 to friend2\\

If you wish to try these questions out on your sqlite3 interpreter, you can download a .sql file with the data we'll be working with can be found here:\\ https://links.cs61a.org/guer06-data\\

Write a query to find the name and age for each person on Fakebook who is 26 years old or younger

\begin{solution}[0.5in]
SELECT name, age FROM people \\
WHERE age $<$= 26;
\end{solution}

\question
Write a query to find the name of the poster and the time of each post on Fakebook before minute 230

\begin{solution}[0.5in]
SELECT poster, time FROM posts \\
WHERE time $<$ 230;
\end{solution}

\question
Write a query to find the names of users who have liked their own post

\begin{solution}[0.5in]
SELECT poster from posts, likes\\
WHERE name = poster AND posts.post\_id = likes.post\_id;
\end{solution}

\question
The requests table stores all requests from one person to another.  Two people are only friends if both people requested to be friends with the other. Create a table \textbf{friends} that has two columns, \textbf{(friend1, friend2)}, which contain the names of each friend pairing. For example, if Hali sends a friend request to Joe and Joe sends a friend request to Hali, both Joe|Hali and Hali|Joe should appear in the table.

\begin{solution}[0.5in]
CREATE TABLE friends AS\\
    SELECT a.friend1 as friend1, a.friend2 as friend2\\
    FROM requests AS a, requests AS b\\
    WHERE a.friend1 = b.friend2 AND a.friend2 = b.friend1;
\end{solution}

\question
Recall: The aggregate functions MAX, MIN, COUNT, and SUM return the maximum, minimum, number, and sum of the values in a column. The GROUP BY clause of a select statement is used to partition rows into groups. HAVING filters through the groupings from the GROUP BY clause.\\
Write a query that outputs all names of people who have at least 4 friends

\begin{solution}[0.5in]
SELECT friend1 FROM friends \\
GROUP BY friend1 \\
HAVING COUNT(*) $>$= 4;
\end{solution}

\question
Write a query that outputs the states that Will's friends live in, and how many friends are in each state

\begin{solution}[0.5in]
SELECT state, COUNT(*) \\
FROM friends as f, people as p\\
WHERE f.friend1 = "Will" AND f.friend2 = name \\
GROUP BY p.state;\\
\end{solution}

\question
Write a query that outputs the text from every post that was liked within 2 minutes of post time

\begin{solution}[0.5in]
SELECT posts.text FROM posts, likes\\
WHERE posts.post\_id = likes.post\_id AND likes.time $<$= posts.time + 2;
\end{solution}

\question
Write a query that outputs every pair of people that share the same hobby, as well as that shared hobby. Make sure your output doesn't have duplicate pairs (e.g. A|B and B|A should not both appear in the output).

\begin{solution}[0.5in]
SELECT a.name, b.name, a.hobby \\
FROM people AS a, people AS b\\
WHERE a.hobby = b.hobby AND a.name $<$ b.name;
\end{solution}

\question
Write a query that outputs the counts of the number of people that live in each state, with each state listed in descending order of count

\begin{solution}[0.5in]
SELECT state, COUNT(*) FROM people\\
GROUP BY state \\
ORDER BY -COUNT(*);
\end{solution}

\question
Send a friend request by inserting a new friend request from 'Denero' to 'Hilfy'

\begin{solution}[0.5in]
INSERT INTO requests(friend1, friend2) VALUES(‘Denero’, ‘Hilfy’);
\end{solution}

\question
Help Fakebook user 'Denero' send a friend request to every person who liked post 349 by inserting into requests

\begin{solution}[0.5in]
INSERT INTO requests(friend1, friend2) SELECT ‘Denero’, name FROM likes WHERE post\_id = 349;
\end{solution}

\question
Change the hobby of every person whose name is Joe to CS

\begin{solution}[0.5in]
UPDATE people SET hobby = ‘CS’ WHERE name = ‘Joe’;
\end{solution}

\question
Create a table num\_likes with columns: name, post\_id, number. Each row should contain a poster’s name, a post\_id, and number of likes for that post

\begin{solution}[0.5in]
CREATE TABLE num\_likes AS\\
      SELECT posts.poster AS name, posts.post\_id AS post\_id, \\ COUNT(likes.name) AS number \\
      FROM posts, likes\\
	WHERE posts.post\_id = likes.post\_id\\
	GROUP BY posts.post\_id;
\end{solution}

\question
Carolyn is a bit shy. Delete all of her posts in the num\_likes table with fewer than 4 likes

\begin{solution}[0.5in]
DELETE FROM num\_likes WHERE number $<$ 4 AND name = ‘Carolyn’;
\end{solution}

\question

Create an empty table called privacy with columns  name and visibility which should hold the default to everyone.

\begin{solution}[0.5in]
CREATE TABLE privacy(name, visibility DEFAULT ‘everyone’);
\end{solution}

\question
Add Hermish to privacy using the default value. 

\begin{solution}[0.5in]
INSERT INTO privacy(name) VALUES (‘Hermish’);
\end{solution}




