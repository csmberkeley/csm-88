So far, we have joined and manipulated individual rows using \texttt{SELECT}
statements.  But we can also perform aggregation operations over multiple rows
with the same \texttt{SELECT} statements.

We can use the \texttt{MAX}, \texttt{MIN}, \texttt{COUNT}, and \texttt{SUM}
functions to retrieve more information from our initial tables.

If we wanted to find the name and salary of the employee who makes the most
money, we might say

\begin{lstlisting}[language=SQL]
sqlite> SELECT name, MAX(salary) FROM records;
Oliver Warbucks|150000
\end{lstlisting}

\begin{minipage}{\textwidth}
Using the special \texttt{COUNT(*)} syntax, we can count the number of rows in
our table to see the number of employees at the company.

\begin{lstlisting}[language=SQL]
sqlite> SELECT COUNT(*) from RECORDS;
9
\end{lstlisting}
\end{minipage}

These commands can be performed on specific sets of rows in our table by using
the \texttt{GROUP BY [column name]} clause.  This clause takes all of the rows
that have the same value in \texttt{column name} and groups them together.

We can find the miniumum salary earned in each division of the company.

\begin{lstlisting}[language=SQL]
sqlite> SELECT division, MIN(salary) FROM records GROUP BY division;
Computer|25000
Administration|25000
Accounting|18000
\end{lstlisting}

These groupings can be additionally filtered by the \texttt{HAVING} clause.
In contrast to the \texttt{WHERE} clause, which filters out rows, the \texttt{HAVING}
clause filters out entire groups.

To find all titles that are held by more than one person, we say

\begin{lstlisting}[language=SQL]
sqlite> SELECT title FROM records GROUP BY title HAVING count(*) > 1;
Programmer
\end{lstlisting}
