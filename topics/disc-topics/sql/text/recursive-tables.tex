We can create recursive table definitions using the \texttt{with} syntax. 

Let's create a table of natural numbers from 0 to 5 (inclusive). We want to
employ the same thought process as we did with the recursive functions in
Python and Scheme: we want a base case and a recursive case. 

We start by defining a local table called \texttt{num} that has 1 column
\texttt{n}. Each row will have 1 value, a natural number. The base case is 0 
(the smallest natural number) and we can create a one row table for it with
\texttt{select 0}. For the recursive case, for each natural number, we can
add 1 to get another natural number with \texttt{select n + 1 from num}. Since
we want numbers up to 5, the recursive case applies only to numbers less than
5 (so $n + 1$ will be 5 or less). Finally, we combine the two cases into a
single table with \texttt{union}.

\begin{lstlisting}[language=SQL]
sqlite> create table naturals as
   ...>   with num(n) as (
   ...>     select 0 union
   ...>     select n + 1 from num where n < 5
   ...>   )
   ...>   select * from num;
\end{lstlisting}

Remember that \texttt{num} is local to this query. We can see the result by
selecting everything from \texttt{naturals}.
\begin{lstlisting}[language=SQL]
sqlite> select * from naturals;
0
1
2
3
4
5
\end{lstlisting}

