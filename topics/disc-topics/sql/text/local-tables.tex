The \texttt{select} statement can optionally include a \texttt{with} clause that
generates and names \textit{local} tables used in computing the final result.
These local tables cannot be used outside of the select statement, and they can
be thought of as ``helper'' tables.
The full syntax of a \texttt{select} statement has the following form:
\begin{lstlisting}[language=SQL]
with [local-tables] select [columns] from [tables]
  where [condition] order by [criteria]
\end{lstlisting}

For example, you can use a \texttt{with} statement to create and immediately use
a new table to compute the final result.

\begin{lstlisting}[language=SQL]
sqlite> with
   ...>   schedule(day, dresscode) as (
   ...>     select "Monday",    "Sports"    union
   ...>     select "Tuesday",   "Drag"      union
   ...>     select "Wednesday", "Regular"   union
   ...>     select "Thursday",  "Throwback" union
   ...>     select "Friday",    "Casual"
   ...>   )
   ...> select a.name, b.dresscode from 
   ...>    records as a, schedule as b, meetings as c
   ...>    where a.division = c.division and 
   ...>    b.day = c.day order by a.name;
Alyssa P Hacker|Regular
Ben Bitddiddle|Regular
Cy D Fect|Regular
DeWitt Aull|Sports
DeWitt Aull|Throwback
Eben Scrooge|Sports
Lem E Tweakit|Regular
Louis Reasoner|Regular
Oliver Warbucks|Sports
Oliver Warbucks|Throwback
Robert Cratchet|Sports
\end{lstlisting}

If you try selecting from \texttt{schedule} outside, like
\begin{lstlisting}[language=SQL]
sqlite> select * from schedule;
Error: no such table: schedule
\end{lstlisting}
you'll find that you cannot reference it because \texttt{schedule} was not made
globally.
