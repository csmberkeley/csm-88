Suppose we have another table \texttt{meetings} which records the divisional
meetings.

\begin{center}
    \texttt{meetings}\\
    \begin{tabular}{ l l l l l }
        \textbf{Division} & \textbf{Day} & \textbf{Time} \\
        \hline
        Accounting     & Monday    & 9am \\
        Computer       & Wednesday & 4pm \\
        Administration & Monday    & 11am\\
        Administration & Wednesday  & 4pm \\
    \end{tabular}
\end{center}

Data are combined by joining multiple tables together into one, a fundamental
operation in database systems. There are many methods of joining, all closely
related, but we will focus on just one method (the inner join) in this class.

When tables are joined, the resulting table contains a new row for each
combination of rows in the input tables. If two tables are joined and the left
table has $m$ rows and the right table has $n$ rows, then the joined table will
have $mn$ rows. Joins are expressed in SQL by separating table names by commas
in the \texttt{FROM} clause of a \texttt{SELECT} statement.

\begin{lstlisting}[language=SQL]
sqlite> SELECT name, day FROM records, meetings;
Ben Bitdiddle | Monday
Ben Bitdiddle | Wednesday
...
Alyssa P Hacker | Monday
...
\end{lstlisting}

Tables may have overlapping column names, and so we need a method for
disambiguating column names by table. A table may also be joined with itself,
and so we need a method for disambiguating tables. To do so, SQL allows us to
give aliases to tables within a \texttt{FROM} clause using the keyword
\texttt{AS} and to refer to a column within a particular table using a dot
expression. In the example below we find the name and title of Louis Reasoner's
supervisor.

\begin{lstlisting}[language=SQL]
sqlite> SELECT b.name, b.title FROM records AS a, records AS b
   ...>   WHERE a.name = "Louis Reasoner" AND
   ...>         a.supervisor = b.name;
Alyssa P Hacker | Programmer
\end{lstlisting}
\begin{solution}[0.1in]
\href{https://youtu.be/XGgJVmpw-SM}{Video walkthrough}
\end{solution}
