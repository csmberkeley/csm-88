Tables don't need to begin fully formed, it's possible to update them
after creation! We'll also introduce this alternative syntax for creating
a table, which creates an empty table with the given columns:
\begin{lstlisting}[language=SQL]
CREATE TABLE [table]([column1], [column2] DEFAULT [val], ...);
\end{lstlisting}
The \textbf{optional} \texttt{DEFAULT} keyword denotes default values for
a given column if they're not specified. This will be relevant when we
insert new elements into our table.  To add a new table entries, use the
\texttt{INSERT INTO} statement:
\begin{lstlisting}[language=SQL]
INSERT INTO [table] ([column1], [column2], ...)
  VALUES ([value1], [value2], ...), ([value1], [value2], ...);
\end{lstlisting}
A couple of notes:
\begin{itemize}
    \item If a value is specified for each column of the table, you don't
        need to specify column names. This is because each value matches
        up with a column, so there's no ambiguity.
    \item For columns where a value is not specified, the default value
        will be used if available. If not a default value was not provided,
        that column in the new row will be left empty!
\end{itemize}

Here's an example of insertion into an empty table:
\begin{lstlisting}[language=SQL]
sqlite> CREATE TABLE dogs(name, age, phrase DEFAULT "woof");
sqlite> INSERT INTO dogs(name, age) VALUES ("Fido", 1), ("Sparky", 2);
sqlite> INSERT INTO dogs VALUES ("Lassie", 2, "I'll save you!"), ("Floofy", 3);
Error: all VALUES must have the same number of terms
sqlite> INSERT INTO dogs VALUES ("Lassie", 2, "I'll save you!"), ("Floofy", 3, "Much doge");
sqlite> SELECT * FROM dogs;
Fido|1|woof
Sparky|2|woof
Lassie|2|I'll save you!
Floofy|3|Much doge
\end{lstlisting}

The \texttt{INSERT INTO} statement can also insert a table returned by a \texttt{SELECT} 
statement, in which case the syntax is:

\begin{lstlisting}[language=SQL]
INSERT INTO [table] ([column1], [column2], ...)
  SELECT ...;
\end{lstlisting}

We can update certain existing entries in a table using \texttt{UPDATE}:
\begin{lstlisting}[language=SQL]
UPDATE [table] SET [column1] = [value1], [column2] = [value2], ... WHERE [condition];
\end{lstlisting}
All rows matching the condition will have their columns updated. If no
condition is specified, \textbf{all} rows will be updated! We can also
remove certain entries in a table using \texttt{DELETE}:
\begin{lstlisting}[language=SQL]
DELETE FROM [table] WHERE [condition];
\end{lstlisting}
Just like with \texttt{UPDATE}, if no condition is specified,
\textbf{all} rows will be deleted!  Here's an example using all of the
above:
\begin{lstlisting}[language=SQL]
sqlite> UPDATE dogs SET age=age+1; -- If condition isn't specified, every row is updated
sqlite> SELECT * FROM dogs;
Fido|2|woof
Sparky|3|woof
Lassie|3|I'll save you!
Floofy|4|Much doge
\end{lstlisting}
\begin{lstlisting}[language=SQL]
sqlite> UPDATE dogs SET phrase = "bark" WHERE age=2;
sqlite> SELECT * FROM dogs WHERE age=2;
Fido|2|bark
sqlite> DELETE FROM dogs WHERE age=3;
sqlite> SELECT * FROM dogs;
Fido|2|bark
Floofy|4|Much doge
\end{lstlisting}
Finally, we can delete an entire table using the \texttt{DROP TABLE
[table]} statement. In this example, the \texttt{.schema} statement shows
us a list of the current tables, along with their column names.
\begin{lstlisting}[language=SQL]
sqlite> .schema
CREATE TABLE dogs(name, age, phrase DEFAULT "woof");
sqlite> DROP TABLE dogs;
sqlite> .schema
sqlite> -- Nothing displayed above
\end{lstlisting}
