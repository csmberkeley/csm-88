An assignment statement assigns a certain value to a variable name.

\begin{blocksection}
\begin{equation*}
    \underbrace{\texttt{x}}_\text{Name}
    \texttt{ = }
    \underbrace{\texttt{2 + 3}}_\text{Expression}
\end{equation*}
\end{blocksection}

\begin{blocksection}
To execute an assignment statement:
\begin{enumerate}
\item Evaluate the expression on the right-hand-side of the statement to obtain
a value.
\item Bind the value to the name on the left-hand-side of the statement.
\end{enumerate}
\end{blocksection}

Let's try to assign the primitive value 6 to the name \texttt{a}, and subsequently
do a lookup on \texttt{a}.

\begin{lstlisting}
>>> a = 6
>>> a
6
\end{lstlisting}

Now, let's reassign \texttt{a} to another value. This time, let's use a more
complex expression. Note that the name is bound to the value, not the
expression!

\begin{lstlisting}
>>> a = (3 + 5) // 2
>>> a
4
\end{lstlisting}
