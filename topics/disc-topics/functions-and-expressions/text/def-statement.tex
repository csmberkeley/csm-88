The \texttt{def} statement defines functions.

\begin{lstlisting}
def square(x):
    return x * x
\end{lstlisting}

When a {\tt def} statement is executed, Python creates a binding from the
name (e.g.\ \texttt{square}) to a function. The names in parentheses are the
function's \define{parameters} (in this case, \texttt{x} is the only parameter).
When the function is called, the \define{body} of the function is executed (in this
case, \texttt{return x * x}).
