A \define{call expression} applies a function to 0 or more arguments and
evaluates to the function's return value.

\begin{blocksection}
\begin{equation*}
    \underbrace{\texttt{add}}_\text{Operator}
    \texttt{ ( }
    \underbrace{\texttt{2}}_\text{Operand 0}
    \texttt{ , }
    \underbrace{\texttt{3}}_\text{Operand 1}
    \texttt{ )}
\end{equation*}
\end{blocksection}

\begin{blocksection}
To evaluate a function call:
\begin{enumerate}
\item Evaluate the operator, which should evaluate to a function.
\item Evaluate the operands from left to right to get the values of the
arguments.
\item Apply the function (the value of the operator) to the arguments (the
values of the operands) to obtain the return value.
\end{enumerate}
\end{blocksection}

If the operator or an operand is itself a call expression, then these steps are
applied in order to evaluate it.
