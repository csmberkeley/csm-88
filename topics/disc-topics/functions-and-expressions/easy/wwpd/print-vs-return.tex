% Question Objectives
% - The `return` statement will cause a function call to evaluate to a value,
%   whereas `print` just displays a value.
% - The `print` function evaluate to `None`.
% - `return` only seems to print a value, since the interpreter will
%   automatically display the value an expression evaluates to if the value is
%   not assigned to a variable.

\begin{blocksection}
\question What would Python display for the following?
\begin{lstlisting}
>>> def om(cookie):
...     return cookie
>>> def nom(cookie):
...     print(cookie)
\end{lstlisting}
\end{blocksection}
\begin{blocksection}
\begin{lstlisting}
>>> om(4)
\end{lstlisting}
\begin{solution}[0.3in]
4
\end{solution}
\begin{lstlisting}
>>> nom(4)
\end{lstlisting}
\begin{solution}[0.3in]
4
\end{solution}
\end{blocksection}
\begin{blocksection}
\begin{lstlisting}
>>> joyce = om(-4)
\end{lstlisting}
\begin{solution}[0.3in]
(\emph{nothing})

Nothing is displayed. This is because the interpreter in an interactive session
will print the value of an expression to the terminal by default if the value
is not assigned to a variable (unless the value is \texttt{None}, in which case
nothing is displayed). However, if a value is assigned to the variable, then
the value is suppressed.
\end{solution}
\begin{lstlisting}
>>> joyce + 1
\end{lstlisting}
\begin{solution}[0.3in]
-3

The previous call to \texttt{om(-4)} returns \texttt{-4} which is then bound to
\texttt{joyce}. So, \texttt{joyce + 1} is \texttt{-3}.
\end{solution}
\end{blocksection}
\begin{blocksection}
\begin{lstlisting}
>>> michelle = nom(4)
\end{lstlisting}
\begin{solution}[0.3in]
4
\newline
\newline
Since we're printing the value of \texttt{cookie} in \texttt{nom}, then the
value of \texttt{cookie} will always be displayed to the console.
\end{solution}
\begin{lstlisting}
>>> michelle + 1
\end{lstlisting}
\begin{solution}[1in]
\texttt{TypeError: unsupported operand type(s) for +:\\
'NoneType' and 'int'}
\newline
\newline
Note that although \texttt{nom} does not have a return statement, even if we
returned the value of \texttt{print(cookie)}, the \texttt{print} function would
return \texttt{None} anyways.
\end{solution}
\end{blocksection}

