\question Consider the function below.

\begin{lstlisting}
def foo(x, y):
    y = y + 4
    x = y / 2
    return x * y
\end{lstlisting}

How many arguments does this function take in, and what type should they be?

\begin{solution}[.3in]
The \texttt{foo} function takes in two arguments; the second has to be a
number, and though it seems like the first should also be a number, it can be
anything, because the first parameter gets reassigned in the second line!
\end{solution}

Consider the call expression \texttt{foo(5, 1 + 3)}. What are \texttt{x} and
\texttt{y} bound to inside the body of the function during this call?

\begin{solution}[.3in]
The parameter \texttt{x} is bound to \texttt{5} and the parameter \texttt{y} is
bound to \texttt{4} (the result of \texttt{1 + 3}).
\end{solution}

What does the call expression \texttt{foo(5, 4)} return?

\begin{solution}[.3in]
\texttt{32}. First, \texttt{y} is reassigned to \texttt{8}. Then, \texttt{x} is
reassigned to \texttt{4}. Finally, the result of \texttt{4 * 8} is returned.
\end{solution}

What about \texttt{foo(10, 4)}?

\begin{solution}[.5in]
The output would still be \texttt{32}. Notice that the first argument will
never affect our result because \texttt{x} gets reassigned in the second line.
\end{solution}

