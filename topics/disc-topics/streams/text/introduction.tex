A \define{stream} is a lazily-evaluated linked list. A stream's elements
(except for the first element) are only computed when those values are needed.

A \lstinline$Stream$ instance is similar to a \lstinline$Link$ instance.
Both have \lstinline$first$ and \lstinline$rest$ attributes.
The rest of a \lstinline$Link$ is either a \lstinline$Link$ or
\lstinline$Link.empty$.
Likewise, the rest of a \lstinline$Stream$ is either a \lstinline$Stream$ or
\lstinline$Stream.empty$.

However, instead of specifying all of the elements in \lstinline$__init__$, we
provide a function, \lstinline$compute_rest$, which will be called to compute
the next element of the stream.
Remember that the code in the function body is not evaluated until it is
called, which lets us implement the desired evaluation behavior.
It's very important that \lstinline$compute_rest$ should return a
\lstinline$Stream$, if you don't want your \lstinline$Stream$ to end.

\begin{lstlisting}
class Stream:
    empty = 'empty'

    def __init__(self, first, compute_rest=lambda: Stream.empty):
        self.first = first
        self.cached_rest = None
        assert callable(compute_rest)
        self.compute_rest = compute_rest

    @property
    def rest(self):
        """Return the rest, computing it if necessary."""
        if self.compute_rest is not None:
            self.cached_rest = self.compute_rest()
            self.compute_rest = None
        return self.cached_rest

    def __repr__(self):
        rest = self.cached_rest if self.compute_rest is None else '<...>'
        return 'Stream({}, {})'.format(self.first, rest)
\end{lstlisting}
