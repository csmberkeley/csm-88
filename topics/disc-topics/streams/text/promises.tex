Under the hood, streams are implemented using \textit{promises}, which represent
a value that is evaluated lazily. We can work with promises directly using the
\texttt{delay} and \texttt{force} primitives. The \texttt{delay} special form
takes an unevaluated expression and packages it up into a \texttt{promise}. The
\texttt{force} function takes a promise, evaluates it, and returns its value.
Subsequent calls to \texttt{force} do not re-evaluate the promise; instead, we
remember the previous value and return that instead of recomputing the value.

\begin{lstlisting}[language=Scheme]
scm> (define p
...>   (delay
...>     (begin
...>       (print "evaluating!")
...>       (+ 3 4)))
p
scm> p
#[promise (not forced)]
scm> (force p)
evaluating!
7
scm> p
#[promise (forced)]
scm> (force p)
7
\end{lstlisting}

When we construct a stream with \texttt{cons-stream}, we get a Scheme pair of
the first element of the stream, as well as a promise that contains an
expression to compute the rest of the stream. That is, if \texttt{s} is a
stream, then \texttt{(car s)} returns an element, and \texttt{(cdr s)} returns
a promise. \texttt{(cdr-stream s)} is shorthand for \texttt{(force (cdr s))},
so it will evaluate the promise to get the rest of the stream.

\begin{lstlisting}[language=Scheme]
scm> (define s (cons-stream 1 nil))
s
scm> s
(1 . #[promise (not forced)])
scm> (cdr s)
#[promise (not forced)]
scm> (cdr-stream s)
nil
scm> s
(1 . #[promise (forced)])
\end{lstlisting}

Here's an example using the \texttt{naturals} function we defined earlier:

\begin{lstlisting}[language=Scheme]
scm> (define nat (naturals 0))
nat
scm> nat
(0 . #[promise (not forced)])
scm> (cdr-stream nat)
(1 . #[promise (not forced)])  ; the value 1 has been forced
scm> nat
(0 . #[promise (forced)])
\end{lstlisting}
