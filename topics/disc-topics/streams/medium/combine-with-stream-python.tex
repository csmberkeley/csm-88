\question This is the function \lstinline$combine_stream$.
Use it to define the infinite stream of factorials below!
You can assume \lstinline$add$ and \lstinline$mul$ have been imported, and
you may also use the infinite stream of naturals from page 4.

\begin{lstlisting}
def combine_stream(f, s1, s2):
    if s1 is Stream.empty or s2 is Stream.empty:
        return Stream.empty
    return Stream(f(s1.first, s2.first), lambda: combine_stream(f, s1.rest, s2.rest))
\end{lstlisting}

\begin{lstlisting}
def evens():
    return combine_stream(add, naturals(0), naturals(0))
\end{lstlisting}

\begin{lstlisting}
def factorials():
\end{lstlisting}
\begin{solution}[2cm]
\begin{lstlisting}
    return Stream(1, lambda: combine_stream(mul, naturals(1), factorials()))
\end{lstlisting}
\end{solution}

Now define a new \lstinline$Stream$, where the $n$th term represents the
degree-$n$ polynomial expansion for $e^x$, which is $\sum_{i=0}^{n} x^i / i!$.
You are allowed to use any of the other functions defined in this problem.

\begin{lstlisting}
def exp(x):
\end{lstlisting}
\begin{solution}[2cm]
\begin{lstlisting}
    terms = combine_stream(lambda a, b: (x**a)/b, naturals(0), factorials())
    return Stream(0, lambda: combine_stream(add, terms, exp(x)))
\end{lstlisting}
\end{solution}
