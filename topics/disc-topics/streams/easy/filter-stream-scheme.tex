\question Using streams can be tricky! Compare the following two
implementations of \texttt{filter-stream}, the first is a correct
implementation whereas the second is wrong in some way. What's wrong with the
second implementation?

\begin{lstlisting}[language=Scheme]
; Correct
(define (filter-stream f s)
  (cond
    ((null? s) nil)
    ((f (car s)) (cons-stream (car s) (filter-stream f (cdr-stream s))))
    (else (filter-stream f (cdr-stream s)))))

; Incorrect
(define (filter-stream f s)
  (if (null? s) nil
    (let ((rest (filter-stream f (cdr-stream s))))
      (if (f (car s))
        (cons-stream (car s) rest)
        rest))))
\end{lstlisting}
\begin{solution}[0.0in]
Evaluating \texttt{rest} will result in infinite recursion if \texttt{s} is an
infinite stream! In the body of \texttt{filter-stream}, \texttt{rest} is always
computed before \texttt{cons-stream} can delay the evaluation.

Another way of thinking about this is that everything in the body of the
\texttt{let} doesn't matter. All we will be doing is repeatedly doing the
recursive call on \texttt{filter-stream}.
\href{https://youtu.be/TGSYZvoIMnE?t=3m39s}{Video walkthrough}
\end{solution}
