\question Write a function \texttt{map-stream}, which takes a function
\texttt{f} and a stream \texttt{s}. It returns a new stream which has all the
elements from \texttt{s}, but with \texttt{f} applied to each one.
\begin{lstlisting}[language=Scheme]
(define (map-stream f s)
\end{lstlisting}
\begin{solution}[1in]
\begin{lstlisting}[language=Scheme]
  (if (null? s)
    nil
    (cons-stream (f (car s)) (map-stream f (cdr-stream s))))
\end{lstlisting}

It might help to also compare this to the version of \texttt{map} for regular
(non-stream) Scheme lists:
\begin{lstlisting}[language=Scheme]
(define (map f s)
  (if (null? s)
    nil
    (cons (f (car s)) (map f (cdr s)))))
\end{lstlisting}

Not too different, eh? The main change we've made is indicating we want to
lazily evaluate the rest of our stream by using \texttt{cons-stream} instead of
\texttt{cons}, and recognizing is a stream rather than a regular list by using
\texttt{cdr-stream}.
\end{solution}
\begin{lstlisting}[language=Scheme]
scm> (define evens (map-stream (lambda (x) (* x 2)) nat))
evens
scm> (car (cdr-stream evens))
2
\end{lstlisting}
