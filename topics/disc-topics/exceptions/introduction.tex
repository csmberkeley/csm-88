\textbf{Exceptions} are used to signify when something goes wrong
in your program. For interpreters, they're often used to categorize a case when
the user inputs something that doesn't make sense (just try typing in {\tt
Hi Soumya} in your Python interpreter and see what happens!)

There are two major things that you do with exceptions: {\tt raise}
and \textbf{handle} them.

Generally, to raise an exception you use the statement {\tt raise
<expression>}.

To handle an exception, you use a {\tt try-except} block. The syntax is as
follows:

\begin{lstlisting}
try:
    <try suite>
except <exception class> as <name>:
    <except suite>
...
\end{lstlisting}

You can have multiple {\tt except} suites for different types of exceptions that
might occur in the try suite.


