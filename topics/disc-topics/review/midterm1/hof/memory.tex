\question (Spring 2015) Implement the {\tt memory} function, which takes a
number {\tt x} and a single-argument function {\tt f}. It returns a function
with a peculiar behavior that you must discover from the doctests. You may only
use names and call expressions in your solution. You may not write numbers or
use features of Python not yet covered in the course.
\begin{lstlisting}
square = lambda x: x * x
double = lambda x: 2 * x
def memory(x, f):
    """Return a higher-order function that prints its
    memories.
    >>> f = memory(3, lambda x: x)
    >>> f = f(square)
    3
    >>> f = f(double)
    9
    >>> f = f(print)
    6
    >>> f = f(square)
    3
    None
    """
    def g(h):

        print(________________________________________)

        return _______________________________________

    return g
\end{lstlisting}
\begin{solution}
\begin{lstlisting}
def memory(x, f):
    def g(h):
        print(f(x))
        return memory(x, h)
    return g
\end{lstlisting}
\end{solution}
