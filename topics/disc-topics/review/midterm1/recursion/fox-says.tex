\question (Fall 2013) The CS61A staff has developed a formula for determining
what a fox might say. Given three strings, a start, a middle, and an end, a fox
will say the start string, followed by the middle string repeated a number of
times, followed by the end string. These parts are all separated by single
hyphens.

Complete the definition of {\tt fox\_says}, which takes the three string parts
of the fox's statement (start, middle, and end) and a positive integer {\tt num}
indicating how many times to repeat middle. It returns a string. You cannot use
any for or while statements. Use recursion in repeat. Moreover, you cannot use
string operations other than the + operator to concatenate strings together.
\begin{lstlisting}
def fox_says(start, middle, end, num ):
    """
    >>> fox_says('wa', 'pa', 'pow', 3)
    'wa-pa-pa-pa-pow'
    >>> fox_says('fraka', 'kaka', 'kow', 4)
    'fraka-kaka-kaka-kaka-kaka-kow'
    """
    def repeat(k):
\end{lstlisting}
\begin{solution}[1.75in]
\begin{lstlisting}
    if k == 1:
        return middle
    else:
        return middle + '-' + repeat(k-1)
\end{lstlisting}
\end{solution}
\begin{lstlisting}
    return start + '-' + repeat(num) + '-' + end
\end{lstlisting}
