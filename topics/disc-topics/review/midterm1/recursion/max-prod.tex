\question
Write a function that takes in a list and returns the maximum product that
can be formed using nonconsecutive elements of the list. The input list
will contain only numbers greater than or equal to 1.

\begin{lstlisting}
def max_product(lst):
    """Return the maximum product that can be formed using lst
    without using any consecutive numbers
    >>> max_product([10,3,1,9,2]) # 10 * 9
    90
    >>> max_product([5,10,5,10,5]) # 5 * 5 * 5
    125
    >>> max_product([])
    1
    """
\end{lstlisting}
\begin{solution}[2in]
\begin{lstlisting}
    if lst == []:
        return 1
    elif len(lst) == 1: # Base case optional
        return lst[0]
    else:
        return max(max_product(lst[1:]), lst[0]*max_product(lst[2:]))
\end{lstlisting}
At each step, we choose if we want to include the current number in our product
or not:
\begin{itemize}
    \item If we include the current number, we cannot use the adjacent number.
    \item If we don't use the current number, we try the adjacent number (and
        obviously ignore the current number).
\end{itemize}
The recursive calls represent these two alternate realities. Finally, we pick
the one that gives us the largest product.\\
\href{https://www.youtube.com/watch?v=Am6m8YgAnYY&list=PLx38hZJ5RLZdJgRCgpaTbmRXKAHOUmomO&index=4&t=3m0s}{Video walkthrough}
\end{solution}
