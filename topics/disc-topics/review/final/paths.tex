\question (Fall 2013) Fill in the blanks in the implementation of
\texttt{paths}, which takes as input two positive integers \texttt{x} and
\texttt{y}. It returns the number of ways of reaching y from x by repeatedly
incrementing or doubling. For instance, we can reach 9 from 3 by incrementing to
4, doubling to 8, then incrementing again to 9.

\medskip

\begin{lstlisting}
def inc(x):
    return x + 1

def double(x):
    return x * 2

def paths(x, y):
    """Return the number of ways to reach y from x by repeated
    incrementing or doubling.
    >>> paths(3, 5) # inc(inc(3))
    1
    >>> paths(3, 6) # double(3), inc(inc(inc(3)))
    2
    >>> paths(3, 9) # E.g. inc(double(inc(3)))
    3
    >>> paths(3, 3) # No calls is a valid path
    1
    """
    if x > y:

        return ______________________________________________

    elif x == y:

        return ______________________________________________

    else:

        return ______________________________________________
\end{lstlisting}
\begin{solution}
\begin{lstlisting}
def paths(x, y):
    if x > y:
        return 0
    elif x == y:
        return 1
    else:
        return paths(inc(x), y) + paths(double(x), y)
\end{lstlisting}
\end{solution}

\clearpage