\question Complete \texttt{redundant\_map}, which takes a tree \texttt{t} and a function \texttt{f}, and applies \texttt{f} to each node ($2^d$) times, where d is the depth of the node. The root has a depth of 0. We should be returning a \textbf{new tree}.

\begin{lstlisting}
def redundant_map(t, f):
  """
  >>> double = lambda x: x*2
  >>> t = tree(1, [tree(1), tree(2, [tree(1, [tree(1)])])])
  >>> print_tree(t)
  1
    1
    2
      1
        1
  >>> new_t = redundant_map(t, double)
  >>> print_tree(new_t)
  2
    4
    8
      16
        256
  """
  new_label = ____________________________________________
  new_f = _________________________________________________
  ____________ = ___________________________________________
  return _____________________________
\end{lstlisting}
\begin{solution}
\begin{lstlisting}
def redundant_map(t, f):
  """
  >>> double = lambda x: x*2
  >>> t = tree(1, [tree(1), tree(2, [tree(1, [tree(1)])])])
  >>> print_tree(t)
  1
    1
    2
      1
        1
  >>> new_t = redundant_map(t, double)
  >>> print_tree(new_t)
  2
    4
    8
      16
        256
  """
  new_label = f(label(t))
  new_f = lambda x: f(f(x))
  new_branches = [redundant_map(b, new_f) for b in branches(t)]
  return tree(new_label, new_branches)
\end{lstlisting}
Every time we recurse, we transform our map function into one that is one level
deeper in terms of calls to input function \texttt{f}. To see why this will
achieve the result we want, let's look at what happens to some input function
\texttt{f}.

\begin{itemize}
    \item The first call to \texttt{redundant\char`_map} will call \texttt{f}
        once.
    \item This means on the second call to \texttt{redundant\char`_map}, we pass
        in a function \texttt{g} that causes the original \texttt{f} to be
        called two times.
    \item On the third call to \texttt{redundant\char`_map}, we pass in a
        function \texttt{h} that causes \texttt{g} to be called two times.
        Remember that \texttt{g} calls original \texttt{f} twice, so \texttt{h}
        will end up calling original \texttt{f} four times.
\end{itemize}
Therefore, each level will have double the calls to \texttt{f} as the previous
level, which matches the requirements.
\end{solution}
