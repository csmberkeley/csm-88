\question (Fall 2013) Fill in the blanks in the implementation of
\texttt{pathfinder}, a higher-order function that takes an increasing function
\texttt{f} and a positive integer \texttt{y}. It returns a function that takes a
positive integer \texttt{x} and returns whether it is possible to reach
\texttt{y} by applying \texttt{f} to \texttt{x} zero or more times. For example,
8 can be reached from 2 by applying \texttt{double} twice. A function \texttt{f}
is \textit{increasing} if $f(x)>x$ for all positive integers \texttt{x}.

\medskip

\begin{lstlisting}
def pathfinder(f, y):
    """
    >>> f = pathfinder(double, 8)
    >>> {k: f(k) for k in (1, 2, 3, 4, 5)}
    {1: True, 2: True, 3: False, 4: True, 5: False}
    >>> g = pathfinder(inc, 3)
    >>> {k: g(k) for k in (1, 2, 3, 4, 5)}
    {1: True, 2: True, 3: True, 4: False, 5: False}
    """
    def find_from(x):

        while ______________________________________________:

            _________________________________________________

        return ______________________________________________

    _________________________________________________________
\end{lstlisting}
\begin{solution}
\begin{lstlisting}
def pathfinder(f, y):
    def find_from(x):
        while x < y:
            x = f(x)
        return x == y
    return find_from
\end{lstlisting}
\end{solution}
