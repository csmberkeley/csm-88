\question (Adapted from Fall 2013) Fill in the blanks in the implementation of
\texttt{paths}, which takes as input two positive integers \texttt{x} and
\texttt{y}. It returns a list of paths, where each path is a list containing
steps to reach \texttt{y} from \texttt{x} by repeated incrementing or doubling
For instance, we can reach 9 from 3 by incrementing to 4, doubling to 8, 
then incrementing again to 9, so one path is \texttt{[3, 4, 8, 9]}

\medskip

\begin{lstlisting}
def paths(x, y):
    """Return a list of ways to reach y from x by repeated
    incrementing or doubling.
    >>> paths(3, 5) 
    [[3, 4, 5]]
    >>> sorted(paths(3, 6))
    [[3, 4, 5, 6], [3, 6]]
    >>> sorted(paths(3, 9))
    [[3, 4, 5, 6, 7, 8, 9], [3, 4, 8, 9], [3, 6, 7, 8, 9]]
    >>> paths(3, 3) # No calls is a valid path
    [[3]]
    """
    if _________________________:

        return ______________________________________________

    elif _______________________:

        return ______________________________________________

    else:

        a = _________________________________________________

        b = _________________________________________________

        return ______________________________________________
\end{lstlisting}
\begin{solution}
\begin{lstlisting}
def paths(x, y):
    if x > y:
        return []
    elif x == y:
        return [[x]]
    else:
        a = paths(x + 1, y)
        b = paths(x * 2, y)
        return [[x] + subpath for subpath in a + b]
\end{lstlisting}
\end{solution}

\clearpage