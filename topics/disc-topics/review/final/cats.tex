\question You're starting a new job at an animal shelter, and you've been
tasked with keeping track of all the cats that are up for adoption!

We'll start with an empty table:
\begin{lstlisting}[language=SQL]
CREATE TABLE cats(name, weight DEFAULT 1, notes DEFAULT "meow");
\end{lstlisting}

\begin{parts}
\part What would SQL display?
\begin{lstlisting}[language=SQL]
sqlite> INSERT INTO cats(name) VALUES ("Tom"), ("Whiskers");
sqlite> SELECT * FROM cats;
\end{lstlisting}
\begin{solution}[0.50in]
\begin{lstlisting}[language=SQL]
Tom|1|meow
Whiskers|1|meow
\end{lstlisting}
\end{solution}
\begin{lstlisting}[language=SQL]
sqlite> INSERT INTO cats VALUES
   ...>   ("Mittens", 2, "Actually likes shoes"),
   ...>   ("Rascal", 4, "Prefers to associate with dogs"),
   ...>   ("Magic", 2, "Expert at card games");
sqlite> SELECT * FROM cats ORDER BY weight, name;
\end{lstlisting}
\begin{solution}[0.75in]
\begin{lstlisting}[language=SQL]
Tom|1|meow
Whiskers|1|meow
Magic|2|Expert at card games
Mittens|2|Actually likes shoes
Rascal|4|Prefers to associate with dogs
\end{lstlisting}
\end{solution}
\begin{lstlisting}[language=SQL]
sqlite> UPDATE cats SET notes = "A cat" WHERE notes = "meow";
sqlite> SELECT name FROM cats WHERE notes = "A cat";
\end{lstlisting}
\begin{solution}[0.50in]
\begin{lstlisting}[language=SQL]
Tom
Whiskers
\end{lstlisting}
\end{solution}

\part Cats of different weights require different quantities of food. We
have the following table:
\begin{lstlisting}[language=SQL]
CREATE TABLE food AS
    SELECT 1 AS cat_weight, 0.5 AS amount UNION
    SELECT 2              , 2.5           UNION
    SELECT 3              , 4.0           UNION
    SELECT 4              , 4.5;
\end{lstlisting}

Write a query that calculates the total amount of food required to feed
all the cats (this should work for any table of cats, not just the one we
created above). In our example, we have two cats of weight 1, two cats of
weight 2, and one cat of weight 4. The total food required is $2 \times 0.5 +
2 \times 2.5 + 1 \times 4.5 = 10.5$.
\begin{lstlisting}[language=SQL]
  SELECT _________________________________________________________________________

    FROM _________________________________________________________________________

    WHERE ________________________________________________________________________;
\end{lstlisting}
\begin{solution}
Specifying the table name in the \texttt{WHERE} clause here is not
necessary and was added just for clarity.
\begin{lstlisting}[language=SQL]
SELECT SUM(amount) FROM cats, food WHERE cats.weight = food.cat_weight;
\end{lstlisting}
\end{solution}
\end{parts}
