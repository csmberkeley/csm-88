\question Let's answer the age-old question of "What came first, the chicken or
the egg?"  Each function takes an argument (pun unintended!), which is a boolean
value that represents an argument for whether or not the corresponding element
came first.  To mirror the debate, let's do this:
\begin{lstlisting}
def chicken(argument=True):
    print("Chickens!")
    return egg(argument)
def egg(argument=True):
    print("Eggs!")
    return chicken(argument)
\end{lstlisting}
We could argue that chickens came first by calling {\tt chicken()}, but then the
argument would never be resolved. That's unfortunate! Therefore, let's also add
{\tt generations}, which is a variable that represents which generation we are
on. For example, if you argue for the 5th generation chicken, you're arguing for
not the 4th generation egg.  The 0th generation determines the winner.

\begin{lstlisting}
def chicken(argument, generations=0):
    """ Argue whether or not the chicken came first.
    >>> chicken(True)
    [Insert reason chickens came first.]
    >>> chicken(False) # is equivalent to egg(True)
    [Insert reason eggs came first.]
    >>> chicken(False, 1) # is equivalent to egg(True, 0)
    [Insert reason eggs came first.]
    """
\end{lstlisting}
\begin{solution}[1.2in]
\begin{lstlisting}
    if generations <= 0:
        if argument:
            return "Chickens lay eggs!"
        return egg(not argument, generations)
    return egg(not argument, generations - 1)
\end{lstlisting}
\end{solution}
\begin{lstlisting}
def egg(argument, generations=0):
    "Argue whether or not the egg came first."""
\end{lstlisting}
\begin{solution}[1.2in]
\begin{lstlisting}
    if generations <= 0:
        if argument:
            return "Eggs hatch chickens! And they're cuter."
        return chicken(not argument, generations)
    return egg(not argument, generations - 1)
\end{lstlisting}
\end{solution}
