\question Write a procedure {\tt merge(s1, s2)} which takes two sorted (smallest
value first) lists and returns a single list with all of the elements of the two
lists, in ascending order. Use recursion.

\textbf{Hint}: If you can figure out which list has the smallest element out of
both, then we know that the resulting merged list will have that smallest
element, followed by the merge of the two lists with the smallest item
removed. Don't forget to handle the case where one list is empty!

\begin{lstlisting}
def merge(s1, s2):
    """ Merges two sorted lists
    >>> merge([1, 3], [2, 4])
    [1, 2, 3, 4]
    >>> merge([1, 2], [])
    [1, 2]
    """
    if ______________________________________________________________________________:


        return s2


    elif ____________________________________________________________________________:


        return s1


    elif ____________________________________________________________________________:


        return ______________________________________________________________________


    else:


        return ______________________________________________________________________
\end{lstlisting}
\begin{solution}[1.5in]
\begin{lstlisting}
    if len(s1) == 0:
        return s2
    elif len(s2) == 0:
        return s1
    elif s1[0] < s2[0]:
        return [s1[0]] + merge(s1[1:], s2)
    else:
        return [s2[0]] + merge(s1, s2[1:])
\end{lstlisting}
\end{solution}
