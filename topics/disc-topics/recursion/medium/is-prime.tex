\question Below is the iterative version of {\tt is\_prime}, which returns 
{\tt True} if positive integer {\tt n} is a prime number and {\tt False} otherwise: 
\begin{lstlisting}
def is_prime(n):
    if n == 1: 
        return False
    k = 2
    while k < n:
        if n % k == 0: 
            return False
        k += 1 
    return True
\end{lstlisting}

Implement the recursive {\tt is\_prime} function. Do not use a while loop, use recursion.

\begin{lstlisting}
def is_prime(n):
\end{lstlisting}

\begin{solution}[2in]
\begin{lstlisting}
    def helper(k):
        if k >= n:
            return True
        if n % k == 0:
            return False
        return helper(k + 1)
    return helper(2)
\end{lstlisting}
\textbf{Note}: The goal of this question was to demonstrate how to use a helper 
function, as well as how to translate between iteration and recursion.
\end{solution}

\iffalse
\question Is there a math trick you can apply to {\tt is\_prime} so it runs faster?
\begin{solution}[2.0in]
If {\tt n} is prime, it can be factored into at least two factors. If both of the factors 
are greater than the $\sqrt{n}$, then their product is greater than {\tt n}. Therefore, 
one of the factors must be less than or equal to $\sqrt{n}$.
\begin{lstlisting}
    def helper(k):
        if k > math.sqrt(n):
            return True
        if n % k == 0:
            return False
        return helper(k + 1)
    return helper(2)
\end{lstlisting}
\textbf{Note}: Don't forget to import {\tt math}.
\end{solution}
\fi
