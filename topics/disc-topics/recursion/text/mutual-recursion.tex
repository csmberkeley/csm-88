Sometimes it isn't enough to have one function call itself; sometimes functions
recursively call one another!  Here is an example:
\begin{lstlisting}
def even(n):
    if n == 0:
        return True
    return odd(n - 1)

def odd(n):
    if n == 0:
        return False
    return even(n - 1)
\end{lstlisting}
Given a positive integer, the function {\tt even} will return a boolean value
representing whether or not the integer is even.  However, notice that the
recursive call is to {\tt odd}, not itself.  We call this mutual recursion
because because {\tt even} and {\tt odd} are defined in terms of one another,
and as a result alternatively call one another to arrive at the answer.

Mutual recursion is especially useful for when you need to deal with different
data structures that interact with one another.  We'll see this later in the
semester with Trees.
