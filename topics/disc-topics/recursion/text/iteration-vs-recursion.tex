We've written {\tt factorial} recursively. Let's compare the two implementations:

\begin{minipage}{0.5\linewidth}
\begin{lstlisting}[linewidth=\linewidth]
def recursive_fact(n):
    if n == 0 or n == 1:
        return 1
    else:
        return n * recursive_fact(n-1)
\end{lstlisting}
\end{minipage}%
\begin{minipage}{0.5\linewidth}
\begin{lstlisting}[linewidth=\linewidth]
def iterative_fact(n):
    total = 1
    while n > 1:
        total = total * n
        n = n - 1
    return total
\end{lstlisting}
\end{minipage}

Let's also compare {\tt fibonacci}.

\begin{minipage}{0.5\linewidth}
\begin{lstlisting}[linewidth=\linewidth]
def recursive_fib(n):
    if n == 0:
        return 0
    elif n == 1:
        return 1
    else:
        return (recursive_fib(n-1)
              + recursive_fib(n-2))
\end{lstlisting}
\end{minipage}%
\begin{minipage}{0.5\linewidth}
\begin{lstlisting}[linewidth=\linewidth]
def iterative_fib(n):
    prev, curr = 0, 1
    while n > 0:
        prev, curr = curr, prev + curr
        n = n - 1
    return prev
\end{lstlisting}
\end{minipage}

For the recursive version, we copied the definition of the Fibonacci sequence straight into code! The $n$th Fibonacci number is simply the sum of the two before it. In iteration, you need to keep track of more numbers and have a better understanding of the code.

Some code is easier to write iteratively and some recursively. Have fun experimenting with both!
