\question
What are three things you find in every recursive function?
\begin{solution}[0.75in]
\begin{verbatim}
1) Base Case(s)
2) Way(s) to reduce the problem into a smaller problem of the same type
3) Recursive case(s) that uses the solution of the smaller problem to solve the original (large) problem   
\end{verbatim}
\end{solution}

\question
When you write a Recursive function, you seem to call it before it has been fully defined. Why doesn't this break the Python interpreter?
\begin{solution}[0.75in]
When you define a function, Python does not evaluate the body of the function.
\end{solution}

\question
Below is a Python function that computes the nth Fibonacci number. Identify the three things it contains as a recursive function (from 1.1).
\begin{lstlisting}[language=Python]
def fib(n):
    if n == 0:
        return 0
    elif n == 1:
        return 1
    else:
        return fib(n-1) + fib(n-2)
|\begin{solution}[0.75in]
\begin{verbatim}
Domain is integers, range is integers.
Base Cases: if n == 0: ..., elif n == 1: ...
Finding Smaller Problems: finding fib(n - 1), fib(n - 2)
Recursive Case: when n is neither 0 nor 1, add together the fib(n - 1) and fib(n - 2) to find fib(n)
\end{verbatim}
\end{solution}|
\end{lstlisting}

\question
With the definition of the Fibonacci function above, draw out a diagram of the recursive calls made when \textbf{fib(4)} is called.
\begin{solution}[1in]
\begin{verbatim}
                            fib(4)
                            /    \
                        fib(3)   fib(2)
                       /    |     |    \
                   fib(2) fib(1) fib(1) fib(0)
                  /    |
              fib(1)  fib(0)
\end{verbatim}
\end{solution}
\newpage
\question
What does the following function \textbf{cascade2} do? What is its domain and range?
\begin{lstlisting}[language=Python]
def cascade2(n):
    print(n)
    if n >= 10:
        cascade2(n//10)
        print(n)
\end{lstlisting}
\begin{solution}[1in]
Domain is integers, range is None. It takes in a number n and prints out n, then prints out n excluding the ones digit, then prints n excluding the hundreds digit, and so on, then back up to the full number. 
\end{solution}


