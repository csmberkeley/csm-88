\question Write a function that takes two numbers {\tt m} and {\tt n} and
returns their product. Assume {\tt m} and {\tt n} are positive integers.
\textbf{Use recursion}, not \texttt{mul} or \texttt{*}!

Hint: \texttt{5*3 = 5 + 5*2 = 5 + 5 + 5*1}.

For the base case, what is the simplest possible input for \texttt{multiply}?
\begin{solution}[0.3in]
If one of the inputs is one, you simply return the other input.
\end{solution}

For the recursive case, what does calling \texttt{multiply(m - 1, n)} do?
What does calling \texttt{multiply(m, n - 1)} do?
Do we prefer one over the other?
\begin{solution}[0.5in]
The first call will calculate a value that is \texttt{m} less than the total,
while the second will calculate a value that is  \texttt{n} less.

Either recursive call will work, but only \texttt{multiply(m, n - 1)}  is
needed.
\end{solution}

\begin{blocksection}
\begin{lstlisting}
def multiply(m, n):
    """
    >>> multiply(5, 3)
    15
    """
\end{lstlisting}
\begin{solution}[1.25in]
\begin{lstlisting}
    if n == 1:
        return m
    else:
        return m + multiply(m, n - 1)
\end{lstlisting}
\href{https://youtu.be/VcZPTlE56G8?t=41m21s}{Video walkthrough}
\end{solution}
\end{blocksection}
