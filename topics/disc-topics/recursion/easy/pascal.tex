\question 
Here's a part of the Pascal's triangle:
\begin{lstlisting}
        Column:   0    1    2    3    4    ...
        Row 0:    1
        Row 1:    1    1
        Row 2:    1    2    1
        Row 3:    1    3    3    1
        Row 4:    1    4    6    4    1
        ...
\end{lstlisting}

Every number in Pascal's triangle is defined as the sum of the item
above it and the item that is directly to the upper left of it, use
{\tt 0} if the entry is empty.
Define the procedure {\tt pascal(row, column)} which takes a row and a column,
and finds the value at that position in the triangle. 

\begin{lstlisting}
def pascal(row, column):
\end{lstlisting}
\begin{solution}[2.3in]
\begin{lstlisting}
    if column == 0:
        return 1
    elif row == 0:
        return 0
    else:
        return pascal(row - 1, column) + \
               pascal(row - 1, column - 1)
\end{lstlisting}

\textbf{Background}: Pascal's triangle is a useful recursive definition that tells
us the coefficients in the expansion of the polynomial $(x + a)^n$.
Each element in the triangle has a coordinate, given by the row it
is on and its position in the row (which you could call its column).

\iffalse
If there is a position that does not have an entry, we treat it as
if we had a 0 there. Below are the first few rows of the triangle:
\fi

\end{solution}
