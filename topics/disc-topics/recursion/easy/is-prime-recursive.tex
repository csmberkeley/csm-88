\question In discussion 1, we implemented the function 
\texttt{is\_prime}, which takes in a positive integer
and returns whether or not that integer is prime, iteratively.

Now, let's implement it recursively! As a reminder, an integer
is considered prime if it has exactly two unique factors: 1 and itself.

\begin{lstlisting}
def is_prime(n):
    """
    >>> is_prime(7)
    True
    >>> is_prime(10)
    False
    >>> is_prime(1)
    False
    """
    def prime_helper(____________________):

        if   ________________________:

             ________________________

        elif ________________________:

             ________________________
        
        else:

             ________________________

    return __________________________
\end{lstlisting}

\begin{blocksection}
\begin{solution}[.5in]
\begin{lstlisting}
    def prime_helper(index):
        if index == n:
            return True
        elif n % index == 0 or n == 1:
            return False
        else:
            return prime_helper(index + 1)
    return prime_helper(2)
\end{lstlisting}
\end{solution}
\end{blocksection}
