A \textit{sequence} is an ordered collection of values. It has two
fundamental properties: length and element selection. In this
discussion, we'll explore one of Python's data types,
the \textit{list}, which implements this abstraction.

In Python, we can have lists of whatever values we want, be it
numbers, strings, functions, or even other lists!  Furthermore, the
types of the list's contents need not be the same. In other words, the
list need not be homogenous.

Lists can be created using square braces. Their elements can be
accessed (or \textit{indexed}) with square braces. Lists are
zero-indexed: to access the first element, we must index at 0; to access
the $i$th element, we must index at $i - 1$.

We can also index with negative numbers. These begin indexing at the
end of the list, so the index $-1$ is equivalent to the index
\texttt{len(list) - 1} and index $-2$ is the same as
\texttt{len(list) - 2}.

Let's try out some indexing:
\begin{lstlisting}
>>> pokemon_team = ['pikachu', 'dratini']
>>> print(pokemon_team)
['pikachu', 'dratini']
>>> pokemon_team[0]
'pikachu'
>>> pokemon_team[len(pokemon_team) - 1]
'dratini'
>>> pokemon_team[-1]
'dratini'
\end{lstlisting}
