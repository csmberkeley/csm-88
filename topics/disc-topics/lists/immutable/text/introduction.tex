\define{Introduction} 

In Python, \textit{lists} are ordered collections of whatever values we want, be it numbers, strings, functions, or even other lists! Each value stored inside a list is called an \textit{element}. We can create lists by using square braces.


\begin{lstlisting}
>>> foods = ['apple', 'oranges', 'banana', 'milk', 'cookies']
>>> print(foods)
['apple', 'oranges', 'banana', 'milk', 'cookies']
\end{lstlisting}

\define{Accessing elements} 

Lists are zero-indexed: to access the first element, we must access the element at index 0;  to access the $i$th element, we must index at $i - 1$.

We can also index with negative numbers. This begins indexing at the end of the list, so the index $-1$ is equivalent to the index \texttt{len(list) - 1} and index $-2$ is the same as \texttt{len(list) - 2}.

Examples:
\begin{lstlisting}
>>> foods[0]
'apple'
>>> foods[2]
'banana'
>>> foods[-3] 
'milk'
\end{lstlisting}

\define{Comprehension Check:} How would you access the element 'oranges' using a positive index? What about a negative index?


