If we want to access more than one element of a list at a time, we can use a
\textit{slice}. Slicing a sequence is very similar to indexing. We specify a
starting index and an ending index, separated by a colon. Python creates a new
list with the elements from the starting index up to (but not including) the
ending index. Specifically, we can write [\textit{start}:\textit{stop}] to slice a list with two integers.

\textit{start} denotes the index for the beginning of the slice (inclusive)\\
\textit{stop} denotes the index for the end of the slice (exclusive)

Using negative indices for start and end behaves in the same way as indexing
into negative indices. Slicing a list always creates a new list.

\begin{lstlisting}
>>> pizza = [1, 2, 3, 4]
>>> pizza[0]
1
>>> pizza[-1]
4
>>> pizza[-4]
1
>>> pizza[1:2]
[2]
>>> pizza[1:]
[2, 3, 4]
>>> pizza[-2:3]
[3]
\end{lstlisting}
