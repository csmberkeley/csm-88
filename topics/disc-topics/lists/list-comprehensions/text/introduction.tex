A \define{list comprehension} is a compact way to create a list whose elements
are the results of applying a fixed expression to elements in another sequence.
\begin{center}
\texttt{[<map exp> for <name> in <iter exp> if <filter exp>]}
\end{center}
It might be helpful to note that we can rewrite a list comprehension as an
equivalent for statement. See the example to the right.

Let's break down an example:
\begin{center}
\texttt{[x * x - 3 for x in [1, 2, 3, 4, 5] if x \% 2 == 1]}
\end{center}

In this list comprehension, we are creating a new list after performing a
series of operations to our initial sequence \texttt{[1, 2, 3, 4, 5]}. We only
keep the elements that satisfy the filter expression \texttt{x \% 2 == 1}
(\texttt{1}, \texttt{3}, and \texttt{5}). For each retained element, we apply
the map expression \texttt{x*x - 3} before adding it to the new list that we are
creating, resulting in the output \texttt{[-2, 6, 22]}.

\emph{Note}: The \texttt{if} clause in a list comprehension is optional.

