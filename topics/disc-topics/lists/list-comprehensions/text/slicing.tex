If we'd like to get back parts of the list, as opposed to single elements, we
{\it slice} the list.  Slicing a list returns us a \textbf{copy} of a chunk of
the original list. We use the following syntax:\\

\centerline{\tt lst[start:end:step]}

where start, end, and step are integers. The slice includes elements at indices
start, start+1*step, start+2*step, and so on up to end. It is legal to omit one
or more of start, end, and step; they default to 0, {\tt len(lst)}, and 1,
respectively.  Start and end can be negative, meaning you count from the end.

\begin{lstlisting}
>>> a = [0, 1, 2, 3, 4, 5, 6]
>>> a[1:4]
[1, 2, 3]
>>> a[1:6:2]
[1, 3, 5]
>>> a[:4]
[0, 1, 2, 3]
>>> a[3:]
[3, 4, 5, 6]
>>> a[-1:]
[6]
>>> a
[0, 1, 2, 3, 4, 5, 6]
\end{lstlisting}

