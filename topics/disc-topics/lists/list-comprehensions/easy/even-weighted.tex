\question
Write a function that takes a list and returns a new list that keeps only
the even-indexed elements of lst and multiplies them by their
corresponding index.

\begin{lstlisting}
def even_weighted(lst):
    """
    >>> x = [1, 2, 3, 4, 5, 6]
    >>> even_weighted(x)
    [0, 6, 20]
    """

    return [_________________________________________________]
\end{lstlisting}
\begin{solution}[0cm]
\begin{lstlisting}
    return [i * lst[i] for i in range(len(lst)) if i % 2 == 0]
\end{lstlisting}

Alternatively, we can take advantage of the step size for range to make sure we
only consider even numbered indices:

\begin{lstlisting}
    return [i * lst[i] for i in range(0, len(lst), 2)]
\end{lstlisting}

The key point to note is that instead of iterating over each element in the
list, we must instead iterate over the indices of the list. Otherwise, there's
no way to tell if we should keep a given element.

One way of solving these problems is to try and write your solution as a for loop first, and then transform it into a list comprehension. The for loop solution might look something like this:
\begin{lstlisting}
    result = []
    for i in range(len(lst)):
        if i % 2 == 0:
            result.append(i * lst[i])
    return result
\end{lstlisting}
\href{https://www.youtube.com/watch?v=Am6m8YgAnYY&list=PLx38hZJ5RLZdJgRCgpaTbmRXKAHOUmomO&index=4}{Video walkthrough}
\end{solution}
