Many of Python's primitive types are considered {\it immutable}, meaning that once they have been created, their value {\it cannot} change. Examples of these immutable types include strings, tuples, and numbers.

However, lists, dictionaries, and some other data types that are considererd {\it mutable}, meaning the values of a specific instance or object of that type {\it may change}.

Imagine you go to {\it CREAM} on Telegraph Avenue and you order an ice-cream sandwich. Suppose {\it CREAM} chooses to represent your order as a list like so:

\begin{lstlisting}
>>> sandwich = ['ice-cream', 'cookie']
\end{lstlisting}

Suppose that, while {\it CREAM} was preparing your order, you decide you want to top your sandwich with sprinkles. Without mutation, CREAM changes your order like so:

\begin{lstlisting}
# creates a new python list
>>> new_sandwich = sandwich + ['sprinkles']
>>> new_sandwich
['ice-cream', 'cookie', 'sprinkles']
>>> sandwich # the original list is unmodified
['ice-cream', 'cookie']
\end{lstlisting}

What was the point of {\it CREAM} having to make an entirely new sandwich just to add sprinkles? They could have simply modified the original sandwich! That's what mutation is all about! Instead, they could have done:

\begin{lstlisting}
>>> sandwich.append('sprinkles') # mutates original list
>>> sandwich
['ice-cream', 'cookie', 'sprinkles']
\end{lstlisting}