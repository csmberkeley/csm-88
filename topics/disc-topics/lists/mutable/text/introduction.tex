Let's imagine you order a mushroom and cheese pizza from La Val's, and that
they represent your order as a list:

\begin{lstlisting}
>>> pizza = ['cheese', mushrooms']
\end{lstlisting}

A couple minutes later, you realize that you really want onions on the pizza.
Based on what we know so far, La Val's would have to build an entirely new list
to add onions:

\begin{lstlisting}
>>> pizza = ['cheese', mushrooms']
>>> new_pizza = pizza + ['onions'] # creates a new python list
>>> new_pizza
['cheese', mushrooms', 'onions']
>>> pizza # the original list is unmodified
['cheese', 'mushrooms']
\end{lstlisting}

This is silly, considering that all La Val's had to do was add onions on top of
{\tt pizza} instead of making an entirely new pizza.

We can fix this issue with \define{list mutation}. In Python, some objects,
such as lists and dictionaries, are \define{mutable}, meaning that their
contents or state can be changed over the course of program execution. Other objects, such as
numeric types, tuples, and strings, are {\it immutable}, meaning they cannot be
changed once they are created.

Therefore, instead of building a new pizza, we can just mutate \texttt{pizza}
to add some onions!

\begin{lstlisting}
>>> pizza.append('onions')
>>> pizza
['cheese', 'mushrooms', 'onions']
\end{lstlisting}

\texttt{append} is what's known as a method, or a function that belongs to an
object, so we have to call it using dot notation. We'll talk more about methods 
later in the course, but for now, here's a list of useful list mutation methods:

\begin{enumerate}
\item {\tt append(el)}: Adds {\tt el} to the end of the list
\item {\tt extend(lst)}: Extends the list by concatenating it with {\tt lst}
\item {\tt insert(i, el)}: Insert {\tt el} at index {\tt i} (does not replace
element but adds a new one)
\item {\tt remove(el)}: Removes the first occurrence of {\tt el} in list,
otherwise errors
\item {\tt pop(i)}: Removes and returns the element at index {\tt i}
\end{enumerate}

We can also use the familiar indexing operator with an assignment statement to
change an existing element in a list. For example, we can change the element at index 1
and to \texttt{'tomatoes'} like so:

\begin{lstlisting}
>>> pizza[1] = 'tomatoes'
>>> pizza
['cheese', 'tomatoes', 'onions']
\end{lstlisting}
