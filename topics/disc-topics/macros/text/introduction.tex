%So far, we've mostly explored similarities between the Python and Scheme
%languages. For example, we saw how tail-call optimization allows us to write
%recursive Scheme functions that use a constant amount of space. This makes it
%feasible to translate iterative code from Python. We have also recently seen
%similarities in their handling of mutable functions.
%
%On the other hand, \textbf{macros} are a Scheme feature that don't have a
%apparent Python equivalent. Like functions, macros are a useful tool for
%simplifying code via abstraction. But while functions typically operate on
%values like numbers and lists, macros have the option of transforming
%unevaluated code, leading to a whole new world of possibilities!
%
%\marginpar{\subimport{../code/}{print.tex}}
%As a reminder, most Scheme functions do not have side effects. One
%exception to this is \texttt{print}. Just like in Python, \texttt{print}
%doesn't return anything! With that in mind, let's consider an example
%where we want to repeat a piece of code twice.

So far we've been able to define our own procedures in Scheme using the
\texttt{define} special form. When we call these procedures, we have to follow
the rules for evaluating call expressions, which involve evaluating all the
operands.

We know that special form expressions do not follow the evaluation rules of
call expressions. Instead, each special form has its own rules of evaluation,
which may include not evaluating all the operands.  Wouldn't it be cool if we
could define our own special forms where we decide which operands are
evaluated?  Consider the following example where we attempt to write a function
that evaluates a given expression twice:

\begin{lstlisting}
scm> (define (twice f) (begin f f))
twice
scm> (twice (print 'woof))
woof
\end{lstlisting}

Since \texttt{twice} is a regular procedure, a call to \texttt{twice} will
follow the same rules of evaluation as regular call expressions; first we
evaluate the operator and then we evaluate the operands.  That means that
\texttt{woof} was printed when we evaluated the operand \texttt{(print 'woof)}.
Inside the body of \texttt{twice}, the name \texttt{f} is bound to the value
\texttt{undefined}, so the expression \texttt{(begin f f)} does nothing at all!

%As an example of this, imagine if the problem were less constrained and we
%could surround our original expression in a \texttt{define} expression. In
%that case, we could use higher order functions to get what we want:
%\begin{lstlisting}[language=Scheme]
%scm> (define (speak) (print 'woof))
%speak
%scm> (define (twice f) (begin (f) (f)))
%twice
%scm> (twice speak)
%woof
%woof
%\end{lstlisting}
%But if the expression is given to us directly, there's no way to ``undo''
%the execution and delay it for later!
%\begin{lstlisting}[language=Scheme]
%scm> (define (twice result)
%       (begin
%         (define (f) result) % This won't work!
%         (f)(f)))
%twice
%scm> (twice (print 'woof))
%woof
%\end{lstlisting}
%Clearly, we need a special form, since we cannot evaluate our operand
%immediately.

The problem here is clear: we need to prevent the given expression from
evaluating until we're inside the body of the procedure. This is where the
\texttt{define-macro} special form, which has identical syntax to the regular
\texttt{define} form, comes in:

\begin{lstlisting}[language=Scheme]
scm> (define-macro (twice f) (list 'begin f f))
twice
\end{lstlisting}

\texttt{define-macro} allows us to define what's known as a \define{macro},
which is simply a way for us to combine unevaluated input expressions together
into another expression. When we call macros, the operands are not evaluated,
but rather are treated as Scheme data. This means that any operands that are
call expressions or special form expression are treated like lists.

If we call \texttt{(twice (print 'woof))}, \texttt{f} will actually be bound to
the list \texttt{(print 'woof)} instead of the value \texttt{undefined}.
Inside the body of \texttt{define-macro}, we can insert these expressions into
a larger Scheme expression. In our case, we would want a \texttt{begin}
expression that looks like the following:

\begin{lstlisting}[language=Scheme]
(begin (print 'woof) (print 'woof))
\end{lstlisting}

As Scheme data, this expression is really just a list containing three
elements: \texttt{begin} and \texttt{(print 'woof)} twice, which is exactly
what \texttt{(list 'begin f f)} returns.  Now, when we call \texttt{twice},
this list is evaluated as an expression and \texttt{(print 'woof)} is evaluated
twice.

\begin{lstlisting}
scm> (twice (print 'woof))
woof
woof
\end{lstlisting}

To recap, macros are called similarly to regular procedures, but the rules for
evaluating them are different. We evaluated lambda procedures in the following
way:

\begin{enumerate}[1.]
\item Evaluate operator
\item Evaluate operands
\item Apply operator to operands, evaluating the body of the procedure
\end{enumerate}

However, the rules for evaluating calls to macro procedures are:

\begin{enumerate}[1.]
\item Evaluate operator
\item Apply operator to unevaluated operands
\item Evaluate the expression returned by the macro in the frame it was called in.
\end{enumerate}

%This looks a bit like a function definition. \texttt{twice} is the name of
%the macro, and everything that follows in the same list is a required
%parameter. When we evaluate the macro form, we won't evaluate any
%parameters immediately. Instead, the body of the macro describes the final
%expression we want to evaluate, with the unevaluated parameters put in
%place! Recall that we want a final expression that looks like:
%\begin{lstlisting}[language=Scheme]
%(begin
%  (print 'woof)
%  (print 'woof))
%\end{lstlisting}
%Now, let \texttt{f} be the snippet of \texttt{print} code from earlier
%(not the result of evaluation, which is simply nothing) The expression:
%\begin{lstlisting}
%(list 'begin f f)
%\end{lstlisting}
%Creates our desired expression, and then finally evaluates it.
