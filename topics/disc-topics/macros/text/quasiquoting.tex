\marginpar{\subimport{../code/}{quote.tex}}

Recall that the \texttt{quote} special form prevents the Scheme interpreter
from executing a following expression. We saw that this helps us create complex
lists more easily than repeatedly calling \texttt{cons} or trying to get the
structure right with \texttt{list}. It seems like this form would come in handy
if we are trying to construct complex Scheme expressions with many nested
lists.

Consider that we rewrite the \texttt{twice} macro as follows:

\begin{lstlisting}
(define-macro (twice f)
  '(begin f f))
\end{lstlisting}

This seems like it would have the same effect, but since the \texttt{quote}
form prevents any evaluation, the resulting expression we create would actually
be \texttt{(begin f f)}, which is not what we want.

The \textbf{quasiquote} allows us to construct literal lists in a similar way
as quote, but also lets us specify if any sub-expression within the list should be
evaluated.

\marginpar{\subimport{../code/}{unquote.tex}}
At first glance, the quasiquote (which can be invoked with the backtick \`{} or
the \texttt{quasiquote} special form) behaves exactly the same as \texttt{'} or
\texttt{quote}.  However, using quasiquotes gives you the ability to
\textbf{unquote} (which can be invoked with the comma \texttt{,} or the
\texttt{unquote} special form). This removes an expression from the quoted
context, evaluates it, and places it back in.

By combining quasiquotes and unquoting, we can often save ourselves a lot
of trouble when building macro expressions.

Here is how we could use quasiquoting to rewrite our previous example:
\begin{lstlisting}[language=Scheme]
(define-macro (twice f)
  `(begin ,f ,f))
\end{lstlisting}
