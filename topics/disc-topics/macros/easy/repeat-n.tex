\question Write a macro that takes an expression and a number \texttt{n} and
repeats the expression \texttt{n} times. For example, \texttt{(repeat-n expr
2)} should behave the same as \texttt{(twice expr)}. Note that it's possible to
pass in a combination as the second argument (e.g. \texttt{(+ 1 2)}) as long as
it evaluates to a number. Be sure that you evaluate this expression in your
macro so that you don't treat it as a list.

Complete the implementation below, making use of the \texttt{replicate}
function given below. The \texttt{replicate} function takes in a value \texttt{x}
and a number \texttt{n} and returns a list with \texttt{x} repeated \texttt{n} 
times.

\begin{lstlisting}
(define (replicate x n)
  (if (= n 0) nil
    (cons x (replicate x (- n 1)))))

(define-macro (repeat-n expr n)
\end{lstlisting}

\begin{solution}[.7in]
\begin{lstlisting}
(define-macro (repeat-n expr n)
    (cons 'begin (replicate expr (eval n))))
\end{lstlisting}
\end{solution}
\begin{lstlisting}
scm> (repeat-n (print '(resistance is futile)) 3)
(resistance is futile)
(resistance is futile)
(resistance is futile)
scm> (repeat-n (print (+ 3 3)) (+ 1 1))  ; Pass a call expression in as n
6
6
\end{lstlisting}
