\question In Discussion 9, we saw how to write a macro
that creates a lambda function given an expression. While creating 
a parameter-less function might not seem that useful at first,
it can be helpful in many cases when we don't want to immediately
evaluate an expression.

Using the \texttt{make-lambda} macro you defined in Discussion 9, define
\texttt{make-stream}, a macro which returns a pair of elements, where
the second element is not evaluated until \texttt{cdr-stream} is called
on it. Also define the procedure \texttt{cdr-stream}, which takes in a
stream returned by \texttt{make-stream} and returns the result of evaluating
the second element in the stream pair. 

Unlike the streams we've seen in lecture and earlier in discussion, if you repeatedly 
call \texttt{cdr-stream} on a stream returned by \texttt{make-stream}, you may evaluate
an expression multiple times.

\begin{lstlisting}
(define-macro (make-stream first second)
\end{lstlisting}

\begin{solution}[.7in]
\begin{lstlisting}
(define-macro (make-stream first second)
	`(list ,first (make-lambda ,second)))
\end{lstlisting}
\end{solution}

\begin{lstlisting}
(define (cdr-stream stream)
\end{lstlisting}

\begin{solution}[.7in]
\begin{lstlisting}
(define (cdr-stream stream)
	((car (cdr stream))))
\end{lstlisting}
\end{solution}
\begin{lstlisting}
scm> (define a (make-stream (print 1) (make-stream (print 2) nil))))
1
a
scm> (define b (cdr-stream a))
2
b
scm> (cdr-stream b)
()
\end{lstlisting}
