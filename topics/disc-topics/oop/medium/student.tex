\question
Below we have defined the classes \texttt{Professor} and \texttt{Student}, implementing some of what was described above.
Remember that we pass the \texttt{self} argument implicitly to instance methods when using dot-notation.
There are more questions on the next page.
\lstinputlisting{student.py}

\newpage

What will the following lines output?
\newline
\begin{lstlisting}
>>> snape = Professor("Snape")
>>> harry = Student("Harry", snape)
\end{lstlisting}
\begin{solution}[.5in]
\begin{lstlisting}
There are now 1 students
\end{lstlisting}
\end{solution}
\begin{lstlisting}
>>> harry.visit_office_hours(snape)
\end{lstlisting}
\begin{solution}[.5in]
\begin{lstlisting}
Thanks, Snape
\end{lstlisting}
\end{solution}

\begin{lstlisting}
>>> harry.visit_office_hours(Professor("Hagrid"))
\end{lstlisting}
\begin{solution}[.5in]
\begin{lstlisting}
Thanks, Hagrid
\end{lstlisting}
\end{solution}

\begin{lstlisting}
>>> harry.understanding
\end{lstlisting}
\begin{solution}[.5in]
\begin{lstlisting}
2
\end{lstlisting}
\end{solution}

\begin{lstlisting}
>>> [name for name in snape.students]
\end{lstlisting}
\begin{solution}[.5in]
\begin{lstlisting}
['Harry']
\end{lstlisting}
\end{solution}

\begin{lstlisting}
>>> Student("Hermione", Professor("McGonagall")).name
\end{lstlisting}
\begin{solution}[.5in]
\begin{lstlisting}
There are now 2 students
'Hermione'
\end{lstlisting}
\end{solution}

\begin{lstlisting}
>>> [name for name in snape.students]
\end{lstlisting}
\begin{solution}[.5in]
\begin{lstlisting}
['Harry']
\end{lstlisting}
\end{solution}
