\question (Summer 2013 Final) What would Python display?

\begin{lstlisting}
class A:
    def f(self):
        return 2
    def g(self, obj, x):
        if x == 0:
            return A.f(obj)
        return obj.f() + self.g(self, x - 1)

class B(A):
    def f(self):
        return 4

>>> x, y = A(), B()
>>> x.f()
\end{lstlisting}
\begin{solution}[0.25in]
\lstinline{2}
\end{solution}

\begin{lstlisting}
>>> B.f()
\end{lstlisting}
\begin{solution}[0.25in]
\lstinline{Error} (missing self argument)
\end{solution}

\begin{lstlisting}
>>> x.g(x, 1)
\end{lstlisting}
\begin{solution}[0.25in]
\lstinline{4}
\end{solution}

\begin{lstlisting}
>>> y.g(x, 2)
\end{lstlisting}
\begin{solution}[0.25in]
\lstinline{8}\\
\href{https://www.youtube.com/watch?v=BatqjYa7WZ8&list=PLx38hZJ5RLZfel-Gi9pjaUbfQDCZIsWMU&vq=hd1080&t=48m20s}{Video walkthrough}
\end{solution}

\question (Summer 2013 Final) Implement the \texttt{Foo} class so that the
following interpreter session works as expected.

\begin{lstlisting}
>>> x = Foo(1)
>>> x.g(3)
4
>>> x.g(5)
6
>>> x.bar = 5
>>> x.g(5)
10
\end{lstlisting}

\begin{lstlisting}
class Foo:
\end{lstlisting}
\begin{solution}[0.5in]
\begin{lstlisting}
    def __init__(self, bar):
        self.bar = bar
    def g(self, n):
        return self.bar + n
\end{lstlisting}
\end{solution}
