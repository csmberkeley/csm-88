\question Below is a skeleton for the \texttt{Cat} class, which inherits from
the \texttt{Pet} class. To complete the implementation, override the
\texttt{\_\_init\_\_} and \texttt{talk} methods and add a new
\texttt{lose\char`_life} method.

\textit{Hint:} You can call the \texttt{\_\_init\_\_} method of \texttt{Pet} to
set a cat's \texttt{name} and \texttt{owner}.

\begin{lstlisting}
class Cat(Pet):
    def __init__(self, name, owner, lives=9):
        \end{lstlisting}

        \begin{solution}[.3in]
        \begin{lstlisting}
        Pet.__init__(self, name, owner)
        self.lives = lives
        \end{lstlisting}
        \end{solution}

        \begin{lstlisting}
    def talk(self):
        """ Print out a cat's greeting.
        >>> Cat('Thomas', 'Tammy').talk()
        Thomas says meow!
        """
        \end{lstlisting}

        \begin{solution}[.3in]
        \begin{lstlisting}
        print(self.name + ' says meow!')
        \end{lstlisting}
        \end{solution}

        \begin{lstlisting}
    def lose_life(self):
        """Decrements a cat's life by 1. When lives reaches zero, 'is_alive'
        becomes False.
        """
        \end{lstlisting}

        \begin{solution}[.3in]
        \begin{lstlisting}
        if self.lives > 0:
            self.lives -= 1
            if self.lives == 0:
                self.is_alive = False
        else:
            print("This cat has no more lives to lose :(")
        \end{lstlisting}
        \end{solution}
\begin{solution}
\href{https://www.youtube.com/watch?v=BatqjYa7WZ8&list=PLx38hZJ5RLZfel-Gi9pjaUbfQDCZIsWMU&vq=hd1080&t=34m50s}{Video walkthrough}
\end{solution}
