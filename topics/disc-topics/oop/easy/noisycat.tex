\question More cats! Fill in this implemention of a class called
\texttt{NoisyCat}, which is just like a normal \texttt{Cat}. However,
\texttt{NoisyCat} talks a lot -- twice as much as a regular \texttt{Cat}!

\begin{lstlisting}
class _____________________: # Fill me in!
\end{lstlisting}
\begin{solution}
\begin{lstlisting}
class NoisyCat(Cat):
\end{lstlisting}
\end{solution}
\begin{lstlisting}
    """A Cat that repeats things twice."""
    def __init__(self, name, owner, lives=9):
        # Is this method necessary? Why or why not?
        \end{lstlisting}

        \begin{solution}[.2in]
        \begin{lstlisting}
        Cat.__init__(self, name, owner, lives)
        \end{lstlisting}
        No, this method is not necessary because NoisyCat already inherits Cat's \_\_init\_\_ method
        \end{solution}

        \begin{lstlisting}
    def talk(self):
        """Talks twice as much as a regular cat.

        >>> NoisyCat('Magic', 'James').talk()
        Magic says meow!
        Magic says meow!
        """
        \end{lstlisting}

        \begin{solution}[.3in]
        \begin{lstlisting}
        Cat.talk(self)
        Cat.talk(self)
        \end{lstlisting}
        \end{solution}
\begin{solution}
\href{https://www.youtube.com/watch?v=BatqjYa7WZ8&list=PLx38hZJ5RLZfel-Gi9pjaUbfQDCZIsWMU&vq=hd1080&t=38m10s}{Video walkthrough}
\end{solution}
