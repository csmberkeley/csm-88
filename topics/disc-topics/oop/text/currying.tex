In a class method, you probably noticed that the first argument is always this
mysterious \texttt{self}. And somehow, we never seem to have to pass in the
argument \texttt{self} when we're calling it. This is the power of the magic
dot. It tells us that \texttt{self} is the instance that's before the
dot. However, if it happens to be the name of the class, then it's the method
itself and \texttt{self} isn't an automatic argument.

As to what this has to do with currying...try the questions and see. Let's start
with the Skittles and Bag classes above.

\begin{lstlisting}
>>> bag1 = Bag()
>>> def curried(f):
...     def outer(instance):
...         def inner(*args):
...             return f(instance, *args)
...         return inner
...     return outer
>>> add_binding = curried(Bag.add_skittle)
>>> bag1_add = add_binding(bag1)
>>> bag1.print_bag()
\end{lstlisting}
\begin{solution}[.2in]
\begin{lstlisting}
[]
\end{lstlisting}
\end{solution}

\begin{lstlisting}
>>> bag1.add_skittle(Skittle("blue"))
>>> bag1.print_bag()
\end{lstlisting}
\begin{solution}[.2in]
\begin{lstlisting}
['blue']
\end{lstlisting}
\end{solution}

\begin{lstlisting}
>>> bag1_add(Skittle("red"))
>>> bag1.add_skittle(Skittle("green"))
>>> bag1_add(Skittle("red"))
>>> bag1.print_bag()
\end{lstlisting}
\begin{solution}[.2in]
\begin{lstlisting}
['blue', 'red', 'green', 'red']
\end{lstlisting}
\end{solution}

\begin{lstlisting}
>>> s = bag1.take_skittle()
>>> bag2 = Bag()
>>> bag2_add = add_binding(bag2)
>>> bag2.print_bag()
\end{lstlisting}
\begin{solution}[.2in]
\begin{lstlisting}
[]
\end{lstlisting}
\end{solution}

\begin{lstlisting}
>>> bag2_add(Skittle("blue"))
>>> bag1.print_bag()
\end{lstlisting}
\begin{solution}[.2in]
\begin{lstlisting}
['red', 'green', 'red']
\end{lstlisting}
\end{solution}

\begin{lstlisting}
>>> bag2.print_bag()
\end{lstlisting}
\begin{solution}[.2in]
\begin{lstlisting}
['blue']
\end{lstlisting}
\end{solution}
