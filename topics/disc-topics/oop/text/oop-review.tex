In a previous lecture, you were introduced to the programming paradigm known as
Object-Oriented Programming (OOP). OOP allows us to treat data as objects - like
we do in real life.

For example, consider the \define{class} \texttt{Student}. Each of
you as individuals is an \define{instance} of this class.
So, a student \texttt{Angela} would be an instance of the class
\texttt{Student}.

Details that all CS 61A students have, such as \texttt{name}, \texttt{year}, and
\texttt{major}, are called \define{instance attributes}.  Every student has
these attributes, but their values differ from student to student.  An attribute
that is shared among all instances of \texttt{Student} is known as a
\define{class attribute}.
An example would be the \texttt{instructors} attribute; the instructor for CS 61A, Professor DeNero, is the same for every student in CS 61A.

All students are able to do homework, attend lecture, and go to office hours.
When functions belong to a specific object, they are said to be
\define{methods}.
In this case, these actions would be bound methods of
\texttt{Student} objects.

Here is a recap of what we discussed above:
\begin{description}
  \item[$\bullet$ class:] a template for creating objects
  \item[$\bullet$ instance:] a single object created from a class
  \item[$\bullet$ instance attribute:] a property of an object, specific to an instance
  \item[$\bullet$ class attribute:] a property of an object, shared by all instances of a class
  \item[$\bullet$ method:] an action (function) that all instances of a class may perform
\end{description}
