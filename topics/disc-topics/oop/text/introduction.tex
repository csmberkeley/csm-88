This week, you were introduced to the programming paradigm known as
Object-Oriented Programming. If you've programmed in a language like Java or
C++, this concept should already be familiar to you.

Object-oriented programming (OOP) is heavily based on the idea of data
abstraction. Think of objects as how you would an object in real life.

For our example, let's think of your laptop. First of all, it must have gotten
its design from somewhere and that blueprint is called a \define{class}. The
laptop itself is an \define{instance} of that class. If your friend has the same
laptop as you, those laptops are just different instances of the same class.

Your laptop performs many actions, e.g.\ turning on, displaying text, etc\@.
Those are called \define{methods}. It also has properties, e.g.\ screen
resolution, how much memory it has, that scratch mark you hope no one else
sees. Those are called \define{attributes}. If it's an attribute that's the same
for all instances, it's called a \define{class attribute}. So, if you were
wondering how many instances of your laptop exists, that would be a class
attribute because no matter which instance got asked that, it would be the same.
If you were wondering how many scratches your laptop has, that's an
\define{instance attribute} because that number depends on each instance.

When discussing objects and classes, it is helpful to distinguish between the
definition of a class and the instantiation of a class, or an object. The
instantiation is referred to as an ``object'', whereas the definition is the
``class''. Following our example, \texttt{OurClass} is a class, while
\texttt{new\_bar} is an instantiation of that class, also referred to as an
object.

So, that's the vocabulary of OOP. (Yes, people say that -- it's quite fun! As
a bonus warm-up, you should say it too.)

