Finally, to use a class or instance's attributes, we use ``dot notation'', which
is aptly named for the use of the magic dot. The dot asks the class for the
value of the attribute. So, if we have an attribute, \texttt{bar}, of a class or
instance, \texttt{foo}, we access it by saying: ``\texttt{foo.bar}'' which says
``Almighty \texttt{foo} class, what is the value of the attribute \texttt{bar}?''

Typically, attributes are defined in the \texttt{\_\_init\_\_} function of a
class:
\begin{lstlisting}
class OurClass(ParentClass):
    bar = "Fruit Bar" # class attribute
    def __init__(self, bar_name):
        self.bar = bar_name # instance attribute
    def method(self, arg):
        # body goes here
\end{lstlisting}

Once an object is constructed, you can also access the attribute by using dot
notation \emph{outside} of the class definition:
\begin{lstlisting}
>>> new_bar = OurClass('Crazy Bar')
>>> new_bar.bar
'Crazy Bar'
\end{lstlisting}

