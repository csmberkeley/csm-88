
Dictionaries are data structures which map keys to values.
Dictionaries in Python are unordered, unlike real-world dictionaries --- in
other words, key-value pairs are not arranged in the dictionary in any
particular order. Let's look at an example:

\begin{lstlisting}

>>> pokemon = {'pikachu': 25, 'dragonair': 148, 'mew': 151}
>>> pokemon['pikachu']
25
>>> pokemon['jolteon'] = 135
>>> pokemon
{'jolteon': 135, 'pikachu': 25, 'dragonair': 148, 'mew': 151}
>>> pokemon['ditto'] = 25
>>> pokemon
{'jolteon': 135, 'pikachu': 25, 'dragonair': 148,
'ditto': 25, 'mew': 151}


\end{lstlisting}
\vspace{5}
The {\it keys} of a dictionary can be any {\it immutable} value, such as
numbers, strings, and tuples.\footnote{To be exact, keys must be {\it
hashable}, which is out of scope for this course. This means that some
mutable objects, such as classes, can be used as dictionary keys.}
Dictionaries themselves are mutable; we can add,
remove, and change entries after creation. There is only one value per key,
however --- if we assign a new value to the same key, it overrides any previous
value which might have existed.

To access the value of {\tt dictionary} at {\tt key}, use the syntax
{\tt dictionary[key]}.

Element selection and reassignment work similarly to sequences, except the
square brackets contain the key, not an index.
