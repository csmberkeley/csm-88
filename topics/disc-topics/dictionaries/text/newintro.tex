
Dictionaries are containers that \textbf{map keys to values}.
Let's look at an example:

\begin{lstlisting}

>>> pokemon = {'pikachu': 25, 'dragonair': 148, 'mew': 151}
>>> pokemon['pikachu']
25
>>> pokemon['jolteon'] = 135
>>> pokemon
{'jolteon': 135, 'pikachu': 25, 'dragonair': 148, 'mew': 151}
>>> pokemon['ditto'] = 25
>>> pokemon
{'jolteon': 135, 'pikachu': 25, 'dragonair': 148,
'ditto': 25, 'mew': 151}


\end{lstlisting}
\vspace{5}
The {\it keys} of a dictionary must be {\it immutable} values, such as
numbers, strings, tuples, etc.
Dictionaries themselves are mutable; we can add,
remove, and change entries after creation. Finally, there is only one value per key,
however --- if we assign a new value to the same key, it overrides any previous
value which might have existed. See below for some common uses of dictionaries:

\begin{itemize}
\item To add {\tt val} corresponding to {\tt key} {\it or} to replace the current value of {\tt key} with {\tt val}:
\begin{lstlisting}
    dictionary[key] = val
\end{lstlisting}
\item To iterate over a dictionary's keys:
\begin{lstlisting}
    for key in dictionary: #OR for key in dictionary.keys()
        do_stuff()
\end{lstlisting}
\item To iterate over a dictionary's values:
\begin{lstlisting}
    for value in dictionary.values():
        do_stuff()
\end{lstlisting}
\item To iterate over a dictionary's keys and values:
\begin{lstlisting}
    for key, value in dictionary.items():
        do_stuff()
\end{lstlisting}
\item To remove an entry in a dictionary:
\begin{lstlisting}
    del dictionary[key]
\end{lstlisting}
\item To get the value corresponding to {\tt key} and remove the entry:
\begin{lstlisting}
    dictionary.pop(key)
\end{lstlisting} 
\end{itemize}

