In lecture, we discussed the {\tt rational} data type, which represents
fractions with the following methods:

\begin{itemize}
\item {\tt rational(n, d)} - constructs a rational number with numerator {\tt
n}, denominator {\tt d}
\item {\tt numer(x)} - returns the numerator of rational number {\tt x}
\item {\tt denom(x)} - returns the denominator of rational number {\tt x}
\end{itemize}

We also presented the following methods that perform operations with rational
numbers:

\begin{itemize}
\item {\tt add\_rationals(x, y)}
\item {\tt mul\_rationals(x, y)}
\item {\tt rationals\_are\_equal(x, y)}
\end{itemize}

There is a lot we can do with just these simple methods.
