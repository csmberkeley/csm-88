Data abstraction is a powerful concept in computer science that
allows programmers to treat code as objects --- for example,
car objects, chair objects, people objects, etc. That way,
programmers don't have to worry about \texttt {how} code is
implemented --- they just have to know \texttt{what} it does.

Data abstraction mimics how we think about the world. For example,
when you want to drive a car, you don't need to know how the
engine was built or what kind of material the tires are made of.
You just have to know how to turn the wheel and press the gas pedal.

An \textit{abstract data type} consists of two types of functions:
\begin{itemize}
\item
Constructors: functions that build the abstract data type.
\item
Selectors: functions that retrieve information from the data type.
\end{itemize}
For example, say we have an abstract data type called {\tt city}.
This {\tt city} object will hold the {\tt city}'s name, and
its latitude and longitude. To create a {\tt city} object,
you'd use a constructor like

\begin{lstlisting}
city = make_city(name, lat, lon)
\end{lstlisting}

To extract the information of a {\tt city} object, you would use the selectors like

\begin{lstlisting}
get_name(city)
get_lat(city)
get_lon(city)
\end{lstlisting}

For example, here is how we would use the {\tt make\_city} constructor to create a city object to represent Berkeley
and the selectors to access its information.

\begin{lstlisting}
>>> berkeley = make_city('Berkeley', 122, 37)
>>> get_name(berkeley)
'Berkeley'
>>> get_lat(berkeley)
122
>>> get_lon(berkeley)
37
\end{lstlisting}

The following code will compute the distance between two city objects:

\begin{lstlisting}
from math import sqrt
def distance(city_1, city_2):

    lat_1, lon_1 = get_lat(city_1), get_lon(city_1)
    lat_2, lon_2 = get_lat(city_2), get_lon(city_2)

    return sqrt((lat_1 - lat_2)**2 + (lon_1 - lon_2)**2)
\end{lstlisting}

Notice that we don't need to know how these functions were implemented.  We are
assuming that someone else has defined them for us.

It's okay if the end user doesn't know how functions were implemented.  However,
the functions still have to be defined by someone. We'll look into defining the
constructors and selectors later in this discussion.
