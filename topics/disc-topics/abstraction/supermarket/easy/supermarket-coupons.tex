\question Implement the coupon abstraction using functional pairs.
A coupon pairs a string item name with an integer amount of discount
representing a deduction from the cost.

\begin{lstlisting}
def make_coupon(name, discount):
\end{lstlisting}
\begin{solution}[0.5in]
\begin{lstlisting}
    return pair(name, discount)
\end{lstlisting}
\end{solution}

\begin{lstlisting}
def get_coupon_name(coupon):
\end{lstlisting}
\begin{solution}[0.5in]
\begin{lstlisting}
    return select(coupon, 0)
\end{lstlisting}
\end{solution}

\begin{lstlisting}
def get_coupon_discount(coupon):
\end{lstlisting}
\begin{solution}[0.5in]
\begin{lstlisting}
    return select(coupon, 1)
\end{lstlisting}
\end{solution}

\question Write {\tt discount\_purchase}, which calculates the total price of a
list of items after being discounted by a list of coupons.

\begin{lstlisting}
def discount_purchase(items, coupons):
\end{lstlisting}
\begin{solution}[2in]
\begin{lstlisting}
    total_cost = 0
    for item in items:
        total_cost += get_item_price(item)
        for coupon in coupons:
            if get_coupon_name(coupon) == get_item_name(item):
                total_cost -= get_coupon_discount(coupon)
    return total_cost
\end{lstlisting}
\end{solution}
