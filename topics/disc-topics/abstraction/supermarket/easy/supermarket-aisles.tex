\question Implement the \emph{supermarket} abstraction.

\begin{lstlisting}
def make_supermarket(items):
\end{lstlisting}
\begin{solution}[0.5in]
\begin{lstlisting}
    return items
\end{lstlisting}
\end{solution}

\begin{lstlisting}
def get_supermarket_items(supermarket):
\end{lstlisting}
\begin{solution}[0.5in]
\begin{lstlisting}
    return supermarket
\end{lstlisting}
\end{solution}

\clearpage

\question Write the {\tt shopping} function. Given a grocery list (a list of
string item names), a supermarket, and coupon list, calculate the total cost of
purchasing every item on the grocery list. You may use any functions you've
already written.

\begin{lstlisting}
def shopping(grocery_list, supermarket, coupons):
    """
    >>> grocery_list = ["Butter", "Bread"]
    >>> items = [create_item("Butter", 2),
    ...          create_item("Bread", 3),
    ...          create_item("Soup", 4)]
    >>> supermarket = make_supermarket(items)
    >>> coupons = [create_coupon("Butter", 1),
    ...            create_coupon("Soup", 1)]
    >>> shopping(grocery_list, supermarket, coupons)
    4
    """
\end{lstlisting}
\begin{solution}[3in]
\begin{lstlisting}
    items = []
    for item_name in grocery_list:
        for item in get_supermarket_items(supermarket):
            if item_name == get_item_name(item):
                items.append(item)
    return discount_purchase(items, coupons)
\end{lstlisting}
\end{solution}

