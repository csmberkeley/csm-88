\begin{blocksection}
\question
In big-$\Theta$ notation, what is the runtime for \texttt{foo}?
\begin{parts}
\part
\begin{lstlisting}
def foo(n):
    for i in range(n):
        print('hello')
\end{lstlisting}
\begin{solution}[0.25in]
$\Theta(n)$. This is simple loop that will run $n$ times.
\end{solution}

\part What's the runtime of \texttt{foo} if we change \texttt{range(n)}:
\begin{subparts}

\subpart To \texttt{range(n / 2)}?
\begin{solution}[0in]
$\Theta(n)$. The loop runs $n / 2$ times, but we ignore constant factors.
\end{solution}

\subpart To \texttt{range(10)}?
\begin{solution}[0in]
$\Theta(1)$. No matter the size of $n$, we will run the loop the same number of
times.
\end{solution}

\subpart To \texttt{range(10000000)}?
\begin{solution}[0in]
$\Theta(1)$. No matter the size of $n$, we will run the loop the same number of
times.
\end{solution}

\end{subparts}
\end{parts}
\end{blocksection}
