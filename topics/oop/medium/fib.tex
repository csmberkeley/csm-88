\begin{blocksection}
\question Lets use OOP to help us implement our good friend, the fibonacci sequence!

As a reminder, here is the fib sequence

$0, 1, 1, 2, 3, 5, 8, 13, 21$


\begin{lstlisting}
fib(0) = 0
fib(1) = 1
fib(n) = fib(n - 1) + fib(n - 2)
\end{lstlisting}


\vspace{2\baselineskip}
\begin{nonsol}
\begin{lstlisting}
>>> tracker1 = FibTracker()
>>> tracker2 = FibTracker()
>>> tracker1.next()
0
>>> tracker1.next()
1
>>> tracker1.next()
1
>>> tracker2.next()
0
>>> tracker1.next()
2
>>> tracker1.next()
3
>>> tracker1.next()
5
>>> tracker2.next()
1

class FibTracker:
    def __init__(self):






    def next(self):
\end{lstlisting}
\end{nonsol}

\begin{solution}[0.3in]
\begin{lstlisting}
class FibTracker:
    def __init__(self):
        self.a = 0
        self.b = 1

    def next(self):
        result = self.a
        self.a, self.b = self.b, self.a + self.b
        return result
\end{lstlisting}
\end{solution}

\end{blocksection}
