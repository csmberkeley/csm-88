\begin{blocksection}
\question \textbf{Hello Zoom} \newline
Given the class \lstinline{ZoomCall} defined below, what will Python print for the following prompts? 

\begin{lstlisting}
class ZoomCall:

    duration = 40
    def __init__(self, teacher, students):
        self.teacher = teacher
        self.students = students
        self.time_spent = 0

    def attempt_problem(self, difficulty):
        groups = self.breakout_groups()
        for group in groups:
            print("How's it going " + str(group[0]))
        self.time_spent += difficulty

    def breakout_groups(self):
        groups = [[], [], []]
        for i in range(len(self.students)):
            groups[i % 3].append(self.students[i])
        return groups

    def time_remaining(self):
        return self.duration - self.time_spent
       
\end{lstlisting}
\end{blocksection}
\newline
\newline
\newline
\begin{blocksection}

\begin{lstlisting}
>>> call_one = ZoomCall('Alina', ['Amit', 'Lukas', 'Vikram', 'Warren', 'Tony'])
>>> call_one.teacher
\end{lstlisting}
\begin{solution}[.2in]
'Alina'
\end{solution}

\begin{lstlisting}
>>> call_one.attempt_problem(25)
\end{lstlisting}
\begin{solution}[.2in]
\begin{lstlisting}
How's it going Amit
How's it going Lukas
How's it going Vikram
\end{lstlisting}
\end{solution}

\begin{lstlisting}
>>> call_one.time_spent
\end{lstlisting}
\begin{solution}[.2in]
25
\end{solution}
\end{blocksection}

\begin{blocksection}
\begin{lstlisting}
>>> call_one.time_remaining()
\end{lstlisting}
\begin{solution}[.2in]
15
\end{solution}

\begin{lstlisting}
>>> call_one.duration = 20
>>> call_one.time_remaining()
\end{lstlisting}
\begin{solution}[.2in]
-5
\end{solution}

\begin{lstlisting}
>>> call_two = ZoomCall("Nikhil", ["Kaitlyn", "Shreya"])
>>> call_two.duration
\end{lstlisting}
\begin{solution}[.2in]
40
\end{solution}
\end{blocksection}
\newpage
\question Add the methods \lstinline$overtime$ and \lstinline$promote$ to the ZoomCall class so they execute the behavior shown below. You can assume that we continue executing the following code \emph{after} executing the code above.

\begin{lstlisting}
>>> call_one.overtime()
True
>>> call_two.overtime()
False
>>> call_two.time_spent = call_two.duration + 1
>>> call_two.overtime()
True
>>> call_two.promote('Shreya')
>>> call_two.teacher
'Shreya'
>>> call_two.students
['Kaitlyn', 'Nikhil']
>>> call_two.promote('Chi')
>>> call_two.teacher
'Chi'
>>> call_two.students
['Kaitlyn', 'Nikhil', 'Shreya']
\end{lstlisting}
\bigbreak
\begin{lstlisting}
def overtime(self):
    
\end{lstlisting}

\begin{solution}[.5in]
\begin{lstlisting}
    return self.time_remaining() < 0
\end{lstlisting}
\end{solution}

\begin{lstlisting}
def promote(self, new_teacher):
    
\end{lstlisting}

\begin{solution}[1.5in]
\begin{lstlisting}
    if new_teacher in self.students:
        self.students.remove(new_teacher)
    self.students.append(self.teacher)
    self.teacher = new_teacher
\end{lstlisting}
\end{solution}