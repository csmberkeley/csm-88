\begin{blocksection}
\question \textbf{Hello Zoom} \newline
Given the class ZoomCall defined below, what will Python print for the following prompts? 

\begin{lstlisting}
class ZoomCall:

    duration = 40
    def __init__(self, teacher, group):
        self.teacher = teacher
        self.group = group
        self.time_spent = 0

    def attempt_problem(self, difficulty):
        groups = self.breakout_groups()
        for group in groups:
            print("How's it going " + str(group[0]))
        self.time_spent += difficulty

    def breakout_groups(self):
        answers = []
        copy = list(self.group)
        while copy:
            mini = []
            for i in range(2):
                if copy:
                    mini.append(copy.pop())
            answers.append(mini)
        return answers

    def time_remaining(self):
        return self.duration - self.time_spent
       
\end{lstlisting}
\end{blocksection}
\newline
\newline
\newline
\begin{blocksection}

\begin{lstlisting}
>>> call_one = ZoomCall("Ada", ["Nicholas", "Jack", "Megan"])
>>> call_one.teacher
\end{lstlisting}
\begin{solution}[.2in]
'Ada'
\end{solution}

\begin{lstlisting}
>>> call_one.attempt_problem(25)
\end{lstlisting}
\begin{solution}[.2in]
\begin{lstlisting}
How's it going Megan
How's it going Nicholas
\end{lstlisting}
\end{solution}

\begin{lstlisting}
>>> call_one.time_spent
\end{lstlisting}
\begin{solution}[.2in]
25
\end{solution}
\end{blocksection}

\begin{blocksection}
\begin{lstlisting}
>>> call_one.time_remaining()
\end{lstlisting}
\begin{solution}[.2in]
15
\end{solution}

\begin{lstlisting}
>>> call_one.duration = 20
>>> call_one.time_remaining()
\end{lstlisting}
\begin{solution}[.2in]
-5
\end{solution}

\begin{lstlisting}
>>> call_two = ZoomCall("Nikhil", ["Kaitlyn", "Shreya"])
>>> call_two.duration
\end{lstlisting}
\begin{solution}[.2in]
40
\end{solution}
\end{blocksection}

\question Add the methods \lstinline$overtime$ and \lstinline$overthrow$ to the ZoomCall class so they execute the behavior shown below. You can assume that we continue executing the following prompts \emph{after} executing the prompts above.

\begin{blocksection}
\begin{lstlisting}
>>> call_one.overtime()
True
>>> call_two.overtime()
False
>>> call_two.overthrow('Shreya')
>>> call_two.teacher
'Shreya'
>>> call_two.group
['Kaitlyn', 'Nikhil']
>>> call_two.overthrow('Chi')
>>> call_two.teacher
'Chi'
>>> call_two.group
['Kaitlyn', 'Nikhil', 'Shreya']
\end{lstlisting}
\end{blocksection}

\begin{lstlisting}
def overtime(self):
    
\end{lstlisting}

\begin{solution}[.5in]
\begin{lstlisting}
    return self.time_remaining() < 0
\end{lstlisting}
\end{solution}

\begin{lstlisting}
def overthrow(self, new_leader):
    
\end{lstlisting}

\begin{solution}[1.5in]
\begin{lstlisting}
    if new_leader in self.group:
        self.group.remove(new_leader)
    self.group.append(self.teacher)
    self.teacher = new_leader
\end{lstlisting}
\end{solution}