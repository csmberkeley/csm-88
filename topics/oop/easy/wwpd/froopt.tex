\documentclass[10pt]{article}
\usepackage[usenames]{color} %used for font color
\usepackage{amssymb} %maths
\usepackage{amsmath} %maths
\usepackage[utf8]{inputenc} %useful to type directly diacritic characters
\begin{document}
\[\[\begin{blocksection}
\question \textbf{(H)OOP} \newline
Given the following code, what will Python output for the following prompts? 

\begin{lstlisting}
class Fruit:
    ripe = False
    def __init__(self, taste, size):
       self.taste = taste
       self.size = size
       self.ripe = True
    
    def eat(self, eater):
       print(eater, 'eats the', 'self.name)
       if not self.ripe:
          print('But it is not ripe!')
       else:
          print('What a', self.taste, 'and', self.size, 'fruit!')

class Tomato(Fruit):
    name = 'tomato'
    def eat(self, eater):  
       print('Adding some sugar first')
       self.taste = 'sweet'
       Fruit.eat(self, eater) 
       
\end{lstlisting}
\end{blocksection}
\newline
\newline
\newline
\begin{blocksection}

\begin{lstlisting}
>>> mystery = Friuit('tart', 'small')
>>> tommy = Tomato('plain', 'normal')
>>> mystery.taste
\end{lstlisting}
\begin{solution}[.2in]
'tart'
\end{solution}

\begin{lstlisting}
>>> mystery.name
\end{lstlisting}
\begin{solution}[.2in]
Error
\end{solution}

\begin{lstlisting}
>>> tommy.eat('Brian')
\end{lstlisting}
\begin{solution}[.2in]
Adding some sugar first \newline
Brian eats the tomato
What a sweet and normal fruit!
\end{solution}
\end{blocksection}

\begin{blocksection}
\begin{lstlisting}
>>> Tomato.ripe
\end{lstlisting}
\begin{solution}[.2in]
False
\end{solution}

\begin{lstlisting}
>>> tommy.name = 'sweet tomato'
>>> Fruit.eat = lambda self, own : print(self.name, 'is too sweet!')
>>> tommy.eat('Marvin')
\end{lstlisting}
\begin{solution}[.2in]
Adding some sugar first \newline
sweet tomato is too sweet!
\end{solution}


\end{blocksection}
\]
\]
\end{document}