\begin{blocksection}
\question Each digit in a non-negative integer n has a digit position. Digit positions begin at 0 and count from the right-most digit of n. For example, in 568789, the digit 9 is at position 0 and digit 7 is at position 2. The digit 8 appears at both positions 1 and 3. 

Implement the find\_digit function, which takes a non-negative integer n and a digit d greater than 0 and less than 10. It returns the largest (left-most) position in n at which digit d appears. If d does not appear in n, then find\_digit returns False. You may not use recursive calls.
\vspace{10}

\begin{lstlisting}
def find_digit (n , d ): 
    """ Return the largest digit position in n for which d is the digit . 
    >>> find_digit (567 , 7) 
    0 
    >>> find_digit (567 , 5) 
    2 
    >>> find_digit (567 , 9) 
    False 
    >>> find_digit (568789 , 8) # 8 appears at positions 1 and 3 
    3 
    """ 
    i = 0
    
    k = ___________________________________________________
    
    while n: 
    
        n, last = n // 10, n % 10 

        if last == ___________:

            _____________________
            
        i = i + 1 

    return ___________________________________________________

\end{lstlisting}
\begin{solution}[1.5in]
\begin{lstlisting}
i , k = 0 , False
while n:
    n , last = n // 10 , n % 10
    if last == d :
        k = i
    i = i + 1
return k
\end{lstlisting}
\end{solution}
\end{blocksection}
