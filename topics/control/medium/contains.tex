\question
\begin{parts}
\part 
Given a list \lstinline{lst}, and an index \lstinline{i}, return whether or not \lstinline{num} appears at index \lstinline{i} or onwards in the given \lstinline{lst}.
\begin{lstlisting}
def contains_num_after_i(lst, num, i):
    """ 
    >>> contains_num_after_i([1, 11, 3, 4, 5, 6, 7, 8], 11, 3)
    False
    >>> contains_num_after_i([1, 2, 11, 4, 5, 6, 7, 8], 11, 3)
    False
    >>> contains_num_after_i([1, 2, 3, 11, 5, 6, 7, 8], 11, 3)
    True
    >>> contains_num_after_i([1, 11, 3, 4, 5, 6, 7, 11], 11, 5)
    True
    """
\end{lstlisting}
\begin{solution}[1.5in]
\begin{lstlisting}
    for j in range(i, len(lst)):
        if lst[j] == num:
            return True
    return False
\end{lstlisting}
\end{solution}

\newpage
\part Return whether or not there are duplicates in the given \lstinline{lst}. Hint: Call the function above!

\begin{lstlisting}
def duplicates(lst):
    """ 
    >>> duplicates([1, 11, 3, 4, 5, 6, 7, 8])
    False
    >>> duplicates([1, 2, 11, 4, 5, 6, 7, 8])
    False
    >>> duplicates([1, 2, 3, 11, 5, 6, 7, 3, 8])
    True
    >>> duplicates([1, 11, 4, 4, 9, 5, 6, 7, 11])
    True
    """
\end{lstlisting}
\begin{solution}
\begin{lstlisting}
    for i in range(len(lst)):
        if contains_num_after_i(lst, lst[i], i):
            return True
    return False
\end{lstlisting}
\end{solution}
\end{parts}
