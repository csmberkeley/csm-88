\question We will now write one of the faster sorting algorithms
commonly used, known as {\it merge sort}. Merge sort works like this:
\begin{enumerate}
    \item If there is only one (or zero) item(s) in the sequence,
        it is already sorted!
    \item If there are more than one item, then we can split the
        sequence in half, sort each half recursively, then merge
        the results, using the {\tt merge} procedure from earlier
        in the notes. The result will be a sorted sequence.
\end{enumerate}
Using the algorithm described, write a function {\tt mergesort(seq)}
that takes an unsorted sequence and sorts it.
\begin{lstlisting}
def mergesort(seq):
\end{lstlisting}
\begin{solution}[1.7in]
\begin{lstlisting}
    if len(seq) <= 1:
        return seq
    else:
        return merge(mergesort(seq[:len(seq)//2]),
            mergesort(seq[len(seq)//2:]))
\end{lstlisting}
\end{solution}
