\begin{blocksection}
\question Implement the function \lstinline$filter_out$ that takes in a list \lstinline$lst$ and returns a one argument function, let’s arbitrarily call this \lstinline$g$. \lstinline$g$ takes in a one argument function \lstinline$f$  and returns a pair — a new list containing only the elements of \lstinline$lst$ that return True when passed in to \lstinline$f$, and a one argument function that behaves identically to \lstinline$g$ but operates on the filtered list.

\begin{lstlisting}
def filter_out(lst):
    """
    >>> g = filter_out([1, 2, 3, 4, 5])
    >>> lst, b = g(lambda x: x < 4)
    >>> lst
    [1, 2, 3]
    >>> lst, c = b(lambda y: y % 2 == 0)
    >>> lst
    [2]
    """

    
\end{lstlisting}

\begin{solution}[1.5in]
\begin{lstlisting}
#Solution using list comprehension
def filter_out(lst):
    def helper(f):
        answer = [x for x in lst if f(x)]
        return answer, filter_out(answer)
    return helper
    
#Solution using for loop
def filter_out(lst):
    def helper(f):
        answer = []
        for x in lst:
            if f(x):
                answer.append(x)
        return answer, filter_out(answer)
    return helper
\end{lstlisting}
\end{solution}
\end{blocksection}
