\begin{blocksection}
\question (61A Fall 2015 MT1 Q4A) Implement the higher-order function \lstinline$decompose1$, 
which takes two functions \lstinline$f$ and \lstinline$h$ as arguments. It returns a function
\lstinline$g$ that relates \lstinline$f$ to \lstinline$h$ in the following way: For any 
non-negative integer \lstinline$x$, \lstinline$h(x)$ equals \lstinline$f(g(x))$. Assume that 
\lstinline$decompose1$ will be called only on arguments for which such a function 
\lstinline$g$ exists. Good practice for CS 88 midterm \\

\begin{lstlisting}
def decompose1(f, h):
    """
    Return g such that h(x) equals f(g(x)) for any non-negative integer x.
    
    >>> def add_one(x):
    ...     return x + 1
    >>> def square_then_add_one(x):
    ...     return x * x + 1
    >>> g = decompose1(add_one, square_then_add_one)
    >>> g(5)
    25
    >>> g(10)
    100
    """
    def g(x):
        def r(y):
            if ____________________________________________:
                return ____________________________________
            else:
                return ____________________________________
        return r(0)
    _______________________________________________________
    
\end{lstlisting}

\begin{solution}[1.5in]
\begin{lstlisting}
def decompose1(f, h):
    def g(x):
        def r(y):
            if h(x) == f(y):
                return y
            else:
                return r(y+1)
        return r(0)
    return g
\end{lstlisting}
\end{solution}
\end{blocksection}
