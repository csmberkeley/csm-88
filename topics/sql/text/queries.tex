We can use a \texttt{SELECT} statement to create tables. The following statement
creates a table with a single row, with columns named ``first'' and ``last'':

\begin{lstlisting}[language=SQL]
  sqlite> SELECT "Ben" AS first, "Bitdiddle" AS last;
  Ben|Bitdiddle
\end{lstlisting}

Given two tables with the same number of columns, we can combine their rows
into a larger table with \texttt{UNION}:

\begin{lstlisting}[language=SQL]
sqlite> SELECT "Ben" AS first, "Bitdiddle" AS last UNION
   ...> SELECT "Louis",        "Reasoner";
Ben|Bitdiddle
Louis|Reasoner
\end{lstlisting}

\begin{blocksection}
To save a table, use \texttt{CREATE TABLE} and a name.
Here we're going to create the table of employees from the
previous section and assign it to the name \texttt{records}:

\begin{lstlisting}[language=SQL]
sqlite> CREATE TABLE records AS
   ...>   SELECT "Ben Bitdiddle" AS name, "Computer" AS division,
   ...>     "Wizard" AS title, 60000 AS salary,
   ...>     "Oliver Warbucks" AS supervisor UNION
   ...>   SELECT "Alyssa P Hacker", "Computer",
   ...>     "Programmer", 40000, "Ben Bitdiddle" UNION ... ;
\end{lstlisting}
\end{blocksection}

We can SELECT specific values from an existing table using a \texttt{FROM} 
clause. This query creates a table with two columns, with a row for each row in the
\texttt{records} table:

\begin{lstlisting}[language=SQL]
sqlite> SELECT name, division FROM records;
Alyssa P Hacker|Computer
Ben Bitdiddle|Computer
Cy D Fect|Computer
Eben Scrooge|Accounting
Lem E Tweakit|Computer
Louis Reasoner|Computer
Oliver Warbucks|Administration
Robert Cratchet|Accounting
\end{lstlisting}

The special syntax \texttt{SELECT *} will select all columns from a table. It's
an easy way to print the contents of a table.

\begin{lstlisting}[language=SQL]
sqlite> SELECT * FROM records;
Alyssa P Hacker|Computer|Programmer|40000|Ben Bitdiddle
Ben Bitdiddle|Computer|Wizard|60000|Oliver Warbucks
Cy D Fect|Computer|Programmer|35000|Ben Bitdiddle
Eben Scrooge|Accounting|Chief Accountant|75000|Oliver Warbucks
Lem E Tweakit|Computer|Technician|25000|Ben Bitdiddle
Louis Reasoner|Computer|Programmer Trainee|30000|Alyssa P Hacker
Oliver Warbucks|Administration|Big Wheel|150000|Oliver Warbucks
Robert Cratchet|Accounting|Scrivener|18000|Eben Scrooge
\end{lstlisting}

We can choose which columns to show in the first part of the \texttt{SELECT}, we
can filter out rows using a \texttt{WHERE} clause, and sort the resulting rows
with an \texttt{ORDER BY} clause. In general the syntax is:

\begin{lstlisting}[language=SQL]
SELECT [columns] FROM [tables]
  WHERE [condition] ORDER BY [criteria];
\end{lstlisting}

For instance, the following statement lists all information about employees with
the ``Programmer'' title.

\begin{lstlisting}[language=SQL]
sqlite> SELECT * FROM records WHERE title = "Programmer";
Alyssa P Hacker|Computer|Programmer|40000|Ben Bitdiddle
Cy D Fect|Computer|Programmer|35000|Ben Bitdiddle
\end{lstlisting}

The following statement lists the names and salaries of each employee under the
accounting division, sorted in \textbf{descending} order by their salaries.

\begin{lstlisting}[language=SQL]
sqlite> SELECT name, salary FROM records
   ...>   WHERE division = "Accounting" ORDER BY -salary;
Eben Scrooge|75000
Robert Cratchet|18000
\end{lstlisting}

Note that all valid SQL statements must be terminated by a semicolon
(\texttt{;}). Additionally, you can split up your statement over many lines and
add as much whitespace as you want, much like Scheme. But keep in mind that
having consistent indentation and line breaking does make your code a lot more
readable to others (and your future self)!
