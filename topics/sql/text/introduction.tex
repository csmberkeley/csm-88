In SQL, data is organized into \textit{tables}.  A table has a fixed number of
named \define{columns}. A \define{row} of the table represents a single data
record and has one \define{value} for each column.  For example, we have a table
named \texttt{records} that stores information about the employees at a small
company\footnote{Example adapted from Structure and Interpretation of Computer
Programs}. Each of the eight rows represents an employee.


\begin{center}
    \texttt{records}\\
    \begin{tabular}{ l l l l l }
        \textbf{Name} & \textbf{Division} & \textbf{Title} & \textbf{Salary} & \textbf{Supervisor} \\
        \hline
        Ben Bitdiddle   &  Computer       & Wizard             & 60000  & Oliver Warbucks \\
        Alyssa P Hacker &  Computer       & Programmer         & 40000  & Ben Bitdiddle   \\
        Cy D Fect       &  Computer       & Programmer         & 35000  & Ben Bitdiddle   \\
        Lem E Tweakit   &  Computer       & Technician         & 25000  & Ben Bitdiddle   \\
        Louis Reasoner  &  Computer       & Programmer Trainee & 30000  & Alyssa P Hacker \\
        Oliver Warbucks &  Administration & Big Wheel          & 150000 & Oliver Warbucks \\
        Eben Scrooge    &  Accounting     & Chief Accountant   & 75000  & Oliver Warbucks \\
        Robert Cratchet &  Accounting     & Scrivener          & 18000  & Eben Scrooge
    \end{tabular}
\end{center}

\begin{solution}[0.1in]
\href{https://youtu.be/OyTnjS94EE8}{Video walkthrough}
\end{solution}
